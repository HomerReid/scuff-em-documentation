\documentclass[letterpaper]{article}

%********************************************************
%* latex shortcuts used for libscuff documentation files
%* 
%* homer reid -- 1994--2011
%**************************************************
\usepackage{graphicx}
\usepackage{color}
\usepackage{dsfont}
\usepackage{bbm}
\usepackage{amsmath}
\usepackage{amssymb}
\usepackage{float}
\usepackage{psfrag}
\usepackage{mathdots}
%\usepackage{algorithm}
%\usepackage{algorithmic}
%\usepackage{listings}
\usepackage{array}
\usepackage{url}
\usepackage{arydshln}
\usepackage{fancybox}
\usepackage{fancyvrb}
\usepackage{verbatim}

%--------------------------------------------------
%- boldface greek letters 
%--------------------------------------------------
\newcommand\vbphi{\mathbf{\phi}}
\newcommand\vbPhi{\mathbf{\Phi}}
\newcommand{\vbDelta}{\boldsymbol{\Delta}}
\newcommand{\vbrho}{\boldsymbol{\rho}}
\newcommand{\vbchi}{\boldsymbol{\chi}}

%--------------------------------------------------
%- colors -----------------------------------------
%--------------------------------------------------
\newcommand{\red}[1]{\textcolor{red}{#1}}
\newcommand{\blue}[1]{\textcolor{blue}{#1}}
\newcommand{\green}[1]{\textcolor{green}{#1}}
\newcommand{\atan}{\text{atan}}

%--------------------------------------------------
% general commands
%--------------------------------------------------
\newcommand\Tr{\hbox{Tr }}
\newcommand\sups[1]{^{\hbox{\scriptsize{#1}}}}
\newcommand\supt[1]{^{\hbox{\tiny{#1}}}}
\newcommand\subs[1]{_{\hbox{\scriptsize{#1}}}}
\newcommand\subt[1]{_{\hbox{\tiny{#1}}}}
\newcommand{\nn}{\nonumber \\}
\newcommand{\vb}[1]{\mathbf{#1}}
\newcommand{\numeq}[2]{\begin{equation} #2 \label{#1} \end{equation}}
\newcommand{\pard}[2]{\frac{\partial #1}{\partial #2}}
\newcommand{\pardn}[3]{\frac{\partial^{#1} #2}{\partial #3^{#1}}}
\newcommand{\pf}[2]{\left(\frac{#1}{#2}\right)}
\newcommand{\pp}{{\prime\prime}}
\newcommand{\vbhat}[1]{\vb{\hat #1}}
\newcommand{\mc}[1]{\mathcal{#1}}
\newcommand{\bmc}[1]{\boldsymbol{\mathcal{#1}}}
\newcommand{\mb}[1]{\mathbb{#1}}
\newcommand{\ds}[1]{\displaystyle{#1}}
\newcommand{\primedsum}{\sideset{}{'}{\sum}}
\newcommand{\pan}{\mathcal{P}}

%--------------------------------------------------
%- special libscuff stuff--------------------------
%--------------------------------------------------
\newcommand\ls{{\sc libscuff}}
\newcommand\lss{{\sc libscuff\,}}
\newcommand{\BG}{\boldsymbol{\Gamma}}
\newcommand{\MInt}{\vb{\hat M}}
\newcommand{\NInt}{\vb{\hat N}}
\newcommand{\MExt}{\vb{\check M}}
\newcommand{\NExt}{\vb{\check N}}
\newcommand{\xInt}{\vb{\hat x}}
\newcommand{\xExt}{\vb{\check x}}

%--------------------------------------------------
%-- superscripts for dyadic green's functions 
%--------------------------------------------------
\newcommand{\EEe}{^{\hbox{\tiny{EE}\scriptsize{,$e$}}}}
\newcommand{\MEe}{^{\hbox{\tiny{ME}\scriptsize{,$e$}}}}
\newcommand{\EMe}{^{\hbox{\tiny{EM}\scriptsize{,$e$}}}}
\newcommand{\MMe}{^{\hbox{\tiny{MM}\scriptsize{,$e$}}}}
\newcommand{\EEr}{^{\hbox{\tiny{EE}\scriptsize{,$r$}}}}
\newcommand{\MEr}{^{\hbox{\tiny{ME}\scriptsize{,$r$}}}}
\newcommand{\EMr}{^{\hbox{\tiny{EM}\scriptsize{,$r$}}}}
\newcommand{\MMr}{^{\hbox{\tiny{MM}\scriptsize{,$r$}}}}
\newcommand{\incr}{^{\text{\scriptsize inc},r}}
\newcommand{\incro}{^{\text{\scriptsize inc},r_1}}
\newcommand{\incrt}{^{\text{\scriptsize inc},r_2}}

%%%%%%%%%%%%%%%%%%%%%%%%%%%%%%%%%%%%%%%%%%%%%%%%%%%%%%%%%%%%%%%%%%%%%%
%%%%%%%%%%%%%%%%%%%%%%%%%%%%%%%%%%%%%%%%%%%%%%%%%%%%%%%%%%%%%%%%%%%%%%
%%%%%%%%%%%%%%%%%%%%%%%%%%%%%%%%%%%%%%%%%%%%%%%%%%%%%%%%%%%%%%%%%%%%%%
\newcommand{\MMExt}{\check{\mathcal{M}}}
\newcommand{\MMInt}{\hat{\mathcal{M}}}
\newcommand{\NNExt}{\check{\mathcal{N}}}
\newcommand{\NNInt}{\hat{\mathcal{N}}}
\newcommand{\BMMExt}{\boldsymbol{\check{\mathcal{M}}}}
\newcommand{\BMMInt}{\boldsymbol{\hat{\mathcal{M}}}}
\newcommand{\BNNExt}{\boldsymbol{\check{\mathcal{N}}}}
\newcommand{\BNNInt}{\boldsymbol{\hat{\mathcal{N}}}}
\newpage

%--------------------------------------------------
%- inner products and operator matrix elements  ---
%--------------------------------------------------
\newcommand{\expval}[1]{ \big\langle #1 \big\rangle}
\newcommand{\Expval}[1]{ \Big\langle #1 \Big\rangle}
\newcommand{\inp}[2]{ \big\langle #1 \big| #2 \big\rangle}
\newcommand{\Inp}[2]{ \Big\langle #1 \Big| #2 \Big\rangle}
\newcommand{\INP}[2]{ \bigg\langle #1 \bigg| #2 \Big\rangle}
\newcommand{\vmv}[3]{ \big\langle #1 \big| #2 \big| #3 \big\rangle}
\newcommand{\VMV}[3]{ \Big\langle #1 \Big| #2 \Big| #3 \Big\rangle}

%------------------------------------------------------------
%- shaded box for source code inclusions 
%------------------------------------------------------------
%\definecolor{lightgrey}{rgb}{0.85,0.85,0.85}
%\newcommand{\SourceBox}[1]
% { \begin{mdframed}[backgroundcolor=lightgrey, linewidth=2pt,
%                    tikzsetting={draw=black, line width=2pt,
%                                 dashed, dash pattern= on 5pt off 5pt}] %
%  \begin{verbatim} #1 \end{verbatim} \end{mdframed} \medskip
%}
%\IfFileExists{mdframed.sty}
% { \usepackage{mdframed}
%   \surroundwithmdframed[backgroundcolor=lightgrey, linewidth=2pt,
%                         framemethod=tikz,
%                         skipabove=10pt,
%                         skipbelow=10pt,
%                         tikzsetting={draw=black, line width=2pt,
%                                      dashed, dash pattern= on 5pt off 5pt}
%                        ]{verbatim}
% }
% {
% }
%--------------------------------------------------
%- shaded verbatim for code inclusions
%--------------------------------------------------
\definecolor{lightgrey}{rgb}{0.85,0.85,0.85}
\newenvironment{verbcode}{\VerbatimEnvironment% 
  \noindent
  %{\columnwidth-\leftmargin-\rightmargin-2\fboxsep-2\fboxrule-4pt} 
  \begin{Sbox} 
  \begin{minipage}{\linewidth}
  \begin{Verbatim}
}{% 
  \end{Verbatim}  
   \end{minipage}
  \end{Sbox} 
  \fcolorbox{black}{lightgrey}{\textcolor{black}{\TheSbox}}
} 


\graphicspath{{figures/}}

%------------------------------------------------------------
%------------------------------------------------------------
%- Special commands for this document -----------------------
%------------------------------------------------------------
%------------------------------------------------------------

%------------------------------------------------------------
%------------------------------------------------------------
%- Document header  -----------------------------------------
%------------------------------------------------------------
%------------------------------------------------------------
\title {{\sc scuff-caspol} Implementation Notes}
\author {Homer Reid}
\date {November 12, 2015}

%------------------------------------------------------------
%------------------------------------------------------------
%- Start of actual document
%------------------------------------------------------------
%------------------------------------------------------------

\begin{document}
\pagestyle{myheadings}
\markright{Homer Reid: {\sc scuff-caspol} Implementation Notes}
\maketitle

\tableofcontents

%%%%%%%%%%%%%%%%%%%%%%%%%%%%%%%%%%%%%%%%%%%%%%%%%%%%%%%%%%%%%%%%%%%%%%
%%%%%%%%%%%%%%%%%%%%%%%%%%%%%%%%%%%%%%%%%%%%%%%%%%%%%%%%%%%%%%%%%%%%%%
%%%%%%%%%%%%%%%%%%%%%%%%%%%%%%%%%%%%%%%%%%%%%%%%%%%%%%%%%%%%%%%%%%%%%%
\section{Overview}
\newcommand{\vbAlpha}{\boldsymbol{\alpha}}
\newcommand{\vbGamma}{\boldsymbol{\Gamma}}

The Casimir-Polder potential felt by a polarizable
particle at a point $\vb x$ in the vicinity of material
bodies is 
%====================================================================%
\begin{subequations}
\begin{align}
U(\vb x) &= \int_0^\infty  \mc U(\xi, \vb x) \, d\xi
\\
\mc U(\xi, \vb x) 
 &= 2\hbar \xi^2
    \text{Tr }\vbAlpha(\xi) \cdot \bmc G\supt{EE}(\xi; \vb x, \vb x)
\end{align}
\label{CPP}
\end{subequations}
%====================================================================%
Here $\vbAlpha(\xi)$ is the $3\times 3$ polarizability tensor
of the particle evaluated at imaginary angular frequency $\omega=i\xi$
and $\bmc G\sups{EE}(\xi, \vb x, \vb x)$ is the scattering part
of the $3\times 3$ electric-electric dyadic Green's function of 
the material geometry, defined by
%====================================================================%
\numeq{DGFDef}
{
 \mc G_{ij}\sups{E}(\xi; \vb x, \vb x^\prime)
   \equiv
  -\frac{1}{\kappa Z_0}
   \left( \quad
   \parbox{0.65\textwidth}
    { $i$-component of scattered $\vb E$-field at $\vb x$
      due to a unit-strength $j$-directed point electric
      dipole radiator
      at $\vb x^\prime$, all quantities having time dependence
      $\sim e^{+\xi t}$
    }
   \quad \right)
}
%====================================================================%
where $Z_0\approx 377\,\Omega$ is the impedance of free
space and $\kappa = \frac{\xi}{c}$ is the imaginary wavenumber.

$\mc G$ is computed numerically by solving a scattering problem 
in which the incident field arises from a $j$-directed point dipole 
at $\vb x_0$,
%====================================================================%
\numeq{IncidentFields}
{
   E_i\sups{inc}(\vb x) = \Gamma\supt{EE}_{ij}(\vb x, \vb x_0),
   \qquad 
   H_i\sups{inc}(\vb x) = \Gamma\supt{ME}_{ij}(\vb x, \vb x_0)
}
%====================================================================%
where the tensor $\vbGamma\supt{PQ}$ gives the P field due to
a Q source (with \{P,Q\}$\in$\{electric, magnetic\}):
%====================================================================%
$$ \vbGamma\supt{EE} = -\kappa Z_0 \mb G, \qquad 
   \vbGamma\supt{EM} = -\kappa     \mb C, \qquad 
   \vbGamma\supt{ME} = +\kappa     \mb C, \qquad 
   \vbGamma\supt{MM} = -\frac{\kappa}{Z_0} \mb G
$$
$$ \mb G_{ij}(\vb r)=
   \left(\delta_{ij} - \frac{1}{\kappa^2}\partial_i \partial_j\right) G_0(\vb r),
   \qquad 
   \mb C_{ij}(\vb r)=-\frac{1}{\kappa}\varepsilon_{ijk}\partial_k G_0(\vb r),
   \qquad
   G_0(\vb r)=\frac{e^{-\kappa r}}{4\pi r}.
$$
%====================================================================%
Given this incident field, we get one full column of $\mc G\supt{EE}$ 
by evaluating the components of the scattered field at $x_0$:
%====================================================================%
\numeq{ScatteredFields}
{
 \mc G\supt{EE}_{ij}(\xi, \vb x_0, \vb x_0) = E\sups{scat}_i(\vb x_0).
}
%====================================================================%

\subsubsection*{Implementation in {\sc scuff-em}}

In {\sc scuff-em}, the scattering problem becomes a linear 
system of the form 
%====================================================================%
$$ \vb M \vb c = -\vb f $$
%====================================================================%
where $\vb c$ and $\vb f$ are respectively the vectors of 
surface-current coefficients and incident-field projections:
%====================================================================%
$$ \vb c = 
   \left(\begin{array}{c} \vb k \\ -\vb n/Z_0 \end{array}\right)
   ,\qquad 
   \vb f =
   \left(\begin{array}{c} \vb e/Z_0 \\ \vb h \end{array}\right).
$$
%====================================================================%
$$ \vb K(\vb x)=\sum_{\alpha} k_\alpha \vb b_\alpha(\vb x), \qquad 
   \vb N(\vb x)=\sum_{\alpha} n_\alpha \vb b_\alpha(\vb x)
$$
%====================================================================%
$$ e_\alpha = \expval{\vb b_\alpha \cdot \vb E\sups{inc}},
   \qquad 
   h_\alpha = \expval{\vb b_\alpha \cdot \vb H\sups{inc}}
$$
%====================================================================%
In the case at hand, the elements of the RHS vector are
%====================================================================%
\begin{align*}
  e_\alpha/Z_0
 &= \frac{1}{Z_0} 
     \Expval{b_{\alpha;\mu}(\vb x) \Gamma\supt{EE}_{\mu j}(\vb x, \vb x_0)}
\\
 &= -\kappa \Expval{b_{\alpha;\mu}(\vb x) \mb G_{\mu j}(\vb x, \vb x_0)}
\\
  h_\alpha
 &= \Expval{b_{\alpha;i}(\vb x) \Gamma\supt{ME}_{\mu j}(\vb x, \vb x_0)}
\\
 &= +\kappa \Expval{b_{\alpha;\mu}(\vb x) \mb C_{\mu j}(\vb x, \vb x_0)}
\end{align*}
which I will write in the form
%====================================================================%
$$ -\vb f = -\left(\begin{array}{c} \vb e/Z_0 \\ \vb h \end{array}\right)
   = +\kappa \vb v_j
$$
$$ \vb v_j = \left(\begin{array}{c}
   \vb v\supt{E}_j \\  \vb v\supt{M}_j
   \end{array}\right)
$$
$$ \vb v\supt{E}_j 
    = \Expval{b_{\alpha;\mu}(\vb x) \mb G_{\mu j}(\vb x, \vb x_0)},
\qquad 
   \vb v\supt{M}_j 
    = -\Expval{b_{\alpha;\mu}(\vb x) \mb C_{\mu j}(\vb x, \vb x_0)}
$$
%====================================================================%
Having computed the surface-current expansion coefficients, the 
scattered fields at $\vb x$ are
%====================================================================%
\begin{align}
   E_{i}(\vb x_0) 
&= \sum_{\alpha} 
   \left\{
      k_\alpha
     \Expval{ \Gamma\supt{EE}_{i\mu}(\vb x_0, \vb x) b_{\alpha;\mu}(\vb x) }
     +n_\alpha
     \Expval{ \Gamma\supt{EM}_{i\mu}(\vb x_0, \vb x) b_{\alpha;\mu}(\vb x) }
   \right\}
\\
&= \sum_{\alpha}
   \left\{
      -\kappa Z_0
      k_\alpha
     \Expval{ \mb G_{i\mu}(\vb x_0, \vb x) b_{\alpha;\mu}(\vb x) }
 -\kappa n_\alpha
     \Expval{ \mb C_{i\mu}(\vb x_0, \vb x) b_{\alpha;\mu}(\vb x) }
   \right\}
\\
&=-\kappa Z_0 \sum_{\alpha}
   \left\{ k_\alpha v\supt{E}_{i\alpha}
                          - \frac{n_\alpha}{Z_0} v\supt{M}_{i\alpha}
  \right\}
\\
&=-\kappa Z_0 \vb v_i^T \vb c
\\
&=-\kappa^2 Z_0 \vb v_i^T \vb W \vb v_j
\end{align}
%====================================================================%
where $\vb W=\vb M^{-1}$. 
The dyadic Green's function (\ref{DGFDef}) then reads
%====================================================================%
\begin{align*}
 \bmc G_{ij}
&= \kappa \Big[ \vb v_{i}^T \vb W \vb v_{j}\Big]
\intertext{and the integrand of the Casimir-Polder potential (\ref{CPP}b)
reads ($c\equiv 1$)}
\mc U(\xi; \vb x)
&= 2\hbar \xi^3 
    \sum_{ij}\alpha_{ij} \Big[ \vb v_{j}^T \vb W \vb v_{i}\Big]
\\
&= 2\hbar \xi^3 
    \left[ \vb v_{j}^T \vb W \Big( \alpha_{ij} \vb v_{i}\Big)\right]
\\
\end{align*}
%====================================================================%

\end{document}
