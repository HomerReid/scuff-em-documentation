\documentclass[letterpaper]{article}

%********************************************************
%* latex shortcuts used for libscuff documentation files
%* 
%* homer reid -- 1994--2011
%**************************************************
\usepackage{graphicx}
\usepackage{color}
\usepackage{dsfont}
\usepackage{bbm}
\usepackage{amsmath}
\usepackage{amssymb}
\usepackage{float}
\usepackage{psfrag}
\usepackage{mathdots}
%\usepackage{algorithm}
%\usepackage{algorithmic}
%\usepackage{listings}
\usepackage{array}
\usepackage{url}
\usepackage{arydshln}
\usepackage{fancybox}
\usepackage{fancyvrb}
\usepackage{verbatim}

%--------------------------------------------------
%- boldface greek letters 
%--------------------------------------------------
\newcommand\vbphi{\mathbf{\phi}}
\newcommand\vbPhi{\mathbf{\Phi}}
\newcommand{\vbDelta}{\boldsymbol{\Delta}}
\newcommand{\vbrho}{\boldsymbol{\rho}}
\newcommand{\vbchi}{\boldsymbol{\chi}}

%--------------------------------------------------
%- colors -----------------------------------------
%--------------------------------------------------
\newcommand{\red}[1]{\textcolor{red}{#1}}
\newcommand{\blue}[1]{\textcolor{blue}{#1}}
\newcommand{\green}[1]{\textcolor{green}{#1}}
\newcommand{\atan}{\text{atan}}

%--------------------------------------------------
% general commands
%--------------------------------------------------
\newcommand\Tr{\hbox{Tr }}
\newcommand\sups[1]{^{\hbox{\scriptsize{#1}}}}
\newcommand\supt[1]{^{\hbox{\tiny{#1}}}}
\newcommand\subs[1]{_{\hbox{\scriptsize{#1}}}}
\newcommand\subt[1]{_{\hbox{\tiny{#1}}}}
\newcommand{\nn}{\nonumber \\}
\newcommand{\vb}[1]{\mathbf{#1}}
\newcommand{\numeq}[2]{\begin{equation} #2 \label{#1} \end{equation}}
\newcommand{\pard}[2]{\frac{\partial #1}{\partial #2}}
\newcommand{\pardn}[3]{\frac{\partial^{#1} #2}{\partial #3^{#1}}}
\newcommand{\pf}[2]{\left(\frac{#1}{#2}\right)}
\newcommand{\pp}{{\prime\prime}}
\newcommand{\vbhat}[1]{\vb{\hat #1}}
\newcommand{\mc}[1]{\mathcal{#1}}
\newcommand{\bmc}[1]{\boldsymbol{\mathcal{#1}}}
\newcommand{\mb}[1]{\mathbb{#1}}
\newcommand{\ds}[1]{\displaystyle{#1}}
\newcommand{\primedsum}{\sideset{}{'}{\sum}}
\newcommand{\pan}{\mathcal{P}}

%--------------------------------------------------
%- special libscuff stuff--------------------------
%--------------------------------------------------
\newcommand\ls{{\sc libscuff}}
\newcommand\lss{{\sc libscuff\,}}
\newcommand{\BG}{\boldsymbol{\Gamma}}
\newcommand{\MInt}{\vb{\hat M}}
\newcommand{\NInt}{\vb{\hat N}}
\newcommand{\MExt}{\vb{\check M}}
\newcommand{\NExt}{\vb{\check N}}
\newcommand{\xInt}{\vb{\hat x}}
\newcommand{\xExt}{\vb{\check x}}

%--------------------------------------------------
%-- superscripts for dyadic green's functions 
%--------------------------------------------------
\newcommand{\EEe}{^{\hbox{\tiny{EE}\scriptsize{,$e$}}}}
\newcommand{\MEe}{^{\hbox{\tiny{ME}\scriptsize{,$e$}}}}
\newcommand{\EMe}{^{\hbox{\tiny{EM}\scriptsize{,$e$}}}}
\newcommand{\MMe}{^{\hbox{\tiny{MM}\scriptsize{,$e$}}}}
\newcommand{\EEr}{^{\hbox{\tiny{EE}\scriptsize{,$r$}}}}
\newcommand{\MEr}{^{\hbox{\tiny{ME}\scriptsize{,$r$}}}}
\newcommand{\EMr}{^{\hbox{\tiny{EM}\scriptsize{,$r$}}}}
\newcommand{\MMr}{^{\hbox{\tiny{MM}\scriptsize{,$r$}}}}
\newcommand{\incr}{^{\text{\scriptsize inc},r}}
\newcommand{\incro}{^{\text{\scriptsize inc},r_1}}
\newcommand{\incrt}{^{\text{\scriptsize inc},r_2}}

%%%%%%%%%%%%%%%%%%%%%%%%%%%%%%%%%%%%%%%%%%%%%%%%%%%%%%%%%%%%%%%%%%%%%%
%%%%%%%%%%%%%%%%%%%%%%%%%%%%%%%%%%%%%%%%%%%%%%%%%%%%%%%%%%%%%%%%%%%%%%
%%%%%%%%%%%%%%%%%%%%%%%%%%%%%%%%%%%%%%%%%%%%%%%%%%%%%%%%%%%%%%%%%%%%%%
\newcommand{\MMExt}{\check{\mathcal{M}}}
\newcommand{\MMInt}{\hat{\mathcal{M}}}
\newcommand{\NNExt}{\check{\mathcal{N}}}
\newcommand{\NNInt}{\hat{\mathcal{N}}}
\newcommand{\BMMExt}{\boldsymbol{\check{\mathcal{M}}}}
\newcommand{\BMMInt}{\boldsymbol{\hat{\mathcal{M}}}}
\newcommand{\BNNExt}{\boldsymbol{\check{\mathcal{N}}}}
\newcommand{\BNNInt}{\boldsymbol{\hat{\mathcal{N}}}}
\newpage

%--------------------------------------------------
%- inner products and operator matrix elements  ---
%--------------------------------------------------
\newcommand{\expval}[1]{ \big\langle #1 \big\rangle}
\newcommand{\Expval}[1]{ \Big\langle #1 \Big\rangle}
\newcommand{\inp}[2]{ \big\langle #1 \big| #2 \big\rangle}
\newcommand{\Inp}[2]{ \Big\langle #1 \Big| #2 \Big\rangle}
\newcommand{\INP}[2]{ \bigg\langle #1 \bigg| #2 \Big\rangle}
\newcommand{\vmv}[3]{ \big\langle #1 \big| #2 \big| #3 \big\rangle}
\newcommand{\VMV}[3]{ \Big\langle #1 \Big| #2 \Big| #3 \Big\rangle}

%------------------------------------------------------------
%- shaded box for source code inclusions 
%------------------------------------------------------------
%\definecolor{lightgrey}{rgb}{0.85,0.85,0.85}
%\newcommand{\SourceBox}[1]
% { \begin{mdframed}[backgroundcolor=lightgrey, linewidth=2pt,
%                    tikzsetting={draw=black, line width=2pt,
%                                 dashed, dash pattern= on 5pt off 5pt}] %
%  \begin{verbatim} #1 \end{verbatim} \end{mdframed} \medskip
%}
%\IfFileExists{mdframed.sty}
% { \usepackage{mdframed}
%   \surroundwithmdframed[backgroundcolor=lightgrey, linewidth=2pt,
%                         framemethod=tikz,
%                         skipabove=10pt,
%                         skipbelow=10pt,
%                         tikzsetting={draw=black, line width=2pt,
%                                      dashed, dash pattern= on 5pt off 5pt}
%                        ]{verbatim}
% }
% {
% }
%--------------------------------------------------
%- shaded verbatim for code inclusions
%--------------------------------------------------
\definecolor{lightgrey}{rgb}{0.85,0.85,0.85}
\newenvironment{verbcode}{\VerbatimEnvironment% 
  \noindent
  %{\columnwidth-\leftmargin-\rightmargin-2\fboxsep-2\fboxrule-4pt} 
  \begin{Sbox} 
  \begin{minipage}{\linewidth}
  \begin{Verbatim}
}{% 
  \end{Verbatim}  
   \end{minipage}
  \end{Sbox} 
  \fcolorbox{black}{lightgrey}{\textcolor{black}{\TheSbox}}
} 


\graphicspath{{figures/}}

%------------------------------------------------------------
%------------------------------------------------------------
%- Special commands for this document -----------------------
%------------------------------------------------------------
%------------------------------------------------------------

%------------------------------------------------------------
%------------------------------------------------------------
%- Document header  -----------------------------------------
%------------------------------------------------------------
%------------------------------------------------------------
\title {Handling of material junctions in {\sc scuff-em}}
\author {Homer Reid}
\date {June 18, 2012}

%------------------------------------------------------------
%------------------------------------------------------------
%- Start of actual document
%------------------------------------------------------------
%------------------------------------------------------------

\begin{document}
\pagestyle{myheadings}
\markright{Homer Reid: Handling of Material Junctions in {\sc scuff-em}}
\maketitle

\tableofcontents

%%%%%%%%%%%%%%%%%%%%%%%%%%%%%%%%%%%%%%%%%%%%%%%%%%%%%%%%%%%%%%%%%%%%%%
%%%%%%%%%%%%%%%%%%%%%%%%%%%%%%%%%%%%%%%%%%%%%%%%%%%%%%%%%%%%%%%%%%%%%%
%%%%%%%%%%%%%%%%%%%%%%%%%%%%%%%%%%%%%%%%%%%%%%%%%%%%%%%%%%%%%%%%%%%%%%
\newpage
\section{Material Junctions}

This is distinct 

%%%%%%%%%%%%%%%%%%%%%%%%%%%%%%%%%%%%%%%%%%%%%%%%%%%%%%%%%%%%%%%%%%%%%%
%%%%%%%%%%%%%%%%%%%%%%%%%%%%%%%%%%%%%%%%%%%%%%%%%%%%%%%%%%%%%%%%%%%%%%
%%%%%%%%%%%%%%%%%%%%%%%%%%%%%%%%%%%%%%%%%%%%%%%%%%%%%%%%%%%%%%%%%%%%%%
\newpage
\section{Half-RWG Basis Functions}

The half-RWG (HRWG) basis function associated with 
an edge $\vb L_\alpha$ of a panel $\pan$ is
$$ \vb h_\alpha(\vb x) = 
   \begin{cases}
   \displaystyle{ \frac{l_\alpha}{2A_\alpha}(\vb x - \vb Q_\alpha)}, 
   \qquad & \vb x \in \pan 
   \\[5pt]
   0, \qquad & \text{otherwise}
   \end{cases}
$$
where $l_\alpha$ is the edge length, $\vb Q_\alpha$ is 
the vertex of $\pan$ opposite the edge, and $A_\alpha$ 
is the area of $\pan$.
Introducing the \textit{indicator function} $\chi_\pan(\vb x)$, 
which equals unity for $\vb x\in \pan$ and 0 otherwise, 
we can rewrite this in the form
$$ \vb h_\alpha(\vb x) = 
   \frac{l_\alpha}{2A_\alpha}(\vb x - \vb Q_\alpha)\chi_\pan(\vb x).
$$
The novelty of the HRWG basis function is that the charge density
associated with it has both a surface-charge portion 
(like the ordinary RWG basis function) and a new \textit{line-charge}
density on edge $\vb L_\alpha$. Suppose we have an HRWG function
$\vb h_\alpha$ populated with strength $k_\alpha$. (Recall $k_\alpha$
has units of current/length.) Then the surface and line charge
densities associated with this current distribution are 
\begin{align*}
  \sigma(\vb x)
 &=
 \frac{lk_\alpha}{i\omega A}\chi_\pan(\vb x)
\\
\textcolor{red}{ \lambda(\vb x)}
 &=
\textcolor{red}{-\frac{k_\alpha}{i\omega}\chi_{\vb L_\alpha}(\vb x)}
\end{align*}

%%%%%%%%%%%%%%%%%%%%%%%%%%%%%%%%%%%%%%%%%%%%%%%%%%%%%%%%%%%%%%%%%%%%%%
%%%%%%%%%%%%%%%%%%%%%%%%%%%%%%%%%%%%%%%%%%%%%%%%%%%%%%%%%%%%%%%%%%%%%%
%%%%%%%%%%%%%%%%%%%%%%%%%%%%%%%%%%%%%%%%%%%%%%%%%%%%%%%%%%%%%%%%%%%%%%
\newpage
\section{Matrix Elements involving HRWG Basis Functions}

\subsection{Matrix elements between RWG and HRWG functions}

Consider an RWG function $\vb f_\alpha$ and an HRWG function
$\vb h_\beta.$ $\vb f_\alpha$ describes surface currents 
on two panels $\pan_\alpha^\pm$, while $\vb h_\beta$ describes
surface currents on a single panel $\pan_\beta$
plus the line charge on an edge $\vb L_\beta$.
The BEM matrix elements between these two basis functions
involve the interactions
\begin{subequations}
\begin{align}
  \VMV{\vb f_\alpha}{\vb G}{\vb h_\beta}
&=
  \VMV{\vb f_\alpha}{\vb G}{\vb h_\beta}\sups{panel--panel}
+
  \VMV{\vb f_\alpha}{\vb G}{\vb h_\beta}\sups{panel--edge}
\\
  \VMV{\vb f_\alpha}{\vb C}{\vb h_\beta}
&=
  \VMV{\vb f_\alpha}{\vb C}{\vb h_\beta}\sups{panel--panel}
\end{align}
\label{fGCh}
\end{subequations}
The first terms here are just the usual panel-panel
interactions between currents on panels $\pan_{\alpha}^\pm$ 
and $\pan_\beta$, and may be computed using the usual
\lss methods for panel-panel interactions. 

The second term in (\ref{fGCh}a) is a new interaction
between the surface charge densities on $\pan_{\alpha}^\pm$ 
and the line charge density on $\vb L_\beta$:
\numeq{FGHPanelEdge}
{
  \VMV{\vb f_\alpha}{\vb G}{\vb h_\beta}\sups{panel--edge}
=
-\frac{1}{k^2}
   \sum \pm
   \int_{\pan_\alpha^\pm} d\vb x
   \int_{\vb L_\beta} d\vb x^\prime
   \left\{ \Big[\nabla \cdot \vb f_\alpha(\vb x)\Big]
           G(\vb x, \vb x^\prime)
           \Big[\nabla \cdot \vb h_\alpha(\vb x^\prime)\Big]
   \right\}
}
Note that there is no panel-edge contribution to 
$\vmv{\vb f}{\vb C}{\vb h}.$

\subsection{Matrix elements between HRWG and HRWG functions}

Next consider an HRWG function $\vb h_\alpha$ and an HRWG function
$\vb h_\beta.$ We have 
\begin{subequations}
\begin{align}
  \VMV{\vb h_\alpha}{\vb G}{\vb h_\beta}
&=
  \VMV{\vb h_\alpha}{\vb G}{\vb h_\beta}\sups{panel--panel}
+
  \VMV{\vb h_\alpha}{\vb G}{\vb h_\beta}\sups{panel--edge}
\\
  \VMV{\vb h_\alpha}{\vb C}{\vb h_\beta}
&=
  \VMV{\vb h_\alpha}{\vb C}{\vb h_\beta}\sups{panel--panel}
\end{align}
\label{hGCh}
\end{subequations}
As above, the panel-panel interactions may be computed using 
the usual \lss methods for panel-panel interactions.

The second term in (\ref{hGCh}a) is similar to (\ref{fGhPanelEdge}),
but without the sum over positive and negative panels, since
there is only one panel in the support of $\vb h_\alpha$:
\begin{align*}
\VMV{\vb h_\alpha}{\vb G}{\vb h_\beta}\sups{panel--edge}
= -&\frac{1}{k^2}
   \int_{\pan_\alpha} d\vb x
   \int_{\vb L_\beta} d\vb x^\prime
   \left\{ \Big[\nabla \cdot \vb h_\alpha(\vb x)\Big]
           G(\vb x, \vb x^\prime)
           \Big[\nabla \cdot \vb h_\alpha(\vb x^\prime)\Big]
   \right\}
\end{align*}

\subsection*{Nonsingular Edge-Panel Interactions}

\begin{align*}
\VMV{\vb f_\alpha}{\vb G}{\vb h_\beta}\sups{panel--edge}
&=-\frac{1}{k^2}\sum\pm \frac{l_\alpha}{A_\alpha^\pm}
   \int_{\pan_\alpha^\pm} d\vb x \int_{\vb L_\beta} d\vb x^\prime
   G(\vb x, \vb x^\prime)
\end{align*}
Consider one of the two integrals here 
[Figure \ref{EdgePanelInteractionFigure}\textbf{(a)}]:
\begin{align}
&
 -\frac{l_\alpha}{Ak^2 }
   \int_{\pan} d\vb x \int_{\vb L_\beta} d\vb x^\prime
   G(\vb x, \vb x^\prime)
\nn
&\quad=
 -\frac{2l_\alpha l_\beta}{k^2 }
  \int_0^1 du \, \int_0^u dv \, \int_0^1 du^\prime
   G\Big[\vb x\big(u,v\big), \vb x^\prime\big(u^\prime\big)\Big]
\label{EdgePanelCubature}
\end{align}
$$ \vb x = \vb V_0 + u\vb A + v\vb B, \qquad 
   \vb x^\prime = \vb V_0^\prime + u^\prime\vb L
$$
This integral is easy to evaluate via three-dimensional
numerical cubature.

\subsection*{Singular Edge-Panel Interactions: The Common Vertex Case}

If the edge or panel share a single vertex 
[Figure \ref{EdgePanelInteractionFigure}\textbf{(bc)}],
then the integral is singular and we have to be more clever.
In this case we parameterize points in $\pan$ and $\vb L$ 
according to
$$ \vb x = \vb V_0 + u\vb A + v\vb B
   \qquad
   \vb x^\prime = \vb V_0 + u^\prime \vb L,
$$
and it is shown in Appendix \ref{EdgePanelAppendix}
that the singular three-dimensional integral in (\ref{EdgePanelCubature})
may be reduced to a nonsingular two-dimensional integral:
\begin{align}
&\hspace{-2in}
 -\frac{l_\alpha}{Ak^2}
   \int_{\pan} d\vb x \int_{\vb L_\beta} d\vb x^\prime
   G(\vb x, \vb x^\prime)
\\
 =
 -\frac{l_\alpha l_\beta}{4\pi k^2}\int_0^1 dx \int_0^1 dy \bigg\{
 \quad &\frac{x}{R_1} e^{ikR_1} \cdot \text{ExpRel}(2,-ikR_1)
\\
      +&\frac{(1-x)}{R_2} e^{ikR_2} \cdot \text{ExpRel}(2,-ikR_2)
\\
      +&\frac{x}{R_3}e^{ikR_3} \cdot \text{ExpRel}(2,-ikR_3)
  \bigg\}
\end{align}
with
$$ R_1(x,y)=\Big| x\vb A + xy \vb B - \vb L \Big|,
   \qquad
   R_2(x,y)=\Big| \vb A + (1-x)y \vb B - x\vb L\Big|
   \qquad
   R_3(x,y)=\Big| \vb A + (1-xy) \vb B - x\vb L\Big|
$$
and
%====================================================================%
\begin{align*}
 \text{ExpRel}(N,X) \equiv
 \frac{N!}{X^N}\Big[e^{X} - 1 - X - \frac{X^2}{2!} - \cdots 
                             - \frac{X^{N-1}}{(N-1)!} \Big].
\end{align*}
%====================================================================%

\appendix
\newpage
\section{Taylor-Duffy transformation of edge-panel integrals}
\label{EdgePanelAppendix}

\subsection{The common vertex case}

In this section we justify equation 
(\ref{TaylorDuffy}).

The integration region for the integral on the LHS of 
(\ref{TaylorDuffy}) is the direct product of a unit
right triangle in the $(u,v)$ plane with the unit 
interval on the $u^\prime$ axis, i.e. 
$$\text{region of integration}=\mathcal{R}=
  \begin{cases}
  0\le u \le 1 \\
  0\le v \le u \\
  0\le u^\prime \le 1.
  \end{cases}
$$

\subsection*{Step 1. Decompose region of integration into tetrahedra}

Following the spirit of the Taylor-Duffy method, we 
decompose $\mathcal{R}$ as the nonintersecting union 
of three tetrahedral regions:
$$ \mathcal{R} = \mathcal{R}_1 \cup
                 \mathcal{R}_2 \cup
                 \mathcal{R}_3
$$
where 
\begin{align*}
  \mathcal{R}_1 = 
  \begin{cases}
  0 \le u^\prime \le 1 \\
  0 \le u \le u^\prime \\
  0 \le v \le u
  \end{cases},
\qquad
  \mathcal{R}_2 = 
  \begin{cases}
  0 \le u \le 1 \\
  0 \le u^\prime \le u \\
  0 \le v \le u-u^\prime
  \end{cases}
\qquad
  \mathcal{R}_3 = 
  \begin{cases}
  0 \le u \le 1 \\
  0 \le u^\prime \le u \\
  u-u^\prime \le v \le u.
  \end{cases}
\end{align*}

\subsection*{Step 2. Introduce Duffy-transformed variables} 

The next step is to introduce Duffy-transformed variables for each 
of the three integration regions. The idea of the Duffy transform
is to single out one of the three variables as the ``independent''
variable (call it $w$), and then to ``measure the other two variables
in units of $w$.'' This ensures that the transformed versions of 
all three of the original variables $u,v,u^\prime$ are now expressed
as multiples of $w$, whereupon a factor of $w$ may be pulled out 
of radicals like $\sqrt{Au^2 + Bv^2 + Cu^{\prime 2}}$. When such radicals
appear in the denominator of the integrand, we then get overall 
multiplicative factors of $w$ in the denominator, which cancel with
the Jacobian of the variable transformation to yield 

It is also convenient, although not strictly necessary, to define our
Duffy-transformed variables $w,x,y$ in such a way that the region of
integration for all three of the transformed integrals is the unit
three-dimensional cube (i.e. $w,x,y$ all running from 0 to 1.) This 
allows the three integrals to be combined into a \textit{single} integral 
over the unit cube.

Such a variable transformation is 

%====================================================================%
\begin{align*}
\mathcal{R}_1:\qquad
    &u^\prime=w, \qquad u=wx, \qquad v=wxy, \quad \qquad \mathcal{J}=w^2 x
\\
\mathcal{R}_2: \qquad 
    &u=w, \qquad u^\prime=wx, \qquad v=w(1-x)y, \qquad \mathcal{J}=w^2 (1-x)
\\
\mathcal{R}_3: \qquad 
    &u=w, \qquad u^\prime=wx, \qquad v=w-wxy, \qquad \mathcal{J}= w^2x
\end{align*}
%====================================================================%
and our integral in terms of the transformed variables becomes
%====================================================================%
\begin{align}
\int_0^1 du \int_0^u dv \int_0^1 du^\prime f(u,v,u^\prime)
\nn
=\int_0^1 dw \int_0^1 dx \int_0^1 dy 
  \bigg\{\quad & w^2 x f\Big(wx, wxy, w\Big)
\nn
       + &w^2(1-x) f\Big(w, w(1-x)y, wx\Big)
\nn
       + &w^2x  f\Big(w, w(1-xy), wx\Big)
  \bigg\}
\label{TransformedEdgePanelIntegral1}
\end{align}
%====================================================================%

\subsection*{Step 3. Evaluate the $w$ integral}

The integrand in question here is 
\begin{align*}
 f(u,v,u^\prime)
&=\frac{e^{ikR(u,v,u^\prime)}}{4\pi R(u,v,u^\prime)},
\qquad 
 R(u,v,u^\prime)=\Big|u\vb A + v\vb B - \vb u^\prime \vb L\Big|
\end{align*}
(where $\vb A=\vb L$ for the shared-edge case).
Applying (\ref{TransformedEdgePanelIntegral1}), we have
%====================================================================%
\begin{align}
&\int_0^1 du \int_0^u dv \int_0^1 du^\prime f(u,v,u^\prime)
\nn
&=\frac{1}{4\pi}\int_0^1 dx \int_0^1 dy \int_0^1 dw
\Big[  \frac{ w x     e^{ikwR_1(x,y)}}{R_1(x,y)} 
     + \frac{ w (1-x) e^{ikwR_2(x,y)}}{R_2(x,y)} 
     + \frac{ w x     e^{ikwR_3(x,y)}}{R_3(x,y)}
\Big] 
\label{TransformedEdgePanelIntegral2}
\end{align}
$$ R_1(x,y)=\Big| x\vb A + xy \vb B - \vb L \Big|,
   \qquad
   R_2(x,y)=\Big| \vb A + (1-x)y \vb B - x\vb L \Big|
   \qquad
   R_3(x,y)=\Big| \vb A + (1-xy) \vb B - x\vb L \Big|.
$$
%====================================================================%
The $w$ integral may be evaluated in closed form:
\numeq{WIntegral}
{\int_0^1 w e^{wX} dw  = \frac{1}{2}e^X \cdot \text{ExpRel}(2,-X)}
where $\text{ExpRel}(N,X)$ is 
$e^X$, minus the first $N$ terms in its series expansion,
normalized so that $\text{ExpRel(N,0)}=1$:
\begin{align*}
 \text{ExpRel}(N,X) &\equiv
   \frac{N!}{X^N}\Big[e^{X} - 1 - X - \frac{X^2}{2!} - \cdots 
                             - \frac{X^{N-1}}{(N-1)!} \Big]
\\
&=
   \frac{N!}{X^N}\Big[\frac{X^N}{N!} + \frac{X^{N+1}}{(N+1)^1} + \cdots\Big] 
\\
&=
   1 + \frac{X}{N+1} + \frac{X^2}{(N+1)(N+2)} + \cdots
\end{align*}

Using (\ref{WIntegral}), equation (\ref{TransformedEdgePanelIntegral2})
becomes 
\begin{align*}
=\frac{1}{8\pi}\int_0^1 dx \int_0^1 dy \bigg\{
 \quad &\frac{x}{R_1} e^{ikR_1} \cdot \text{ExpRel}(2,-ikR_1)
\\
      +&\frac{(1-x)}{R_2} e^{ikR_2} \cdot \text{ExpRel}(2,-ikR_2)
\\
      +&\frac{x}{R_3}e^{ikR_3} \cdot \text{ExpRel}(2,-ikR_3)
  \bigg\}
\end{align*}
where $R_1, R_2, R_3$ are functions of $x$ and $y$. This is a nonsingular
integral over the unit square which may be evaluated by straighforward
numerical cubature.

%%%%%%%%%%%%%%%%%%%%%%%%%%%%%%%%%%%%%%%%%%%%%%%%%%%%%%%%%%%%%%%%%%%%%%
%%%%%%%%%%%%%%%%%%%%%%%%%%%%%%%%%%%%%%%%%%%%%%%%%%%%%%%%%%%%%%%%%%%%%%
%%%%%%%%%%%%%%%%%%%%%%%%%%%%%%%%%%%%%%%%%%%%%%%%%%%%%%%%%%%%%%%%%%%%%%
\subsection{The common edge case}

In this case we parameterize points in the triangle and on the edge according to 
$$\vb x=\vb V_0 + u\vb A + v \vb B, \qquad \vb x^\prime = \vb V_0 + u\vb A.$$
The edge-panel integral is 
\numeq{EPICE1}
{
   \texttt{EPI} \equiv 
   \int_0^1 du^\prime \, \int_0^1 du \, \int_0^u dv \, 
   \mathcal{I}(u-u^\prime,v) 
}
$$ \mathcal{I}(u-u^\prime, v) = 
   \frac{e^{ikR(u-u^\prime,v)}}{4\pi R(u-u^\prime,v)}, 
   \qquad 
   R(u-u^\prime,v)=\Big|(u-u^\prime)\vb A + v\vb B\Big| 
$$
We begin by rewriting the integral over the triangle in the $u,v$ 
plane: for any integrand function $F(u,v)$, we have 
$$ \int_0^1 du \, \int_0^u dv \, F(u,v)
   =
   \int_0^1 dv \, \int_v^1 du \, F(u,v)
$$
Applying this to (\ref{EPICE1}) yields 
\numeq{EPICE2}
{ \texttt{EPI} = 
  \int_0^1 dv \, \int_0^1 du^\prime \, \int_y^1 du \, \mathcal{I}(u-u^\prime,v).
}
Now we rewrite the $u,u^\prime$ integral to take advantage of the 
fact that the integrand only depends on the difference $u-u^\prime$.
Putting $s=(u+u^\prime), t=u-u^\prime,$ the result we need
(again valid for arbitrary integrand functions $g(u,u^\prime)$ 
is 
\begin{align*}
&
 \int_0^1 du^\prime\, \int_v^1 du \, g(u,u^\prime)
\\
&=\frac{1}{2}\Bigg\{     \int_{v-1}^0 dt \, \int_{2v-t}^{2}   ds \, 
                   \,+\, \int_0^v     dt \, \int_{2v-t}^{2-t} ds \, 
                   \,+\, \int_v^1     dt \, \int_{t}^{2-t}    ds
             \Bigg\}
             g\Big( \frac{1}{2}(s+t), \frac{1}{2}(s-t)\Big).
\end{align*}
Applying this to (\ref{EPICE2}) and evaluating the $s$ integrals 
allows us to express the EPI as a sum of three two-dimensional 
integrals:

\begin{align*}
  \texttt{EPI}
&=\texttt{EPI}_1 + \texttt{EPI}_2 + \texttt{EPI}_3
\\[5pt]
  \texttt{EPI}_1
&=\frac{1}{2}\int_0^1\,dv\,\int_{v-1}^0 dt \,\, (2+t-2v)\mathcal{I}(t,v) \\
\\
  \texttt{EPI}_2
&=\frac{1}{2}\int_0^1\,dv\,\int_{0}^v dt \,\, (2-2v) \, \mathcal{I}(t,v) \\
\\
  \texttt{EPI}_3
&=\frac{1}{2}\int_0^1\,dv\,\int_v^1 dt \,\, (2-2t) \, \mathcal{I}(t,v).
\end{align*}
Now make Duffy transformations:
$$\begin{array}{lcl}
  \texttt{EPI}_1
 &\xrightarrow[v=xy]{t=x(y-1)}&
  \displaystyle{
 \frac{1}{2}\int_0^1\,dx\,\int_0^1 dy \,\, 
              x(2-xy-x)\,\mathcal{I}\Big(x(y-1),xy\Big)
               }
\\[10pt]
  \texttt{EPI}_2
 &\xrightarrow[y=t/v]{x=v}&
  \displaystyle{
  \frac{1}{2}\int_0^1\,dx\,\int_0^1 dy \,\, 
              x(2-2x)\,\mathcal{I}\Big(xy,x\Big)
               }
\\[10pt]
  \texttt{EPI}_3
 &\xrightarrow[y=v/t]{x=t}&
  \displaystyle{
  \frac{1}{2}\int_0^1\,dx\,\int_0^1 dy \,\, 
              x(2-2x)\,\mathcal{I}\Big(x,xy\Big)
               }
\end{array}$$
Recombine into a single big integral:
\begin{align*}
 \texttt{EPI}
&=\frac{1}{8\pi}\int_0^1 \, dx \int_0^1 \, dy \, 
 \Big\{  (2-xy-x)\frac{e^{ikxR_1(y)}}{R_1(y)}
        +(2-2x)\frac{e^{ikxR_2(y)}}{R_2(y)}
        +(2-2x)\frac{e^{ikxR_3(y)}}{R_3(y)}
 \Big\}
\end{align*}
$$
 R_1(y) = \Big|(y-1)\vb A + y\vb B\Big|, 
 \qquad 
 R_2(y) = \Big|y\vb A + \vb B\Big|, 
 \qquad 
 R_3(y) = \Big|\vb A + y\vb B\Big|.
$$
Finally, evaluate the $x$ integral, leaving behind
a single one-dimensional integral to be evaluated
numerically:
\begin{align*}
 \texttt{EPI}=\frac{1}{8\pi}\int_0^1 dy \, \Bigg\{ 
\,\,\,\,\,
&\frac{e^{ikR_1}}{R_1}
 \Big[ 2\,\texttt{ExpRel}\big(1,-ikR_1\big)
       -\frac{y+1}{2} \texttt{ExpRel}\big(2,-ikR_1\big)
 \Big]
\\
+&\frac{e^{ikR_2}}{R_2}
 \Big[ 2\, \texttt{ExpRel}\big(1,-ikR_2\big)
       - \texttt{ExpRel}\big(2,-ikR_2\big)
 \Big]
\\
+&\frac{e^{ikR_3}}{R_3}
 \Big[ 2\,\texttt{ExpRel}\big(1,-ikR_3\big)
       - \texttt{ExpRel}\big(2,-ikR_3\big)
 \Big]\Bigg\}.
\end{align*}

\end{document}
