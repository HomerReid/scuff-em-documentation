\documentclass[letterpaper]{article}

%********************************************************
%* latex shortcuts used for libscuff documentation files
%* 
%* homer reid -- 1994--2011
%**************************************************
\usepackage{graphicx}
\usepackage{color}
\usepackage{dsfont}
\usepackage{bbm}
\usepackage{amsmath}
\usepackage{amssymb}
\usepackage{float}
\usepackage{psfrag}
\usepackage{mathdots}
%\usepackage{algorithm}
%\usepackage{algorithmic}
%\usepackage{listings}
\usepackage{array}
\usepackage{url}
\usepackage{arydshln}
\usepackage{fancybox}
\usepackage{fancyvrb}
\usepackage{verbatim}

%--------------------------------------------------
%- boldface greek letters 
%--------------------------------------------------
\newcommand\vbphi{\mathbf{\phi}}
\newcommand\vbPhi{\mathbf{\Phi}}
\newcommand{\vbDelta}{\boldsymbol{\Delta}}
\newcommand{\vbrho}{\boldsymbol{\rho}}
\newcommand{\vbchi}{\boldsymbol{\chi}}

%--------------------------------------------------
%- colors -----------------------------------------
%--------------------------------------------------
\newcommand{\red}[1]{\textcolor{red}{#1}}
\newcommand{\blue}[1]{\textcolor{blue}{#1}}
\newcommand{\green}[1]{\textcolor{green}{#1}}
\newcommand{\atan}{\text{atan}}

%--------------------------------------------------
% general commands
%--------------------------------------------------
\newcommand\Tr{\hbox{Tr }}
\newcommand\sups[1]{^{\hbox{\scriptsize{#1}}}}
\newcommand\supt[1]{^{\hbox{\tiny{#1}}}}
\newcommand\subs[1]{_{\hbox{\scriptsize{#1}}}}
\newcommand\subt[1]{_{\hbox{\tiny{#1}}}}
\newcommand{\nn}{\nonumber \\}
\newcommand{\vb}[1]{\mathbf{#1}}
\newcommand{\numeq}[2]{\begin{equation} #2 \label{#1} \end{equation}}
\newcommand{\pard}[2]{\frac{\partial #1}{\partial #2}}
\newcommand{\pardn}[3]{\frac{\partial^{#1} #2}{\partial #3^{#1}}}
\newcommand{\pf}[2]{\left(\frac{#1}{#2}\right)}
\newcommand{\pp}{{\prime\prime}}
\newcommand{\vbhat}[1]{\vb{\hat #1}}
\newcommand{\mc}[1]{\mathcal{#1}}
\newcommand{\bmc}[1]{\boldsymbol{\mathcal{#1}}}
\newcommand{\mb}[1]{\mathbb{#1}}
\newcommand{\ds}[1]{\displaystyle{#1}}
\newcommand{\primedsum}{\sideset{}{'}{\sum}}
\newcommand{\pan}{\mathcal{P}}

%--------------------------------------------------
%- special libscuff stuff--------------------------
%--------------------------------------------------
\newcommand\ls{{\sc libscuff}}
\newcommand\lss{{\sc libscuff\,}}
\newcommand{\BG}{\boldsymbol{\Gamma}}
\newcommand{\MInt}{\vb{\hat M}}
\newcommand{\NInt}{\vb{\hat N}}
\newcommand{\MExt}{\vb{\check M}}
\newcommand{\NExt}{\vb{\check N}}
\newcommand{\xInt}{\vb{\hat x}}
\newcommand{\xExt}{\vb{\check x}}

%--------------------------------------------------
%-- superscripts for dyadic green's functions 
%--------------------------------------------------
\newcommand{\EEe}{^{\hbox{\tiny{EE}\scriptsize{,$e$}}}}
\newcommand{\MEe}{^{\hbox{\tiny{ME}\scriptsize{,$e$}}}}
\newcommand{\EMe}{^{\hbox{\tiny{EM}\scriptsize{,$e$}}}}
\newcommand{\MMe}{^{\hbox{\tiny{MM}\scriptsize{,$e$}}}}
\newcommand{\EEr}{^{\hbox{\tiny{EE}\scriptsize{,$r$}}}}
\newcommand{\MEr}{^{\hbox{\tiny{ME}\scriptsize{,$r$}}}}
\newcommand{\EMr}{^{\hbox{\tiny{EM}\scriptsize{,$r$}}}}
\newcommand{\MMr}{^{\hbox{\tiny{MM}\scriptsize{,$r$}}}}
\newcommand{\incr}{^{\text{\scriptsize inc},r}}
\newcommand{\incro}{^{\text{\scriptsize inc},r_1}}
\newcommand{\incrt}{^{\text{\scriptsize inc},r_2}}

%%%%%%%%%%%%%%%%%%%%%%%%%%%%%%%%%%%%%%%%%%%%%%%%%%%%%%%%%%%%%%%%%%%%%%
%%%%%%%%%%%%%%%%%%%%%%%%%%%%%%%%%%%%%%%%%%%%%%%%%%%%%%%%%%%%%%%%%%%%%%
%%%%%%%%%%%%%%%%%%%%%%%%%%%%%%%%%%%%%%%%%%%%%%%%%%%%%%%%%%%%%%%%%%%%%%
\newcommand{\MMExt}{\check{\mathcal{M}}}
\newcommand{\MMInt}{\hat{\mathcal{M}}}
\newcommand{\NNExt}{\check{\mathcal{N}}}
\newcommand{\NNInt}{\hat{\mathcal{N}}}
\newcommand{\BMMExt}{\boldsymbol{\check{\mathcal{M}}}}
\newcommand{\BMMInt}{\boldsymbol{\hat{\mathcal{M}}}}
\newcommand{\BNNExt}{\boldsymbol{\check{\mathcal{N}}}}
\newcommand{\BNNInt}{\boldsymbol{\hat{\mathcal{N}}}}
\newpage

%--------------------------------------------------
%- inner products and operator matrix elements  ---
%--------------------------------------------------
\newcommand{\expval}[1]{ \big\langle #1 \big\rangle}
\newcommand{\Expval}[1]{ \Big\langle #1 \Big\rangle}
\newcommand{\inp}[2]{ \big\langle #1 \big| #2 \big\rangle}
\newcommand{\Inp}[2]{ \Big\langle #1 \Big| #2 \Big\rangle}
\newcommand{\INP}[2]{ \bigg\langle #1 \bigg| #2 \Big\rangle}
\newcommand{\vmv}[3]{ \big\langle #1 \big| #2 \big| #3 \big\rangle}
\newcommand{\VMV}[3]{ \Big\langle #1 \Big| #2 \Big| #3 \Big\rangle}

%------------------------------------------------------------
%- shaded box for source code inclusions 
%------------------------------------------------------------
%\definecolor{lightgrey}{rgb}{0.85,0.85,0.85}
%\newcommand{\SourceBox}[1]
% { \begin{mdframed}[backgroundcolor=lightgrey, linewidth=2pt,
%                    tikzsetting={draw=black, line width=2pt,
%                                 dashed, dash pattern= on 5pt off 5pt}] %
%  \begin{verbatim} #1 \end{verbatim} \end{mdframed} \medskip
%}
%\IfFileExists{mdframed.sty}
% { \usepackage{mdframed}
%   \surroundwithmdframed[backgroundcolor=lightgrey, linewidth=2pt,
%                         framemethod=tikz,
%                         skipabove=10pt,
%                         skipbelow=10pt,
%                         tikzsetting={draw=black, line width=2pt,
%                                      dashed, dash pattern= on 5pt off 5pt}
%                        ]{verbatim}
% }
% {
% }
%--------------------------------------------------
%- shaded verbatim for code inclusions
%--------------------------------------------------
\definecolor{lightgrey}{rgb}{0.85,0.85,0.85}
\newenvironment{verbcode}{\VerbatimEnvironment% 
  \noindent
  %{\columnwidth-\leftmargin-\rightmargin-2\fboxsep-2\fboxrule-4pt} 
  \begin{Sbox} 
  \begin{minipage}{\linewidth}
  \begin{Verbatim}
}{% 
  \end{Verbatim}  
   \end{minipage}
  \end{Sbox} 
  \fcolorbox{black}{lightgrey}{\textcolor{black}{\TheSbox}}
} 


\newcommand\supsstar[1]{^{\hbox{\scriptsize{#1}}*}}
\newcommand\suptstar[1]{^{\hbox{\scriptsize{#1}}*}}

\graphicspath{{figures/}}

%------------------------------------------------------------
%------------------------------------------------------------
%- Special commands for this document -----------------------
%------------------------------------------------------------
%------------------------------------------------------------

%------------------------------------------------------------
%------------------------------------------------------------
%- Document header  -----------------------------------------
%------------------------------------------------------------
%------------------------------------------------------------
\title {{\sc scuff-scatter} Implementation Notes}
\author {Homer Reid}
\date {March 18, 2012}

%------------------------------------------------------------
%------------------------------------------------------------
%- Start of actual document
%------------------------------------------------------------
%------------------------------------------------------------

\begin{document}
\pagestyle{myheadings}
\markright{Homer Reid: {\sc scuff-scatter} Implementation Notes}
\maketitle

\tableofcontents

%%%%%%%%%%%%%%%%%%%%%%%%%%%%%%%%%%%%%%%%%%%%%%%%%%%%%%%%%%%%%%%%%%%%%%
%%%%%%%%%%%%%%%%%%%%%%%%%%%%%%%%%%%%%%%%%%%%%%%%%%%%%%%%%%%%%%%%%%%%%%
%%%%%%%%%%%%%%%%%%%%%%%%%%%%%%%%%%%%%%%%%%%%%%%%%%%%%%%%%%%%%%%%%%%%%%
\newpage
\section{Concise Formulas for Scattered and Absorbed Power}

In many scattering problems we will want to compute the total
power scattered from, and the total power absorbed by, the
object(s) in a scattering geometry.  The na\"ive way to do 
this would be to integrate the normal Poynting vector over
some fictitious bounding surface surrounding the object(s);
this integral could be evaluated by numerical quadrature,
with the values of the Poynting vector at each quadrature
point computed by the usual {\sc libscuff} methods for
calculating scattered fields. {\sc scuff-scatter} uses a 
more efficient approach, obtaining the scattered and absorbed 
power \textit{directly} from matrix-vector and vector-vector 
(dot) products involving the BEM matrices and 
vectors.\footnote{The possibility of deriving compact 
expressions like these seems to have been first noticed by
Steven Johnson.}

Throughout this section, $\mathcal{S}$ denotes the boundary
of the exterior medium (that is, the union of the outer surfaces
of all objects contained in the exterior medium), and 
$\vbhat{n}$ denotes the normal vector at a point on 
$\mathcal{S}$; $\vbhat{n}$ is taken positive pointing
into the exterior medium (away from the object).
We work at a single frequency and suppress $\omega$ arguments.

\subsection{Absorbed Power}

The absorbed power is the integral of the inward-directed normal 
component of the total Poynting vector over the full surface
$\mathcal{S}$:

%====================================================================%
\begin{align}
 P\sups{abs} 
&= - \oint \vb P\sups{tot}(\vb x) \cdot \vbhat{n} \, dA 
\label{AbsorbedPower}
\end{align}
%====================================================================%
where the minus sign arises because by convention we define $\vbhat{n}$ 
to be the outward-directed surface normal.

At a point $\vb x$ on $\mathcal{S}$, the (outward-directed) normal 
component of the total Poynting vector is 
%====================================================================%
\begin{align}
 \vb P\sups{tot}(\vb x) \cdot \vbhat{n}(\vb x)
  &=\frac{1}{2}\text{Re} 
    \bigg\{ \vbhat{n}(\vb x) \cdot 
            \Big[\vb E\supsstar{tot}(\vb x) \times \vb H\sups{tot}(\vb x)
            \Big]
    \bigg\}
\\
%--------------------------------------------------------------------%
\intertext{Using the fact that $|\vbhat{n}|=1$, we may rewrite this 
           in the form (temporarily suppressing $\vb x$ arguments)}
  &=-\frac{1}{2}\text{Re} 
    \Bigg\{ \big(\vbhat{n}\times \vb H\supsstar{tot}\big)
            \cdot
            \big(\vbhat{n} \times \big(-\vbhat{n} \times \vb E\sups{tot}\big)
            \big)
    \Bigg\}.
\nonumber
%--------------------------------------------------------------------%
\intertext{But the quantities in parentheses here are just the electric
           and magnetic surface currents that enter into the SIE 
           formulation of scattering problems, so we find simply}
%--------------------------------------------------------------------%
\vb P\sups{tot}(\vb x) \cdot \vbhat{n}(\vb x)
  &=-\frac{1}{2}\text{Re}
     \bigg\{ \vb K^* \cdot \Big[\vbhat{n} \times \vb N \Big] \bigg\}.
\label{PTotDotN}
\end{align}
%====================================================================%
%
% some quick matlab code that verifies the above:
%  E=rand(3,1) + sqrt(-1)*rand(3,1);
%  EStar=conj(E);
%  H=rand(3,1) + sqrt(-1)*rand(3,1);
%  nHat=rand(3,1); 
%  nHat=nHat/norm(nHat);
%  K=cross(nHat,H);
%  KStar=conj(K);
%  N=-cross(nHat,E);
%  EStarxH=cross(conj(E),H);
%  P1 = 0.5*real( nHat' * EStarxH )
%  P2 = -0.5*real( KStar.' * cross(nHat, N) );
Inserting (\ref{PTotDotN}) into (\ref{AbsorbedPower}), we have
%====================================================================%
\begin{align}
 P\sups{abs}
&=\frac{1}{2}\text{Re }\oint 
  \bigg\{ \vb K^*(\vb x) \cdot 
          \Big[\vbhat{n} \times \vb N(\vb x) \Big] 
  \bigg\} d\vb x
%--------------------------------------------------------------------%
\intertext{
Insert the surface-current expansions
$\vb K(\vb x)=\sum_\alpha k_\alpha \vb f_\alpha(\vb x), \, 
 \vb N(\vb x)=-Z_0\sum_\alpha n_\alpha \vb f_\alpha(\vb x)
$:}
%--------------------------------------------------------------------%
&=-\frac{Z_0}{2}\text{Re }\sum_{\alpha\beta}  k^*_\alpha n_\beta
  \oint  \vb f_\alpha(\vb x) \cdot  
         \Big[\vbhat{n} \times \vb f_\beta(\vb x) \Big]
         d\vb x
\\
&=-\frac{Z_0}{2}\text{Re }\sum_{\alpha\beta}  k^*_\alpha n_\beta
  \oint  \vb f_\alpha(\vb x) \cdot  
         \Big[\vbhat{n} \times \vb f_\beta(\vb x) \Big]
         d\vb x
\\
&=-\frac{Z_0}{2}\text{Re }\sum_{\alpha\beta} 
    k^*_\alpha O^{(\times)}_{\alpha\beta} n_\beta
\label{PAbsFormula}
\end{align}
%====================================================================%
where I have introduced the ``crossed overlap matrix'' $\vb O^{(\times)}$
for the RWG basis, with matrix elements 
\begin{align*}
O^{(\times)}_{\alpha\beta}
=\oint \vb f_\alpha(\vb x) \cdot 
       \Big[\vbhat{n} \times \vb f_\beta(\vb x) \Big]
       d\vb x.
\end{align*}
%====================================================================%
%The matrix of quantities $O^{(\times)}_{\alpha\beta}$ is sort of like 
%a crossed version of an overlap matrix for the set of 
%basis functions $\{\vb f_\alpha.\}$ 
This is an extremely 
sparse matrix, with vanishing diagonals and precisely 4 nonzero 
entries per row, which may be computed in
closed form (cf. {\sc libscuff} technical memo, Section 12.2).
Thus the operation count for evaluating (\ref{PAbsFormula}) 
is $\mathcal{O}(N)$, as compared to $\mathcal{O}(N^2)$ for 
the equivalent matrix-vector product formulas.

\subsubsection*{Units}
The crossed-overlap matrix element $O^{(\times)}_{\alpha\beta}$ has
units of area.
The electric and magnetic surface-current expansion coefficients
$\{k_\alpha, n_\alpha\}$ both have units of current/length.
(The magnetic current $\vb N$ has units of voltage/length, 
but in \lss conventions the magnetic-current expansion 
coefficients $\{n_\alpha\}$ are defined with the $Z_0$ 
prefactor that ensures they have the same units as $\{k_\alpha\}$.)
Then the units of (\ref{PAbsFormula}) are 
$$ \left[\frac{\text{voltage}}{\text{current}}\right]
   \cdot
   \left[\frac{\text{current}}{\text{length}}\right]
   \cdot
   \Big[\text{length}^2\Big]
   \cdot
   \left[\frac{\text{current}}{\text{length}}\right]
  =\left[\text{voltage }\cdot\text{ current}\right] 
  =\left[\text{power}\right] \checkmark
$$

\subsubsection*{Nested Surfaces}

Expression \ref{PAbsFormula} remains valid as-is when nested
surfaces are present, as long as the summation is understood
to run only over basis functions defined on the surfaces of 
objects that border the exterior medium.

\subsection{Scattered Power}

The scattered power is the integral of the outward-directed normal 
component of the \textit{scattered} Poynting vector over 
$\mathcal{S}$:
%====================================================================%
\begin{align}
P\sups{scat} 
&= + \oint \vb P\sups{scat}(\vb x) \cdot \vbhat{n} \, dA 
\label{ScatteredPower}
\end{align}
%====================================================================%
where the scattered Poynting vector is the 
Poynting vector as computed using only the scattered fields.
In analogy to equation (\ref{PTotDotN}), we write
%====================================================================%
\begin{subequations}
\begin{align}
 \vb P\sups{scat}(\vb x) \cdot \vbhat{n}(\vb x)
  &=\frac{1}{2}\text{Re } 
    \bigg\{ \vbhat{n}(\vb x) \cdot
            \Big[ \vb E\supsstar{scat}(\vb x) \times \vb H\sups{scat}(\vb x)
            \Big]
    \bigg\}
\nonumber
\intertext{Noting that scattered fields are the differences between
total and incident fields and again suppressing $\vb x$ arguments, we
find }
%--------------------------------------------------------------------%
  &=\frac{1}{2}\text{Re }
    \bigg\{ \vbhat{n}\cdot
            \Big[ \big(\vb E\supsstar{tot} - \vb E\supsstar{inc}\big)
                   \times
                  \big(\vb H\sups{tot} - \vb H\sups{inc}\big)
            \Big]
    \bigg\}
\nonumber
\intertext{which we write as a sum of three terms:}
  &= \hspace{0.18in} \vb P\sups{tot} \cdot \vbhat{n}
\\[3pt]
  & \hspace{0.2in}
     +\frac{1}{2}\text{Re } 
        \bigg\{ \vbhat{n}\cdot \Big[ \vb E\supsstar{inc} \times \vb H\sups{inc} \Big]
        \bigg\}
\\[3pt]
  & \hspace{0.2in}
     -\frac{1}{2}\text{Re }
      \bigg\{ \vbhat{n}\cdot 
              \Big[ \vb E\supsstar{inc} \times \vb H\sups{tot} \Big]
             \, + \, 
             \vbhat{n}\cdot 
              \Big[ \vb E\supsstar{tot} \times \vb H\sups{inc} \Big]
      \bigg\}.
\end{align}
\label{ThreeTerms}
\end{subequations}

\subsubsection*{First Term}

The first term here (\ref{ThreeTerms}a) is the normal component of the 
total Poynting vector,
as considered in the previous section; the surface integral of this
term yields $-P\sups{abs}.$

\subsubsection*{Second Term}

The second term (\ref{ThreeTerms}b) is the normal Poynting 
flux due to the incident field sources alone. The surface integral 
of this term yields the net power delivered to the volume of the 
scatterer in the absence of the scatterer. But in the absence of the 
scatterer there is nowhere
for this power to go; there is nothing to absorb it, so any power
that flows in must flow back out. Hence the surface integral
of this term vanishes.

\subsubsection*{Third Term}

The third term (\ref{ThreeTerms}b) is 
%====================================================================%
\begin{align*}
&\hspace{-0.3in}
 -\frac{1}{2}\text{Re }
      \bigg\{ \vbhat{n}\cdot 
              \Big[ \vb E\supsstar{inc} \times \vb H\sups{tot} \Big]
             \, + \, 
             \vbhat{n}\cdot 
              \Big[ \vb E\supsstar{tot} \times \vb H\sups{inc} \Big]
      \bigg\}
\\
&=
 +\frac{1}{2}\text{Re }
      \bigg\{ \vb E\supsstar{inc} \cdot \Big[\vbhat{n}\times \vb H\sups{tot}\Big]
              \, + \, 
              \vb H\supsstar{inc} \cdot \Big[ -\vbhat{n}\times \vb E\sups{tot}\Big]
      \bigg\}
\\
&=
 +\frac{1}{2}\text{Re }
      \bigg\{ \vb E\supsstar{inc} \cdot \vb K \, + \, 
              \vb H\supsstar{inc} \cdot \vb N
      \bigg\}
\intertext{Insert the surface-current expansions 
           $\vb K=\sum k_\alpha \vb f_\alpha, 
           \vb N=-Z_0 \sum n_\alpha \vb f_\alpha$:}
&=
 +\frac{1}{2}\text{Re }\sum_{\alpha}
      \bigg\{ k_\alpha \vb E\supsstar{inc} \cdot \vb f_\alpha 
             \, 
              - Z_0 n_\alpha \vb H\supsstar{inc} \cdot \vb f_\alpha
      \bigg\}
\intertext{The surface integral of this is }
\oint \left\{ \right\} 
&= \frac{Z_0}{2} \text{Re }\sum_{\alpha} 
   \bigg\{ k^*_\alpha 
           \underbrace{\Big[\oint \vb E\supsstar{inc}(\vb x) \cdot \vb f(\vb x) d\vb x\Big]/Z_0}
                     _{v\supt{E}_\alpha}
          -n^*_\alpha
           \underbrace{\Big[\oint \vb H\supsstar{inc}(\vb x) \cdot \vb f(\vb x) d\vb x\Big]}
                     _{v\supt{H}_\alpha}
   \bigg\}
\\[5pt]
&= \frac{Z_0}{2} \sum_{\alpha} \Big[k^*_\alpha v\supt{E}_\alpha 
                                      -n^*_\alpha v\supt{H}_\alpha \Big]
\end{align*}
where $\{ v\supt{E}_\alpha, v\supt{H}_\alpha\}$ are the 
elements of the RHS vector of the BEM system as computed by 
\ls.

\subsubsection*{Total Scattered Power}

Combining all of the above, the total scattered power is 

$$P\sups{scat} = -P\sups{abs} + \frac{1}{2}\texttt{KN}^* \cdot \overline{\texttt{RHS}}$$
where \texttt{KN} and \texttt{RHS} are respectively the vector of 
surface-current expansion coefficients and the RHS vector, as computed
by \ls, and the bar on the second term in the dot product indicates that 
the magnetic contributions enter with a minus sign.

\section{Plane-wave scattering from a dielectric sphere}

To demonstrate the validity of the concise power formulas, we'll use
them here to investigate the scattering and absorption cross-sections
for a dielectric sphere illuminated by a plane wave.

\subsection*{Exact Results}

\subsection*{Calculations using {\sc scuff-scatter}}

\subsection*{Converting power data to cross-sections}
The quantities that {\sc scuff-scatter} writes to the
{\texttt Sphere.power} file are the total 
scattered and absorbed power. To obtain the scattering 
and absorption cross-sections, we divide the power by
the incident power per unit area, which in this case 
is 
$$\text{incident flux of a plane wave in vacuum}=\frac{|\vb E_0|^2}{2Z_0}.$$

\end{document}
