\documentclass{article} 
%********************************************************
%* latex shortcuts used for libscuff documentation files
%* 
%* homer reid -- 1994--2011
%**************************************************
\usepackage{graphicx}
\usepackage{color}
\usepackage{dsfont}
\usepackage{bbm}
\usepackage{amsmath}
\usepackage{amssymb}
\usepackage{float}
\usepackage{psfrag}
\usepackage{mathdots}
%\usepackage{algorithm}
%\usepackage{algorithmic}
%\usepackage{listings}
\usepackage{array}
\usepackage{url}
\usepackage{arydshln}
\usepackage{fancybox}
\usepackage{fancyvrb}
\usepackage{verbatim}

%--------------------------------------------------
%- boldface greek letters 
%--------------------------------------------------
\newcommand\vbphi{\mathbf{\phi}}
\newcommand\vbPhi{\mathbf{\Phi}}
\newcommand{\vbDelta}{\boldsymbol{\Delta}}
\newcommand{\vbrho}{\boldsymbol{\rho}}
\newcommand{\vbchi}{\boldsymbol{\chi}}

%--------------------------------------------------
%- colors -----------------------------------------
%--------------------------------------------------
\newcommand{\red}[1]{\textcolor{red}{#1}}
\newcommand{\blue}[1]{\textcolor{blue}{#1}}
\newcommand{\green}[1]{\textcolor{green}{#1}}
\newcommand{\atan}{\text{atan}}

%--------------------------------------------------
% general commands
%--------------------------------------------------
\newcommand\Tr{\hbox{Tr }}
\newcommand\sups[1]{^{\hbox{\scriptsize{#1}}}}
\newcommand\supt[1]{^{\hbox{\tiny{#1}}}}
\newcommand\subs[1]{_{\hbox{\scriptsize{#1}}}}
\newcommand\subt[1]{_{\hbox{\tiny{#1}}}}
\newcommand{\nn}{\nonumber \\}
\newcommand{\vb}[1]{\mathbf{#1}}
\newcommand{\numeq}[2]{\begin{equation} #2 \label{#1} \end{equation}}
\newcommand{\pard}[2]{\frac{\partial #1}{\partial #2}}
\newcommand{\pardn}[3]{\frac{\partial^{#1} #2}{\partial #3^{#1}}}
\newcommand{\pf}[2]{\left(\frac{#1}{#2}\right)}
\newcommand{\pp}{{\prime\prime}}
\newcommand{\vbhat}[1]{\vb{\hat #1}}
\newcommand{\mc}[1]{\mathcal{#1}}
\newcommand{\bmc}[1]{\boldsymbol{\mathcal{#1}}}
\newcommand{\mb}[1]{\mathbb{#1}}
\newcommand{\ds}[1]{\displaystyle{#1}}
\newcommand{\primedsum}{\sideset{}{'}{\sum}}
\newcommand{\pan}{\mathcal{P}}

%--------------------------------------------------
%- special libscuff stuff--------------------------
%--------------------------------------------------
\newcommand\ls{{\sc libscuff}}
\newcommand\lss{{\sc libscuff\,}}
\newcommand{\BG}{\boldsymbol{\Gamma}}
\newcommand{\MInt}{\vb{\hat M}}
\newcommand{\NInt}{\vb{\hat N}}
\newcommand{\MExt}{\vb{\check M}}
\newcommand{\NExt}{\vb{\check N}}
\newcommand{\xInt}{\vb{\hat x}}
\newcommand{\xExt}{\vb{\check x}}

%--------------------------------------------------
%-- superscripts for dyadic green's functions 
%--------------------------------------------------
\newcommand{\EEe}{^{\hbox{\tiny{EE}\scriptsize{,$e$}}}}
\newcommand{\MEe}{^{\hbox{\tiny{ME}\scriptsize{,$e$}}}}
\newcommand{\EMe}{^{\hbox{\tiny{EM}\scriptsize{,$e$}}}}
\newcommand{\MMe}{^{\hbox{\tiny{MM}\scriptsize{,$e$}}}}
\newcommand{\EEr}{^{\hbox{\tiny{EE}\scriptsize{,$r$}}}}
\newcommand{\MEr}{^{\hbox{\tiny{ME}\scriptsize{,$r$}}}}
\newcommand{\EMr}{^{\hbox{\tiny{EM}\scriptsize{,$r$}}}}
\newcommand{\MMr}{^{\hbox{\tiny{MM}\scriptsize{,$r$}}}}
\newcommand{\incr}{^{\text{\scriptsize inc},r}}
\newcommand{\incro}{^{\text{\scriptsize inc},r_1}}
\newcommand{\incrt}{^{\text{\scriptsize inc},r_2}}

%%%%%%%%%%%%%%%%%%%%%%%%%%%%%%%%%%%%%%%%%%%%%%%%%%%%%%%%%%%%%%%%%%%%%%
%%%%%%%%%%%%%%%%%%%%%%%%%%%%%%%%%%%%%%%%%%%%%%%%%%%%%%%%%%%%%%%%%%%%%%
%%%%%%%%%%%%%%%%%%%%%%%%%%%%%%%%%%%%%%%%%%%%%%%%%%%%%%%%%%%%%%%%%%%%%%
\newcommand{\MMExt}{\check{\mathcal{M}}}
\newcommand{\MMInt}{\hat{\mathcal{M}}}
\newcommand{\NNExt}{\check{\mathcal{N}}}
\newcommand{\NNInt}{\hat{\mathcal{N}}}
\newcommand{\BMMExt}{\boldsymbol{\check{\mathcal{M}}}}
\newcommand{\BMMInt}{\boldsymbol{\hat{\mathcal{M}}}}
\newcommand{\BNNExt}{\boldsymbol{\check{\mathcal{N}}}}
\newcommand{\BNNInt}{\boldsymbol{\hat{\mathcal{N}}}}
\newpage

%--------------------------------------------------
%- inner products and operator matrix elements  ---
%--------------------------------------------------
\newcommand{\expval}[1]{ \big\langle #1 \big\rangle}
\newcommand{\Expval}[1]{ \Big\langle #1 \Big\rangle}
\newcommand{\inp}[2]{ \big\langle #1 \big| #2 \big\rangle}
\newcommand{\Inp}[2]{ \Big\langle #1 \Big| #2 \Big\rangle}
\newcommand{\INP}[2]{ \bigg\langle #1 \bigg| #2 \Big\rangle}
\newcommand{\vmv}[3]{ \big\langle #1 \big| #2 \big| #3 \big\rangle}
\newcommand{\VMV}[3]{ \Big\langle #1 \Big| #2 \Big| #3 \Big\rangle}

%------------------------------------------------------------
%- shaded box for source code inclusions 
%------------------------------------------------------------
%\definecolor{lightgrey}{rgb}{0.85,0.85,0.85}
%\newcommand{\SourceBox}[1]
% { \begin{mdframed}[backgroundcolor=lightgrey, linewidth=2pt,
%                    tikzsetting={draw=black, line width=2pt,
%                                 dashed, dash pattern= on 5pt off 5pt}] %
%  \begin{verbatim} #1 \end{verbatim} \end{mdframed} \medskip
%}
%\IfFileExists{mdframed.sty}
% { \usepackage{mdframed}
%   \surroundwithmdframed[backgroundcolor=lightgrey, linewidth=2pt,
%                         framemethod=tikz,
%                         skipabove=10pt,
%                         skipbelow=10pt,
%                         tikzsetting={draw=black, line width=2pt,
%                                      dashed, dash pattern= on 5pt off 5pt}
%                        ]{verbatim}
% }
% {
% }
%--------------------------------------------------
%- shaded verbatim for code inclusions
%--------------------------------------------------
\definecolor{lightgrey}{rgb}{0.85,0.85,0.85}
\newenvironment{verbcode}{\VerbatimEnvironment% 
  \noindent
  %{\columnwidth-\leftmargin-\rightmargin-2\fboxsep-2\fboxrule-4pt} 
  \begin{Sbox} 
  \begin{minipage}{\linewidth}
  \begin{Verbatim}
}{% 
  \end{Verbatim}  
   \end{minipage}
  \end{Sbox} 
  \fcolorbox{black}{lightgrey}{\textcolor{black}{\TheSbox}}
} 


\graphicspath{{figures/}}

%%%%%%%%%%%%%%%%%%%%%%%%%%%%%%%%%%%%%%%%%%%%%%%%%%%%%%%%%%%%%%%%%%%%%%
% special commands for this document %%%%%%%%%%%%%%%%%%%%%%%%%%%%%%%%% 
%%%%%%%%%%%%%%%%%%%%%%%%%%%%%%%%%%%%%%%%%%%%%%%%%%%%%%%%%%%%%%%%%%%%%%

%**************************************************
%* Document header info ***************************
%**************************************************
\title{Surface Impedance Boundary Conditions in {\sc scuff-em}}
\author {Homer Reid}
\date {May 9, 2012}

%**************************************************
%* Start of actual document ***********************
%**************************************************

\begin{document}
\maketitle

\pagestyle{myheadings}
\markright{Homer Reid: Surface Impedance Boundary Conditions in {\sc scuff-em} }

\tableofcontents 

%%%%%%%%%%%%%%%%%%%%%%%%%%%%%%%%%%%%%%%%%%%%%%%%%%%%%%%%%%%%%%%%%%%%%%
%%%%%%%%%%%%%%%%%%%%%%%%%%%%%%%%%%%%%%%%%%%%%%%%%%%%%%%%%%%%%%%%%%%%%%
%%%%%%%%%%%%%%%%%%%%%%%%%%%%%%%%%%%%%%%%%%%%%%%%%%%%%%%%%%%%%%%%%%%%%%
\newpage
\section{The Surface-Impedance Boundary Condition}

The usual boundary condition imposed at the surface of a
perfectly electrically conducting (PEC) scatterer is that
the total tangential electric field vanish:
%====================================================================%
\numeq{PECBC1}
 { \vb E_{\parallel}\sups{tot}(\vb x) = 0. }
%====================================================================%

At the surface of an \textit{imperfectly} electrically conducting
(IPEC) scatterer with dimensionless relative surface impedance
$\zeta$, the boundary condition (\ref{PECBC1}) is modified to
read 
%====================================================================%
\numeq{IPECBC1}
{
\vb E_{\parallel}\sups{tot}(\vb x)
= \zeta Z_0 \vbhat{n}\times \vb H\sups{tot}(\vb x)
}
%====================================================================%
where $Z_0\approx 377\,\Omega$ is the impedance of vacuum.

I will refer to (\ref{IPECBC1}) as the ``impedance boundary condition'' (IBC).

%%%%%%%%%%%%%%%%%%%%%%%%%%%%%%%%%%%%%%%%%%%%%%%%%%%%%%%%%%%%%%%%%%%%%%
%%%%%%%%%%%%%%%%%%%%%%%%%%%%%%%%%%%%%%%%%%%%%%%%%%%%%%%%%%%%%%%%%%%%%%
%%%%%%%%%%%%%%%%%%%%%%%%%%%%%%%%%%%%%%%%%%%%%%%%%%%%%%%%%%%%%%%%%%%%%%
\newpage
\section{Two SIE formulations for IPEC bodies}

\subsection{Review: SIE formulation for PEC bodies}

I will consider two distinct SIE formulations for IPEC bodies.
These are both variants of the usual SIE procedure for PEC
bodies, which---by way of review---I summarize thusly:
%
\begin{enumerate}
 \item We introduce an electric surface current $\vb K(\vb x)$ on 
       the surface of a PEC scatterer. This current is related to 
       the total tangential $\vb H$-field according to 
       \numeq{KDef}
       {\vb K(\vb x) = \vbhat{n}\times \vb H\sups{tot}(\vb x).}
 \item We do \textit{not} need to introduce a magnetic surface 
       current; such a current would be proportional to the total
       tangential $\vb E$ field, but this vanishes in view of 
       the boundary condition (\ref{PECBC1}):
       \numeq{NVanishes}
        {\vb N(\vb x) = -\vbhat{n}\times \vb E\sups{tot}(\vb x)\equiv0.}
 \item $\vb K$ gives rise to scattered $\vb E$ and $\vb H$ fields according to
       \begin{align}
        \vb E\sups{scat} &= \int \BG_{\parallel}\supt{EE}(\vb x, \vb x^\prime) 
                            \cdot \vb K(\vb x^\prime) d\vb x^\prime
        \\
        \vb H\sups{scat} &= \int \BG_{\parallel}\supt{ME}(\vb x, \vb x^\prime) 
                            \cdot \vb K(\vb x^\prime) d\vb x^\prime.
       \end{align}
 \item We solve for $\vb K$ by demanding that the scattered field
       to which it gives rise satisfy the boundary condition
       (\ref{PECBC1}):
       \begin{align}
                \vb E_{\parallel}\sups{scat}(\vb x)
            &= -\vb E_{\parallel}\sups{inc}(\vb x)
        \intertext{or} 
             \underbrace{
                \int \BG_{\parallel}\supt{EE}(\vb x, \vb x^\prime) 
                     \cdot \vb K(\vb x^\prime) d\vb x^\prime 
                        }\BG_{\parallel}\supt{EE} \star \vb K
            &= - \vb E\sups{inc}(\vb x).
        \end{align}
\end{enumerate}

%=================================================
%=================================================
%=================================================
\subsection{SIE formulation for IPEC bodies}

My first SIE formulation for IPEC bodies goes like this:
%
\begin{enumerate}
 \item As in the PEC case, to each IPEC I continue to assign an electric
       surface current $\vb K$ related to the total $\vb H$ field 
       by equation (\ref{KDef}).
 \item In contrast to the PEC case, I now assign to each IPEC
       surface a nonvanishing magnetic current, which is
       not an independent unknown but is instead
       determined by $\vb K$ via the (\ref{IPECBC1}):
       %====================================================================%
       $$\vb N(\vb x) = -\vbhat{n}\times \vb E\sups{tot}(\vb x)
                      = -\zeta Z_0 \vbhat{n}\times \vb K(\vb x).
       $$
       %====================================================================%
 \item $\vb K$ and $\vb N\equiv \vb N[\vb K]$ give rise to scattered $\vb E$ and 
       $\vb H$ fields according to
       \numeq{EHScat1}
       { \vb E\sups{scat} =
           \BG\supt{EE} \star \vb K
          +\BG\supt{EM} \star \vb N,
          \qquad
         \vb H\sups{scat} =
          \BG\supt{ME} \star \vb K
         +\BG\supt{MM} \star \vb N
       }
%       which I could write alternatively as 
%       \begin{align*}
%         \vb E\sups{scat} &= 
%           \int \left\{ \BG_{\parallel}\supt{EE}(\vb x, \vb x^\prime) 
%                        +Z_s \Big[\vbhat{n}\times \BG_{\parallel}\supt{ME}(\vb x, \vb x^\prime)\Big]
%                \right\}
%                        \cdot \vb K(\vb x^\prime) \,d\vb x^\prime
%         \\
%         \vb H\sups{scat} &= 
%           \int \left\{ \BG_{\parallel}\supt{ME}(\vb x, \vb x^\prime) 
%                        +Z_s \Big[\vbhat{n}\times \BG_{\parallel}\supt{MM}(\vb x, \vb x^\prime)\Big]
%                \right\}
%                        \cdot \vb K(\vb x^\prime) \,d\vb x^\prime
%       \end{align*}
 \item I solve for $\vb K$ by demanding that the scattered fields
       (\ref{EHScat1})
       satisfy the boundary condition (\ref{IPECBC1}):
       $$
           \vb E_{\parallel}\sups{scat}(\vb x) 
            -\zeta Z_0 \vbhat{n}\times \vb H_{\parallel}\sups{scat}(\vb x)
       = 
          -\vb E_{\parallel}\sups{inc}(\vb x) 
            +\zeta Z_0 s \vbhat{n}\times \vb H_{\parallel}\sups{inc}(\vb x).
       $$
       or
       \begin{align*}
       \int\bigg\{ &\BG_{\parallel}\supt{EE}(\vb x, \vb x^\prime) 
                   \cdot \vb K(\vb x^\prime)
                   \\
                   &- Z_s\BG_{\parallel}\supt{ME}(\vb x, \vb x^\prime)
                   \cdot \Big[\vbhat{n}\times \vb K(\vb x^\prime)\Big] 
                   \\
                   &-Z_s\vbhat{n}\times \BG_{\parallel}\supt{EM}(\vb x, \vb x^\prime) 
                   \cdot \vb K(\vb x^\prime)
                   \\
                   &+Z_s^2
                   \vbhat{n}\times\BG_{\parallel}\supt{MM}(\vb x, \vb x^\prime)
                   \cdot \Big[\vbhat{n}\times \vb K(\vb x^\prime)\Big]
       \bigg\}
       = -\vb E_{\parallel}\sups{inc}(\vb x) 
            +Z_s \vbhat{n}\times \vb H_{\parallel}\sups{inc}(\vb x).
       \end{align*}
\end{enumerate}

%%%%%%%%%%%%%%%%%%%%%%%%%%%%%%%%%%%%%%%%%%%%%%%%%%%%%%%%%%%%%%%%%%%%%%
%%%%%%%%%%%%%%%%%%%%%%%%%%%%%%%%%%%%%%%%%%%%%%%%%%%%%%%%%%%%%%%%%%%%%%
%%%%%%%%%%%%%%%%%%%%%%%%%%%%%%%%%%%%%%%%%%%%%%%%%%%%%%%%%%%%%%%%%%%%%%
\newpage
\section{Alternative formulation}

\end{document}
