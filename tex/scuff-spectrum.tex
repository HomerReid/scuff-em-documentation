\documentclass[letterpaper]{article}

%********************************************************
%* latex shortcuts used for libscuff documentation files
%* 
%* homer reid -- 1994--2011
%**************************************************
\usepackage{graphicx}
\usepackage{color}
\usepackage{dsfont}
\usepackage{bbm}
\usepackage{amsmath}
\usepackage{amssymb}
\usepackage{float}
\usepackage{psfrag}
\usepackage{mathdots}
%\usepackage{algorithm}
%\usepackage{algorithmic}
%\usepackage{listings}
\usepackage{array}
\usepackage{url}
\usepackage{arydshln}
\usepackage{fancybox}
\usepackage{fancyvrb}
\usepackage{verbatim}

%--------------------------------------------------
%- boldface greek letters 
%--------------------------------------------------
\newcommand\vbphi{\mathbf{\phi}}
\newcommand\vbPhi{\mathbf{\Phi}}
\newcommand{\vbDelta}{\boldsymbol{\Delta}}
\newcommand{\vbrho}{\boldsymbol{\rho}}
\newcommand{\vbchi}{\boldsymbol{\chi}}

%--------------------------------------------------
%- colors -----------------------------------------
%--------------------------------------------------
\newcommand{\red}[1]{\textcolor{red}{#1}}
\newcommand{\blue}[1]{\textcolor{blue}{#1}}
\newcommand{\green}[1]{\textcolor{green}{#1}}
\newcommand{\atan}{\text{atan}}

%--------------------------------------------------
% general commands
%--------------------------------------------------
\newcommand\Tr{\hbox{Tr }}
\newcommand\sups[1]{^{\hbox{\scriptsize{#1}}}}
\newcommand\supt[1]{^{\hbox{\tiny{#1}}}}
\newcommand\subs[1]{_{\hbox{\scriptsize{#1}}}}
\newcommand\subt[1]{_{\hbox{\tiny{#1}}}}
\newcommand{\nn}{\nonumber \\}
\newcommand{\vb}[1]{\mathbf{#1}}
\newcommand{\numeq}[2]{\begin{equation} #2 \label{#1} \end{equation}}
\newcommand{\pard}[2]{\frac{\partial #1}{\partial #2}}
\newcommand{\pardn}[3]{\frac{\partial^{#1} #2}{\partial #3^{#1}}}
\newcommand{\pf}[2]{\left(\frac{#1}{#2}\right)}
\newcommand{\pp}{{\prime\prime}}
\newcommand{\vbhat}[1]{\vb{\hat #1}}
\newcommand{\mc}[1]{\mathcal{#1}}
\newcommand{\bmc}[1]{\boldsymbol{\mathcal{#1}}}
\newcommand{\mb}[1]{\mathbb{#1}}
\newcommand{\ds}[1]{\displaystyle{#1}}
\newcommand{\primedsum}{\sideset{}{'}{\sum}}
\newcommand{\pan}{\mathcal{P}}

%--------------------------------------------------
%- special libscuff stuff--------------------------
%--------------------------------------------------
\newcommand\ls{{\sc libscuff}}
\newcommand\lss{{\sc libscuff\,}}
\newcommand{\BG}{\boldsymbol{\Gamma}}
\newcommand{\MInt}{\vb{\hat M}}
\newcommand{\NInt}{\vb{\hat N}}
\newcommand{\MExt}{\vb{\check M}}
\newcommand{\NExt}{\vb{\check N}}
\newcommand{\xInt}{\vb{\hat x}}
\newcommand{\xExt}{\vb{\check x}}

%--------------------------------------------------
%-- superscripts for dyadic green's functions 
%--------------------------------------------------
\newcommand{\EEe}{^{\hbox{\tiny{EE}\scriptsize{,$e$}}}}
\newcommand{\MEe}{^{\hbox{\tiny{ME}\scriptsize{,$e$}}}}
\newcommand{\EMe}{^{\hbox{\tiny{EM}\scriptsize{,$e$}}}}
\newcommand{\MMe}{^{\hbox{\tiny{MM}\scriptsize{,$e$}}}}
\newcommand{\EEr}{^{\hbox{\tiny{EE}\scriptsize{,$r$}}}}
\newcommand{\MEr}{^{\hbox{\tiny{ME}\scriptsize{,$r$}}}}
\newcommand{\EMr}{^{\hbox{\tiny{EM}\scriptsize{,$r$}}}}
\newcommand{\MMr}{^{\hbox{\tiny{MM}\scriptsize{,$r$}}}}
\newcommand{\incr}{^{\text{\scriptsize inc},r}}
\newcommand{\incro}{^{\text{\scriptsize inc},r_1}}
\newcommand{\incrt}{^{\text{\scriptsize inc},r_2}}

%%%%%%%%%%%%%%%%%%%%%%%%%%%%%%%%%%%%%%%%%%%%%%%%%%%%%%%%%%%%%%%%%%%%%%
%%%%%%%%%%%%%%%%%%%%%%%%%%%%%%%%%%%%%%%%%%%%%%%%%%%%%%%%%%%%%%%%%%%%%%
%%%%%%%%%%%%%%%%%%%%%%%%%%%%%%%%%%%%%%%%%%%%%%%%%%%%%%%%%%%%%%%%%%%%%%
\newcommand{\MMExt}{\check{\mathcal{M}}}
\newcommand{\MMInt}{\hat{\mathcal{M}}}
\newcommand{\NNExt}{\check{\mathcal{N}}}
\newcommand{\NNInt}{\hat{\mathcal{N}}}
\newcommand{\BMMExt}{\boldsymbol{\check{\mathcal{M}}}}
\newcommand{\BMMInt}{\boldsymbol{\hat{\mathcal{M}}}}
\newcommand{\BNNExt}{\boldsymbol{\check{\mathcal{N}}}}
\newcommand{\BNNInt}{\boldsymbol{\hat{\mathcal{N}}}}
\newpage

%--------------------------------------------------
%- inner products and operator matrix elements  ---
%--------------------------------------------------
\newcommand{\expval}[1]{ \big\langle #1 \big\rangle}
\newcommand{\Expval}[1]{ \Big\langle #1 \Big\rangle}
\newcommand{\inp}[2]{ \big\langle #1 \big| #2 \big\rangle}
\newcommand{\Inp}[2]{ \Big\langle #1 \Big| #2 \Big\rangle}
\newcommand{\INP}[2]{ \bigg\langle #1 \bigg| #2 \Big\rangle}
\newcommand{\vmv}[3]{ \big\langle #1 \big| #2 \big| #3 \big\rangle}
\newcommand{\VMV}[3]{ \Big\langle #1 \Big| #2 \Big| #3 \Big\rangle}

%------------------------------------------------------------
%- shaded box for source code inclusions 
%------------------------------------------------------------
%\definecolor{lightgrey}{rgb}{0.85,0.85,0.85}
%\newcommand{\SourceBox}[1]
% { \begin{mdframed}[backgroundcolor=lightgrey, linewidth=2pt,
%                    tikzsetting={draw=black, line width=2pt,
%                                 dashed, dash pattern= on 5pt off 5pt}] %
%  \begin{verbatim} #1 \end{verbatim} \end{mdframed} \medskip
%}
%\IfFileExists{mdframed.sty}
% { \usepackage{mdframed}
%   \surroundwithmdframed[backgroundcolor=lightgrey, linewidth=2pt,
%                         framemethod=tikz,
%                         skipabove=10pt,
%                         skipbelow=10pt,
%                         tikzsetting={draw=black, line width=2pt,
%                                      dashed, dash pattern= on 5pt off 5pt}
%                        ]{verbatim}
% }
% {
% }
%--------------------------------------------------
%- shaded verbatim for code inclusions
%--------------------------------------------------
\definecolor{lightgrey}{rgb}{0.85,0.85,0.85}
\newenvironment{verbcode}{\VerbatimEnvironment% 
  \noindent
  %{\columnwidth-\leftmargin-\rightmargin-2\fboxsep-2\fboxrule-4pt} 
  \begin{Sbox} 
  \begin{minipage}{\linewidth}
  \begin{Verbatim}
}{% 
  \end{Verbatim}  
   \end{minipage}
  \end{Sbox} 
  \fcolorbox{black}{lightgrey}{\textcolor{black}{\TheSbox}}
} 


\graphicspath{{figures/}}

%------------------------------------------------------------
%------------------------------------------------------------
%- Special commands for this document -----------------------
%------------------------------------------------------------
%------------------------------------------------------------
\newcommand{\vbsigma}{\boldsymbol{\sigma}}
\newcommand{\YY}{\mc Y}
\newcommand{\BE}{\begin{equation}}
\newcommand{\EE}{\end{equation}}

%------------------------------------------------------------
%------------------------------------------------------------
%- Document header  -----------------------------------------
%------------------------------------------------------------
%------------------------------------------------------------
\title {{\sc scuff-spectrum}}
\author {Homer Reid}
\date {November 20, 2015}

%------------------------------------------------------------
%------------------------------------------------------------
%- Start of actual document
%------------------------------------------------------------
%------------------------------------------------------------

\begin{document}
\pagestyle{myheadings}
\markright{Homer Reid: {\sc scuff-spectrum}}
\maketitle 

\tableofcontents

%====================================================================%
%====================================================================%
%====================================================================%
\newpage
\section{Frequency derivatives of BEM matrix elements}

%====================================================================%
\begin{align*}
 M_{\alpha\beta}(\omega)
&= M_{\alpha\beta}\sups{ext}(\omega)
  +M_{\alpha\beta}\sups{int}(\omega),
   \qquad \bmc B_\alpha, \bmc B_\beta \quad \text{on same surface}
\\
&= M_{\alpha\beta}\sups{ext}(\omega)
   \qquad \bmc B_\alpha, \bmc B_\beta \quad \text{on different surfaces}
\end{align*}
%====================================================================%

%====================================================================%
\begin{align*}
 M_{ab}^r
 &=i\frac{\omega}{c_0} 
   \left(\begin{array}{cc}
       \mu_r \mb G_{ab}(k_r) & -n_r \mb C_{ab}(k_r) \\[5pt]
       -n_r \mb C_{ab}(k_r) & -\epsilon_r \mb G_{ab}(k_r)
   \end{array}\right)
\end{align*}
$$ n_r=\sqrt{\epsilon_r \mu_r}$$
%====================================================================%

%====================================================================%
\begin{align*}
\frac{d}{d\omega}
 M_{ab}^r
 &= \frac{1}{\omega} M_{ab}^r
    +i\frac{\omega}{c_0} 
     \left(\begin{array}{cc}
       \mu_r^\prime \mb G_{ab}(k_r) & -n_r^\prime \mb C_{ab}(k_r) \\[5pt]
       -n_r^\prime \mb C_{ab}(k_r) & -\epsilon_r^\prime \mb G_{ab}(k_r)
     \end{array}\right)
\\
&\qquad
    +i\frac{\omega}{c_0} 
     \left(\begin{array}{cc}
       \mu_r n_r \mb G^\prime_{ab}(k_r) & -n_r^2 \mb C_{ab}(k_r) \\[5pt]
       -n_r^2 \mb C^\prime_{ab}(k_r) & -\epsilon_r n_r \mb G_{ab}(k_r)
     \end{array}\right)
\end{align*}
%====================================================================%
Here primes on $\{\epsilon_r, \mu_r, n_r\}$ denote differentiation with 
respect to $\omega$, while primes on $\mb G$ and $\mb C$ denote 
differentiation with respect to $k$. 

The $\mb G, \mb C$ matrix elements and their $k$ derivatives are 
%====================================================================%
\begin{align*}
\mb G_{ab}(k)
&=
 \int \left(\vb b_a \cdot \vb b_b
            - \frac{ \left[\nabla \cdot \vb b_a\right]
                     \left[\nabla \cdot \vb b_b\right]
                   }{k^2}
      \right) G_0(k, \vb r) d^4 \vb r
\\
%--------------------------------------------------------------------%
 \mb G^\prime_{ab}(k)
&= \frac{2}{k^3}
   \int \left[\nabla \cdot \vb b_a\right] 
        \left[\nabla \cdot \vb b_b\right] G_0(k,\vb r)
   \, d^4 \vb r
\\
%--------------------------------------------------------------------%
&\qquad
 +
 \int \left(\vb b_a \cdot \vb b_b 
            - \frac{ \left[\nabla \cdot \vb b_a\right]
                     \left[\nabla \cdot \vb b_b\right] 
                   }{k^2}
      \right) G_0^\prime(k,\vb r)\, d^4\vb r
\\
%--------------------------------------------------------------------%
\mb C_{ab}(k)
&=
 \frac{1}{ik}
 \int \left(\vb b_a \times \vb b_b \right) \cdot \nabla G_0(k,\vb r)
 \,d^4 \vb r
\\
%--------------------------------------------------------------------%
\mb C^\prime_{ab}(k)
&= -\frac{1}{k}\mb C_{ab}(k)
   +
   \int \left(\vb b_a \times \vb b_b \right) \cdot \nabla G_0^\prime(k,\vb r)
 \,d^4 \vb r
\end{align*}
%====================================================================%
In these equations, I have 
%====================================================================%
\begin{align*}
 G_0(k,\vb r) &= 
  \begin{cases} 
    \displaystyle{ \frac{e^{ikr}}{4\pi r} }, \qquad &\text{non-periodic} 
       \\[10pt]
    \displaystyle{ \sum_{\vb L} e^{i\vb k\subt{B}\cdot \vb L}
                   \frac{e^{ik|\vb r+\vb L|}}{4\pi |\vb r+\vb L|}
                 }, \qquad &\text{Bloch-periodic with Bloch vector $\vb k\subt{B}$} \\
  \end{cases} 
\end{align*}
%====================================================================%
In either case, $k$ derivatives of $G_0$ may be related to 
spatial derivatives according to
%====================================================================%
\begin{align}
 \pard{}{k} G_0
 &= -i|\vb r|^2 \left( \frac{\vb r \cdot \nabla G_0}{|\vb r|} - ik G_0\right)
\\
 \pard{}{k} \nabla G_0 
&= -k \vb r G_0
\label{NonsingularKernels}
\end{align}
%====================================================================%
Importantly, the kernels defined by (\ref{NonsingularKernels}) are
both \textit{nonsingular} at $\vb r=0$, allowing the use of simple
numerical cubature to evaluate matrix elements.

\end{document}
