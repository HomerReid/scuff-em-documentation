\documentclass[letterpaper]{article}

\input{scufftex}

\graphicspath{{figures/}}

%------------------------------------------------------------
%------------------------------------------------------------
%- Special commands for this document -----------------------
%------------------------------------------------------------
%------------------------------------------------------------
\newcommand{\vbsigma}{\boldsymbol{\sigma}}
\newcommand{\YY}{\mc Y}
\newcommand{\BE}{\begin{equation}}
\newcommand{\EE}{\end{equation}}

%------------------------------------------------------------
%------------------------------------------------------------
%- Document header  -----------------------------------------
%------------------------------------------------------------
%------------------------------------------------------------
\title {{\sc scuff-spectrum}}
\author {Homer Reid}
\date {November 20, 2015}

%------------------------------------------------------------
%------------------------------------------------------------
%- Start of actual document
%------------------------------------------------------------
%------------------------------------------------------------

\begin{document}
\pagestyle{myheadings}
\markright{Homer Reid: {\sc scuff-spectrum}}
\maketitle 

\tableofcontents

%====================================================================%
%====================================================================%
%====================================================================%
\newpage
\section{Frequency derivatives of BEM matrix elements}

%====================================================================%
\begin{align*}
 M_{\alpha\beta}(\omega)
&= M_{\alpha\beta}\sups{ext}(\omega)
  +M_{\alpha\beta}\sups{int}(\omega),
   \qquad \bmc B_\alpha, \bmc B_\beta \quad \text{on same surface}
\\
&= M_{\alpha\beta}\sups{ext}(\omega)
   \qquad \bmc B_\alpha, \bmc B_\beta \quad \text{on different surfaces}
\end{align*}
%====================================================================%

%====================================================================%
\begin{align*}
 M_{ab}^r
 &=i\frac{\omega}{c_0} 
   \left(\begin{array}{cc}
       \mu_r \mb G_{ab}(k_r) & -n_r \mb C_{ab}(k_r) \\[5pt]
       -n_r \mb C_{ab}(k_r) & -\epsilon_r \mb G_{ab}(k_r)
   \end{array}\right)
\end{align*}
$$ n_r=\sqrt{\epsilon_r \mu_r}$$
%====================================================================%

%====================================================================%
\begin{align*}
\frac{d}{d\omega}
 M_{ab}^r
 &= \frac{1}{\omega} M_{ab}^r
    +i\frac{\omega}{c_0} 
     \left(\begin{array}{cc}
       \mu_r^\prime \mb G_{ab}(k_r) & -n_r^\prime \mb C_{ab}(k_r) \\[5pt]
       -n_r^\prime \mb C_{ab}(k_r) & -\epsilon_r^\prime \mb G_{ab}(k_r)
     \end{array}\right)
\\
&\qquad
    +i\frac{\omega}{c_0} 
     \left(\begin{array}{cc}
       \mu_r n_r \mb G^\prime_{ab}(k_r) & -n_r^2 \mb C_{ab}(k_r) \\[5pt]
       -n_r^2 \mb C^\prime_{ab}(k_r) & -\epsilon_r n_r \mb G_{ab}(k_r)
     \end{array}\right)
\end{align*}
%====================================================================%
Here primes on $\{\epsilon_r, \mu_r, n_r\}$ denote differentiation with 
respect to $\omega$, while primes on $\mb G$ and $\mb C$ denote 
differentiation with respect to $k$. 

The $\mb G, \mb C$ matrix elements and their $k$ derivatives are 
%====================================================================%
\begin{align*}
\mb G_{ab}(k)
&=
 \int \left(\vb b_a \cdot \vb b_b
            - \frac{ \left[\nabla \cdot \vb b_a\right]
                     \left[\nabla \cdot \vb b_b\right]
                   }{k^2}
      \right) G_0(k, \vb r) d^4 \vb r
\\
%--------------------------------------------------------------------%
 \mb G^\prime_{ab}(k)
&= \frac{2}{k^3}
   \int \left[\nabla \cdot \vb b_a\right] 
        \left[\nabla \cdot \vb b_b\right] G_0(k,\vb r)
   \, d^4 \vb r
\\
%--------------------------------------------------------------------%
&\qquad
 +
 \int \left(\vb b_a \cdot \vb b_b 
            - \frac{ \left[\nabla \cdot \vb b_a\right]
                     \left[\nabla \cdot \vb b_b\right] 
                   }{k^2}
      \right) G_0^\prime(k,\vb r)\, d^4\vb r
\\
%--------------------------------------------------------------------%
\mb C_{ab}(k)
&=
 \frac{1}{ik}
 \int \left(\vb b_a \times \vb b_b \right) \cdot \nabla G_0(k,\vb r)
 \,d^4 \vb r
\\
%--------------------------------------------------------------------%
\mb C^\prime_{ab}(k)
&= -\frac{1}{k}\mb C_{ab}(k)
   +
   \int \left(\vb b_a \times \vb b_b \right) \cdot \nabla G_0^\prime(k,\vb r)
 \,d^4 \vb r
\end{align*}
%====================================================================%
In these equations, I have 
%====================================================================%
\begin{align*}
 G_0(k,\vb r) &= 
  \begin{cases} 
    \displaystyle{ \frac{e^{ikr}}{4\pi r} }, \qquad &\text{non-periodic} 
       \\[10pt]
    \displaystyle{ \sum_{\vb L} e^{i\vb k\subt{B}\cdot \vb L}
                   \frac{e^{ik|\vb r+\vb L|}}{4\pi |\vb r+\vb L|}
                 }, \qquad &\text{Bloch-periodic with Bloch vector $\vb k\subt{B}$} \\
  \end{cases} 
\end{align*}
%====================================================================%
In either case, $k$ derivatives of $G_0$ may be related to 
spatial derivatives according to
%====================================================================%
\begin{align}
 \pard{}{k} G_0
 &= -i|\vb r|^2 \left( \frac{\vb r \cdot \nabla G_0}{|\vb r|} - ik G_0\right)
\\
 \pard{}{k} \nabla G_0 
&= -k \vb r G_0
\label{NonsingularKernels}
\end{align}
%====================================================================%
Importantly, the kernels defined by (\ref{NonsingularKernels}) are
both \textit{nonsingular} at $\vb r=0$, allowing the use of simple
numerical cubature to evaluate matrix elements.

\end{document}
