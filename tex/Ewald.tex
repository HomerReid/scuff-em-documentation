\documentclass[letterpaper]{article}

%********************************************************
%* latex shortcuts used for libscuff documentation files
%* 
%* homer reid -- 1994--2011
%**************************************************
\usepackage{graphicx}
\usepackage{color}
\usepackage{dsfont}
\usepackage{bbm}
\usepackage{amsmath}
\usepackage{amssymb}
\usepackage{float}
\usepackage{psfrag}
\usepackage{mathdots}
%\usepackage{algorithm}
%\usepackage{algorithmic}
%\usepackage{listings}
\usepackage{array}
\usepackage{url}
\usepackage{arydshln}
\usepackage{fancybox}
\usepackage{fancyvrb}
\usepackage{verbatim}

%--------------------------------------------------
%- boldface greek letters 
%--------------------------------------------------
\newcommand\vbphi{\mathbf{\phi}}
\newcommand\vbPhi{\mathbf{\Phi}}
\newcommand{\vbDelta}{\boldsymbol{\Delta}}
\newcommand{\vbrho}{\boldsymbol{\rho}}
\newcommand{\vbchi}{\boldsymbol{\chi}}

%--------------------------------------------------
%- colors -----------------------------------------
%--------------------------------------------------
\newcommand{\red}[1]{\textcolor{red}{#1}}
\newcommand{\blue}[1]{\textcolor{blue}{#1}}
\newcommand{\green}[1]{\textcolor{green}{#1}}
\newcommand{\atan}{\text{atan}}

%--------------------------------------------------
% general commands
%--------------------------------------------------
\newcommand\Tr{\hbox{Tr }}
\newcommand\sups[1]{^{\hbox{\scriptsize{#1}}}}
\newcommand\supt[1]{^{\hbox{\tiny{#1}}}}
\newcommand\subs[1]{_{\hbox{\scriptsize{#1}}}}
\newcommand\subt[1]{_{\hbox{\tiny{#1}}}}
\newcommand{\nn}{\nonumber \\}
\newcommand{\vb}[1]{\mathbf{#1}}
\newcommand{\numeq}[2]{\begin{equation} #2 \label{#1} \end{equation}}
\newcommand{\pard}[2]{\frac{\partial #1}{\partial #2}}
\newcommand{\pardn}[3]{\frac{\partial^{#1} #2}{\partial #3^{#1}}}
\newcommand{\pf}[2]{\left(\frac{#1}{#2}\right)}
\newcommand{\pp}{{\prime\prime}}
\newcommand{\vbhat}[1]{\vb{\hat #1}}
\newcommand{\mc}[1]{\mathcal{#1}}
\newcommand{\bmc}[1]{\boldsymbol{\mathcal{#1}}}
\newcommand{\mb}[1]{\mathbb{#1}}
\newcommand{\ds}[1]{\displaystyle{#1}}
\newcommand{\primedsum}{\sideset{}{'}{\sum}}
\newcommand{\pan}{\mathcal{P}}

%--------------------------------------------------
%- special libscuff stuff--------------------------
%--------------------------------------------------
\newcommand\ls{{\sc libscuff}}
\newcommand\lss{{\sc libscuff\,}}
\newcommand{\BG}{\boldsymbol{\Gamma}}
\newcommand{\MInt}{\vb{\hat M}}
\newcommand{\NInt}{\vb{\hat N}}
\newcommand{\MExt}{\vb{\check M}}
\newcommand{\NExt}{\vb{\check N}}
\newcommand{\xInt}{\vb{\hat x}}
\newcommand{\xExt}{\vb{\check x}}

%--------------------------------------------------
%-- superscripts for dyadic green's functions 
%--------------------------------------------------
\newcommand{\EEe}{^{\hbox{\tiny{EE}\scriptsize{,$e$}}}}
\newcommand{\MEe}{^{\hbox{\tiny{ME}\scriptsize{,$e$}}}}
\newcommand{\EMe}{^{\hbox{\tiny{EM}\scriptsize{,$e$}}}}
\newcommand{\MMe}{^{\hbox{\tiny{MM}\scriptsize{,$e$}}}}
\newcommand{\EEr}{^{\hbox{\tiny{EE}\scriptsize{,$r$}}}}
\newcommand{\MEr}{^{\hbox{\tiny{ME}\scriptsize{,$r$}}}}
\newcommand{\EMr}{^{\hbox{\tiny{EM}\scriptsize{,$r$}}}}
\newcommand{\MMr}{^{\hbox{\tiny{MM}\scriptsize{,$r$}}}}
\newcommand{\incr}{^{\text{\scriptsize inc},r}}
\newcommand{\incro}{^{\text{\scriptsize inc},r_1}}
\newcommand{\incrt}{^{\text{\scriptsize inc},r_2}}

%%%%%%%%%%%%%%%%%%%%%%%%%%%%%%%%%%%%%%%%%%%%%%%%%%%%%%%%%%%%%%%%%%%%%%
%%%%%%%%%%%%%%%%%%%%%%%%%%%%%%%%%%%%%%%%%%%%%%%%%%%%%%%%%%%%%%%%%%%%%%
%%%%%%%%%%%%%%%%%%%%%%%%%%%%%%%%%%%%%%%%%%%%%%%%%%%%%%%%%%%%%%%%%%%%%%
\newcommand{\MMExt}{\check{\mathcal{M}}}
\newcommand{\MMInt}{\hat{\mathcal{M}}}
\newcommand{\NNExt}{\check{\mathcal{N}}}
\newcommand{\NNInt}{\hat{\mathcal{N}}}
\newcommand{\BMMExt}{\boldsymbol{\check{\mathcal{M}}}}
\newcommand{\BMMInt}{\boldsymbol{\hat{\mathcal{M}}}}
\newcommand{\BNNExt}{\boldsymbol{\check{\mathcal{N}}}}
\newcommand{\BNNInt}{\boldsymbol{\hat{\mathcal{N}}}}
\newpage

%--------------------------------------------------
%- inner products and operator matrix elements  ---
%--------------------------------------------------
\newcommand{\expval}[1]{ \big\langle #1 \big\rangle}
\newcommand{\Expval}[1]{ \Big\langle #1 \Big\rangle}
\newcommand{\inp}[2]{ \big\langle #1 \big| #2 \big\rangle}
\newcommand{\Inp}[2]{ \Big\langle #1 \Big| #2 \Big\rangle}
\newcommand{\INP}[2]{ \bigg\langle #1 \bigg| #2 \Big\rangle}
\newcommand{\vmv}[3]{ \big\langle #1 \big| #2 \big| #3 \big\rangle}
\newcommand{\VMV}[3]{ \Big\langle #1 \Big| #2 \Big| #3 \Big\rangle}

%------------------------------------------------------------
%- shaded box for source code inclusions 
%------------------------------------------------------------
%\definecolor{lightgrey}{rgb}{0.85,0.85,0.85}
%\newcommand{\SourceBox}[1]
% { \begin{mdframed}[backgroundcolor=lightgrey, linewidth=2pt,
%                    tikzsetting={draw=black, line width=2pt,
%                                 dashed, dash pattern= on 5pt off 5pt}] %
%  \begin{verbatim} #1 \end{verbatim} \end{mdframed} \medskip
%}
%\IfFileExists{mdframed.sty}
% { \usepackage{mdframed}
%   \surroundwithmdframed[backgroundcolor=lightgrey, linewidth=2pt,
%                         framemethod=tikz,
%                         skipabove=10pt,
%                         skipbelow=10pt,
%                         tikzsetting={draw=black, line width=2pt,
%                                      dashed, dash pattern= on 5pt off 5pt}
%                        ]{verbatim}
% }
% {
% }
%--------------------------------------------------
%- shaded verbatim for code inclusions
%--------------------------------------------------
\definecolor{lightgrey}{rgb}{0.85,0.85,0.85}
\newenvironment{verbcode}{\VerbatimEnvironment% 
  \noindent
  %{\columnwidth-\leftmargin-\rightmargin-2\fboxsep-2\fboxrule-4pt} 
  \begin{Sbox} 
  \begin{minipage}{\linewidth}
  \begin{Verbatim}
}{% 
  \end{Verbatim}  
   \end{minipage}
  \end{Sbox} 
  \fcolorbox{black}{lightgrey}{\textcolor{black}{\TheSbox}}
} 

%\usepackage{hyperref}

\graphicspath{{figures/}}

%------------------------------------------------------------
%------------------------------------------------------------
%- Special commands for this document -----------------------
%------------------------------------------------------------
%------------------------------------------------------------
\newcommand{\GB}{\overline{G}}
\newcommand{\vbGB}{\overline{\vb G}}
%\newcommand{\BG}{\boldsymbol{\Gamma}}
\newcommand{\erfc}{\text{erfc}}
\newcommand{\erf}{\text{erf}}
\newcommand{\wt}{\widetilde}
\newcommand{\vbGamma}{\boldsymbol{\Gamma}}
\newcommand{\kbar}{\overline{k}}

%------------------------------------------------------------
%------------------------------------------------------------
%- Document header  -----------------------------------------
%------------------------------------------------------------
%------------------------------------------------------------
\title {Implementation of Ewald Summation in {\sc scuff-em}}
\author {Homer Reid}
\date {April 16, 2014}

%------------------------------------------------------------
%------------------------------------------------------------
%- Start of actual document
%------------------------------------------------------------
%------------------------------------------------------------

\begin{document}
\pagestyle{myheadings}
\markright{Homer Reid: Ewald Summation in {\sc scuff-em}}
\maketitle

\tableofcontents

%%%%%%%%%%%%%%%%%%%%%%%%%%%%%%%%%%%%%%%%%%%%%%%%%%%%%%%%%%%%%%%%%%%%%%
%%%%%%%%%%%%%%%%%%%%%%%%%%%%%%%%%%%%%%%%%%%%%%%%%%%%%%%%%%%%%%%%%%%%%%
%%%%%%%%%%%%%%%%%%%%%%%%%%%%%%%%%%%%%%%%%%%%%%%%%%%%%%%%%%%%%%%%%%%%%%
\newpage

\section{The Periodic Green's Function}

Consider a 1D or 2D lattice consisting of a set of
lattice vectors $\{\vb L\}$.
We use the symbol $\vb p$ to denote a two-dimensional
Bloch wavevector.

The Bloch-periodic version of the scalar Helmholtz Green's function is
%====================================================================%
\numeq{PGF}
{ \GB(\vb p; \vb x) 
  \equiv 
  \sum_{\vb L} e^{i\vb p \cdot \vb L} G\Big( \big |\vb x-\vb L \big| \Big)
}
where the sum ranges over all lattice vectors $\vb L$ and
%====================================================================%
\numeq{HelmholtzKernel}
{ G(r)
  \equiv \frac{ e^{ i k r} } { 4\pi r} .
}
Note that my notation here hides the dependence of $\GB$ and $G$ 
on the photon wavenumber $k$.

\subsection*{The $\GB\sups{ABI}$ kernel}

For computation of BEM matrix elements in {\sc scuff-em} we
also need a version of $\GB$ in which the contribution
of the innermost lattice cells are excluded. [Here the
``innermost'' cells include the cell at the origin $(\vb L=0)$
and all cells within one lattice vector of the origin
in any direction.] I refer to this as the ``all but innermost''
(ABI) periodic Green's function.

For a one-dimensional lattice with lattice vector $\vb L_{0}$,
$\GB\supt{ABI}$ is given by the sum (\ref{PGF}) with three
terms excluded:
%====================================================================%
\numeq{GBABI1D}
{ \GB\supt{ABI,1D}(\vb p; \vb x) 
  \equiv 
  \sum_{|n|>1} e^{i\vb p \cdot (n\vb L_0)} G\Big( \big |\vb x-n\vb L_0 \big| \Big)
}
%====================================================================%
For a two-dimensional lattice with lattice vectors $\vb L_{01}, \vb L_{02}$,
the sum excludes 9 terms:
%====================================================================%
\numeq{GBABI2D}
{ \GB\supt{ABI,2D}(\vb p; \vb x) 
  \equiv 
  \sum_{|n_1|>1, |n_2>1|} 
  e^{i\vb p \cdot (n_1\vb L_{01} + n_2 \vb L_{02})} 
  G\Big( \big |\vb x-n_1\vb L_{01}-n_2\vb L_{02} \big| \Big)
}
%====================================================================%

\subsection*{Symmetries of $\GB$ and $\GB\supt{ABI}$}

\subsubsection*{The 1D case}

Consider a 1D lattice with lattice vector $\vb L_0=L_{0}\vbhat{x}.$
Consider a point $\vb x=(x,\vbrho)$ whose $x$ coordinate
lies outside the unit cell. Write $x=mL_0 + \overline{x}$
where $\overline{x}$ lies within the unit cell. Then we have
%====================================================================%
\begin{align*}
 \GB\supt{1D}(\vb p; x, \rho)
&= e^{i m \vb p \cdot \vb L_0} \GB\supt{1D}(\vb p; \overline{x}, \rho)
\\
 \GB\supt{ABI,1D}(\vb p; x, \rho)
&= \sum_{|n|>1} e^{i n\vb p \cdot \vb L} G\big( \overline{x}+(m-n)L_0, \rho\big)
\intertext{Change variables to $n^\prime = n-m$:}
 \GB\supt{ABI,1D}(\vb p; x, \rho)
&= e^{im\vb p \cdot \vb L}
   \sum_{|n^\prime + m|>1}
     e^{i n^\prime \vb p \cdot \vb L} G\big( \overline{x}-n^\prime L_0, \rho\big)
\\
&= e^{im\vb p \cdot \vb L}
   \sum_{|n^\prime + m|>1} T_{n^\prime}
\\
\intertext{[where $T_{n}\equiv e^{in\vb p \cdot \vb L} G(\overline{x}-nL_0, \rho)$]. 
           For the special case $m=1$ we have}
&= e^{im\vb p \cdot \vb L}
   \Big[ \GB\supt{1D} - T_{-2} - T_{-1} - T_0 \Big]
\\
&= e^{im\vb p \cdot \vb L}
   \Big[ \GB\supt{ABI1D} + T_{1} - T_{-2} \Big].
\intertext{For $m=-1$ we have instead}
&= e^{im\vb p \cdot \vb L}
   \Big[ \GB\supt{ABI1D} + T_{-1} - T_{2} \Big].
\end{align*}

%%%%%%%%%%%%%%%%%%%%%%%%%%%%%%%%%%%%%%%%%%%%%%%%%%%%%%%%%%%%%%%%%%%%%%
%%%%%%%%%%%%%%%%%%%%%%%%%%%%%%%%%%%%%%%%%%%%%%%%%%%%%%%%%%%%%%%%%%%%%%
%%%%%%%%%%%%%%%%%%%%%%%%%%%%%%%%%%%%%%%%%%%%%%%%%%%%%%%%%%%%%%%%%%%%%%
\newpage
\section{Ewald summation}
\subsection{Decompose kernel into short-range and long-range contributions}

The kernel (\ref{HelmholtzKernel}) exhibits two pathologies
which, together, make it unwieldy to work with:
\textbf{(a)} It decays slowly as $r\to\infty$, which 
ensures that the real-space sum (\ref{PGF}) is slowly 
convergent. This suggests using the Poisson summation
formula to rewrite the real-space sum as a Fourier-space
sum.
However, upon doing this we are stymied by the 
second pathology of (\ref{HelmholtzKernel}), namely,
\textbf{(b)} It is singular at $r=0$, which makes it 
long-ranged in Fourier space and thus prevents na\"ive 
application of Poisson summation.

To address this difficulty, we split the bare kernel (\ref{HelmholtzKernel}) 
into a ``short-ranged'' component which avoids 
pathology \textbf{(a)}, plus a ``long-ranged'' component
which avoids pathology \textbf{(b)}:
%====================================================================%
\begin{align}
G(r)
&= G\sups{short}(r) + G\sups{long}(r)
\\
%--------------------------------------------------------------------%
 G\sups{short}(r)
&\equiv
 \frac{1}{2\pi^{3/2}} \int_{\mc C_2}
  e^{ -u^2 r^2 + k^2/(4u^2)} \, du
\label{GShort}
\\
%--------------------------------------------------------------------%
 G\sups{long}(r)
&\equiv
 \frac{1}{2\pi^{3/2}} \int_{\mc C_1}
  e^{ -u^2 r^2 + k^2/(4u^2)} \, du
\label{GLong}
\end{align}
%====================================================================%
where $\{\mc C_1,\mc C_2\}$ are two branches of 
a certain contour in the complex plane (see Appendix).
The periodic DGF naturally decomposes into a contribution
arising primarily from nearby lattice cells plus a contribution
arising primarily from distant lattice cells
(where ``nearby'' and ``distant'' are reckoned relative to
the evaluation point $\vb x$:
%====================================================================%
\begin{align}
\GB(\vb p; \vb x) 
&=\GB\sups{nearby}(\vb p; \vb x) + \GB\sups{distant}(\vb p; \vb x)
\label{GBDecomposition} \\
\GB\sups{nearby}(\vb p; \vb x) 
 &= \sum_{\vb L} e^{i\vb p \cdot \vb L} 
    G\sups{short}\Big( \big |\vb x-\vb L \big| \Big)
\label{GNearby} \\
\GB\sups{distant}(\vb p; \vb x) 
 &= \sum_{\vb L} e^{i\vb p \cdot \vb L} 
    G\sups{long}\Big( \big |\vb x-\vb L \big| \Big).
\label{GDistant}
\end{align}
%====================================================================%

%=================================================
%=================================================
%=================================================
\subsection{Evaluate $\GB\sups{nearby}$ in real space}

The sum defining $\GB\sups{nearby}$ is now rapidly convergent
and may be evaluated directly via simple code. To this end
it is convenient to invoke the identity (\ref{PHDefinition})
to write
%====================================================================%
\begin{align}
  G\sups{short}(r)
&= \frac{1}{8\pi r}
     \Bigg\{ e^{ ikr} \erfc\left[  \eta r
                                 +i\frac{k}{2\eta}\right]
            +e^{-ikr} \erfc\left[ \eta r
                                 -i\frac{k}{2\eta}\right]
     \Bigg\}
\label{GShortConvenient} \\
&\equiv \texttt{PH}(\eta, r, k)
\end{align}
%====================================================================%
where the last line defines a convenient shorthand notation
for the function of the first line (``\texttt{PH}'' stands
for ``partial Helmholtz'').
Evaluation of $\GB\sups{nearby}$ now proceeds by straightforward
numerical summation of equation (\ref{GNearby}) using
(\ref{GShortConvenient}) to compute summand values: 
\numeq{GNearbySum}
{
 \GB\sups{nearby}(\vb p; \vb x) = \sum_{\vb L} e^{i\vb p\cdot \vb L}
   \, \texttt{PH}\Big(\eta, |\vb r - \vb L|, k\Big).
}
Note that the form of this equation is the same for the 
1D and 2D cases; only the dimension of the summation 
changes.

Typically
the partial sum converges to 10 or more decimal places after
summing $\sim 10$ terms (in the 1D case) or $\sim 100$ terms
(in the 2D case).

%=================================================
%=================================================
%=================================================
\subsection{Evaluate $\GB\sups{distant}$ in Fourier space}

On the other hand, the real-space sum defining $\GB\sups{distant}$ 
is slowly convergent, but the non-singular behavior of $G\sups{long}$
allows the use of Poisson summation to recast the sum (\ref{GDistant}) 
as a rapidly convergent sum in reciprocal space. This sum takes
slightly different forms in the 1D and 2D cases.

\subsubsection{The 1D case}

We first consider the case in which the fundamental lattice
vector is aligned with the $\vbhat{x}$ direction, i.e. 
$\vb L_0=L_{0x} \vbhat{x}.$ The extension to an arbitrary
two-dimensional lattice vector 
$\vb L_0=L_{0x}\vbhat{x} + L_{0y}\vbhat{y}$ is then immediate.

For a 1D lattice with basis vectors 
$\{\vb L = n_x L_{0x} \vbhat{x}\}$ (for all $n_x \in \mathbb{Z})$ we have 
%====================================================================%
\begin{align}
\GB\sups{distant}(\vb p; \vb x) 
 &= \sum_{\vb L} e^{i\vb p \cdot \vb L} 
    G\sups{long}\Big( \big |\vb x-\vb L \big| \Big)
\nn
 &= \sum_{n=-\infty}^\infty e^{ i n p_x L_{0x}}
    G\sups{long}\Big( \sqrt{ (x-n L_{0x})^2 + \rho^2} \Big)
\nonumber
\intertext{with $\rho^2 = y^2 + z^2.$ Introduce shorthand:}
 &\equiv \sum_{n=-\infty}^\infty f(n )
\label{RealSpaceSum1D} \\
\intertext{Now just use Poisson summation:}
 &= 2\pi
    \sum_{m=-\infty}^\infty 
    \wt f\left( 2\pi m\right)
\label{Poisson1D}
\end{align}
%====================================================================%
where $\wt f(\nu)$ is the Fourier transform of $f(n)$ with respect
to $n$. To figure out what this is, introduce the
Fourier-synthesis representation of $G\sups{long}$
[Appendix A]:
%====================================================================%
\begin{align*}
 f(n) 
&= e^{in p_x L_{0x}} G\sups{long}\Big( \sqrt{ (x-n L_{0x})^2 + \rho^2}\Big)\\
\\
&= e^{in p_x L_{0x}} \int_{-\infty}^\infty 
 \wt{G\sups{long}}(k_x; \rho) e^{ik_x(x-n L_{0x})} \, dk_x 
\\
&= 
 \int_{-\infty}^\infty 
 e^{ik_x x}
 \wt{G\sups{long}}(k_x; \rho) e^{i(p_x-k_x) n_x L_{0x}} \, dk_x 
\intertext{Change integration variables to $\nu =-(k_x-p_x) L_{0x}$:}
&= 
 \int_{-\infty}^\infty
 \underbrace{ \frac{1}{L_{0x}} 
              e^{i (p_x - \frac{\nu}{L_{0x}}) x}
              \wt{G\sups{long}}\left(p_x - \frac{\nu}{L_{0x}}; \rho\right) 
            }_{\wt f(\nu)}
 e^{i\nu n_x} \, d\nu 
\end{align*}
%====================================================================%
This identifies the Fourier transform of the function $f(n)$ 
that enters (\ref{RealSpaceSum1D}) as
\begin{align*}
   \wt f(\nu) 
 &= \frac{1}{L_{0x}} e^{i(p_x-\frac{\nu}{L_{0x}}) x}
                   \wt{G\sups{long}}\Big(p_x-\frac{\nu}{L_{0x}}; \rho\Big)
\end{align*}
and hence the sum that defines the distant contribution
to the periodic GF, equation (\ref{Poisson1D}), reads
%====================================================================%
\numeq{GDistantSum1}
{
  \overline{G}\sups{distant}(\vb p, \vb x) 
  =\frac{2\pi}{L_{0x}} \sum_{m} e^{i(p_x - \frac{2\pi m}{L_{0x}})x}
   \wt{G\sups{long}}\Big( p_x - \frac{2\pi m}{L_{0x}}; \rho\Big).
}
%====================================================================%
This derivation assumed that the fundamental lattice vector was aligned 
with the positive $x$-direction, i.e. I had $\vb L_0=L_{0x}\vbhat{x}$.
A more general form which is valid for any lattice vector $\vb L$
is 
%====================================================================%
\numeq{GDistant}
{
  \overline{G}\sups{distant}(\vb p, \vb x)
  =\mc{V}\subt{BZ}
   \sum_{m} e^{i(\vb p - m\vbGamma_0)\cdot \vb x}
          \wt{G\sups{long}}\Big( \big|\vb p - m\vbGamma_0\big|; \rho\Big)
}
%====================================================================%
where $\vbGamma_0=\frac{2\pi}{|\vb L_0|^2}\vb L_0$ 
is the fundamental lattice vector of the 1-dimensional
Brillouin zone and 
$\mc V\subt{BZ}=|\vbGamma|$ is its volume; in (\ref{GDistant}) 
the quantity $\rho$ must now be interpreted as the 
$\left| \vb x - \frac{(\vb x \cdot \vb L)}{|\vb L|^2}\vb L\right|.$

%=================================================
%=================================================
%=================================================
\subsubsection{The 2D case}

For a square 2D lattice with basis vectors 
$\vb L = n_x L_{0x} \vbhat{x} + n_y L_{0y} \vbhat{y} $ 
(for all $n_x,n_y \in \mathbb{Z})$ we have 
%====================================================================%
\begin{align}
\GB\sups{distant}(\vb p; \vb x) 
 &= \sum_{\vb L} e^{i\vb p \cdot \vb L} 
    G\sups{long}\Big( \big |\vb x-\vb L \big| \Big)
\nn
 &= \sum_{n_x, n_y=-\infty}^\infty e^{ i (n_x p_x L_{0x} + n_y p_y L_{0y})}
    G\sups{long}\Big( \sqrt{ (x-n_x L_{0x})^2 + (y-n_y L_{0y})^2 + z^2} \Big)
\nonumber
\intertext{Again introduce shorthand:}
 &\equiv \sum_{n_x, n_y=-\infty}^\infty f(n_x, n_y)
\label{RealSpaceSum2D} \\
\intertext{and again use Poisson summation:}
 &= (2\pi)^2 \sum_{m_x,m_y=-\infty}^\infty \wt f(2\pi m_x, 2\pi m_y)
\label{Poisson2D}
\end{align}
%====================================================================%
where $\wt f(\nu_x, \nu_y)$ is the two-dimensional Fourier transform of 
$f(n_x,n_y)$ with respect to $n_x,n_y$. To figure out what this is, 
introduce the Fourier-synthesis representation of $G\sups{long}$
[Appendix A]:
%====================================================================%
\begin{align*}
 f(n_x, n_y)
&= e^{i (n_x p_x L_{0x} + n_y p_y L_{0y})} 
   G\sups{long}\Big( \sqrt{ (x-n_x L_{0x})^2 + (y-n_y L_{0y})^2 + z^2}\Big)\\
\\
&= e^{i (n_x p_x L_{0x} + n_y p_y L_{0y})}
  \int
  \wt{G\sups{long}}(\vb k; z) 
  e^{ik_x(x-n_x L_{0x}) + ik_y(y-n_y L_{0y})}
   \, d\vb k
\intertext{Change integration variables to $\nu_i = (p_i-k_i)L_i$:}
&= 
 \int
 \underbrace{ \frac{1}{L_{0x} L_{0y}}
              e^{i(p_x-\frac{\nu_x}{L_{0x}})x + i(p_y-\frac{\nu_y}{L_{0y}})y}
              \wt{G\sups{long}}(p_x -\frac{\nu_x}{L_{0x}}, p_y -\frac{\nu_y}{L_{0y}}; z) 
            }_{\wt f(\nu_x, \nu_y)}
 e^{i(\nu_x n_x + \nu_y n_y)} \, d\boldsymbol{\nu}
\end{align*}
%====================================================================%
This identifies the Fourier transform of the function $f(n_x,n_y)$ 
that enters (\ref{RealSpaceSum2D}) as 
%====================================================================%
\begin{align*}
 \wt f(\nu_x, \nu_y)
 &=
 \frac{1}{L_{0x} L_{0y}}
 e^{i(p_x- \frac{\nu_x}{L_{0x}})x + i(p_y-\frac{\nu_y}{L_{0y}})y}
 \wt{G\sups{long}}(p_x-\frac{\nu_x}{L_{0x}}, p_y-\frac{\nu_y}{L_{0y}}; z) 
\\
 &=
 \frac{1}{2\pi L_{0x} L_{0y}}
 e^{i(p_x- \frac{\nu_x}{L_{0x}})x + i(p_y-\frac{\nu_y}{L_{0y}})y}
 \texttt{PH}\Big( \frac{1}{2\eta}, 
                  \sqrt{k^2 - (p_x-\frac{\nu_x}{L_{0x}})^2
                            - (p_y-\frac{\nu_y}{L_{0y}})^2 
                       },
                  \frac{z}{2}
            \Big)
\end{align*}
%====================================================================%
and hence the sum that defines the distant contribution to the
2D periodic Green's function, equation (\ref{Poisson2D}), 
reads
%====================================================================%
\begin{align*}
&\hspace{-0.1in}\GB\sups{distant}(\vb p; \vb x) 
\\
&=\frac{2\pi}{L_{0x} L_{0y}} \sum_{m_x, m_y}
    e^{i (p_x - m_x\Gamma_{0x}) x + (p_y-m_y \Gamma_{0y}) y}
    \texttt{PH}\left( \frac{1}{2\eta}, 
                      \sqrt{k^2 + (p_x-m_x \Gamma_{0x})^2 + (p_y-m_y\Gamma_{0y})^2},
                      \frac{z}{2}
               \right)
\end{align*}
%====================================================================%
where $\{\Gamma_{0x}, \Gamma_{0y}\}=\left\{\frac{2\pi}{L_{0x}}, \frac{2\pi}{L_{0y}}\right\}$.
I could alternatively write this equation as a sum over all 
2D reciprocal lattice vectors $\vbGamma$:
\numeq{GBDistant2D}
{
   \GB\sups{distant}(\vb p; \vb x) 
   =\frac{1}{2\pi}\mc V_{BZ} \sum_{\vbGamma} 
    e^{i (\vb p - \vbGamma) \cdot \vb x} \,
    \texttt{PH}\left( \frac{1}{2\eta}, 
                      \sqrt{k^2 + |\vb p - \vbGamma|^2},
                      \frac{z}{2}
               \right)
}
where $\mc V_{BZ}$ is the volume (really the area since we 
are in two dimensions) of the Brillouin zone.

Although we derived it above for the case of a square lattice,
the result in the form (\ref{GBDistant2D}) holds for any 
shape of lattice.

%====================================================================%
%====================================================================%
%====================================================================%
\appendix 

\newpage
\section{Fourier Transforms of $G\sups{long}$ in 1 and 2 Dimensions}

%=================================================
%=================================================
%=================================================
\subsection{1D}

The Fourier-synthesized form of $G\sups{long}(r)$ at 
a point $r=\sqrt{x^2+\rho^2}$ (with $\rho^2=y^2+z^2$) is
%====================================================================%
\begin{align*}
 G\sups{long}(r)
   &=G\sups{long}\Big( \sqrt{x^2 + \rho^2} \Big)
\\
   &=\int_{-\infty}^\infty \wt{G\sups{long}}(k_x; \rho) e^{ik_x x} \, dk_x
\end{align*}
%====================================================================%
where
%====================================================================%
\begin{align}
 \wt{G\sups{long}}(k_x; \rho)
   &=\frac{1}{2\pi} \int_{-\infty}^\infty 
     G\sups{long}\Big( \sqrt{x^2 + \rho^2} \Big) e^{-ik_x x} \, dx.
\nonumber
\intertext{Insert (\ref{GLong}):}
   &=\frac{1}{4\pi^{5/2}}
     \int_{\mc C_1}\, du \, e^{-u^2\rho^2 + \frac{k^2}{4u^2}}
    \underbrace{\int_{-\infty}^\infty e^{-u^2 x^2 - ik_x x} \, dx}
              _{\sqrt{\pi} \cdot u^{-1} \cdot e^{-k_x^2 / 4u^2 }}
\nn
   &=\frac{1}{4\pi^2}
     \int_{\mc C_1} \, \frac{du}{u} \, e^{-u^2 \rho^2 + (k^2-k_x^2)/(4u^2)}.
\nonumber
\intertext{Now put $k_t^2=k_x^2-k^2$ and change variables to 
           $t=\eta/u^2$, $dt=-2t du/u$:}
   &=\frac{1}{8\pi^2}
     \int_{1}^\infty \, \frac{dv}{v} \, 
      e^{-\frac{k_t^2}{4\eta^2} - \frac{ \rho^2 \eta^2}{4t} }
\nonumber
\intertext{Series-expand the quantity $e^{-(\rho^2 \eta^2)/4t}:$}
   &=\frac{1}{8\pi^2} \sum_{q=0}^\infty \frac{1}{q!}
     \left(-\frac{\rho^2 \eta^2}{4}\right)^{q}
    \underbrace{\int_1^\infty \, \frac{dt}{t^{1+q}} \, e^{-\frac{k_t^2}{4\eta^2}}}
              _{\text{E}_{1+q}\pf{k_t^2}{4\eta^2}}
\nn
   &=\frac{1}{8\pi^2} \sum_{q=0}^\infty \frac{1}{q!}
     \left( -\frac{\rho^2 \eta^2}{4} \right)^{q}
     \text{E}_{1+q}\pf{k_t^2}{4\eta^2}
\label{GLongTwiddle}
\end{align}
where $\text{E}_{1+q}$ is the exponential integral function
of order $1+q$.
%====================================================================%
In what follows we will also need the first and second partial derivatives
of $\wt{G\sups{long}}$ with respect to $\rho$. These wind up being given
by almost the same sum as (\ref{GLongTwiddle}), but with extra
factors inserted into the summand:
%====================================================================%
\begin{align*}
 \partial_\rho \wt{G\sups{long}}(k_x, \rho)
   &=\frac{1}{8\pi^2} \sum_{q=1}^\infty \frac{1}{q!}
     \pf{2q}{\rho}
     \left( -\frac{\rho^2 \eta^2}{4} \right)^{q}
     \text{E}_{1+q}\pf{k_t^2}{4\eta^2}
\\
 \partial^2_\rho \wt{G\sups{long}}(k_x, \rho)
   &=\frac{1}{8\pi^2} \sum_{q=1}^\infty \frac{1}{q!}
     \pf{2q(2q-1)}{\rho^2}
     \left( -\frac{\rho^2 \eta^2}{4} \right)^{q}
     \text{E}_{1+q}\pf{k_t^2}{4\eta^2}
\end{align*}
%====================================================================%

\subsection*{Computation in the large-$\rho$ regime}

The series (\ref{GLongTwiddle}) is poorly convergent for 
large $\rho$ (where ``large'' means ``large compared to $1/\eta.$'').
However, this is the regime in which $G\sups{long}(r)$ 
is nearly equal\footnote{By my calculations,
$G\sups{long}(r)$ and $G\sups{full}(r)$ seem to agree to 9 or more
digits whenever $r>4.5\eta.$}
to the full Helmholtz Green's function
$G\supt{full}(r)=\frac{e^{ikr}}{4\pi r},$ so we may 
approximate $\wt{G\sups{long}}$ by the 1D Fourier transform
of $G\sups{full}$:
%====================================================================%
\begin{align*}
 \wt{G\sups{full}}(k_x; \rho)
&=\frac{1}{2\pi}
  \int_{-\infty}^\infty 
  \frac{ e^{ik\sqrt{x^2 + \rho^2}}}{4\pi\sqrt{x^2 + \rho^2}}\,e^{-ik_x x} \,dx
\intertext{Insert the Fourier representation
           of $G\sups{full}$, which reads
           $\frac{e^{ik|\vb r|}}{4\pi |\vb r|}=
            \int \frac{d^3 \vb q}{(2\pi)^3}
                          \frac{e^{i\vb q \cdot \vb r}}{|\vb q|^2-k^2}:
           $
          }
&=\frac{1}{2\pi}
  \int \, dx 
  \int \frac{d\vb q}{(2\pi)^3}
  \frac{ e^{i\vb q \cdot \vb r - ik_x x} } 
       { \vb q^2 - k^2 } \\
\\
&= \int \frac{d\vb q}{(2\pi)^3} 
   \frac{ e^{i (q_y y + q_z z)}} { \vb q^2 - k^2 }
   \cdot 
   \underbrace{ \frac{1}{2\pi}
                \int \, dx \, e^{i(q_x - k_x)x}
              }_{\delta(q_x-k_q)}
\\
&= \int \frac{d \vb q_\perp}{(2\pi)^3}
   \frac{ e^{i \vb q_\perp \cdot \vbrho}} { k_x^2 + \vb q_\perp^2 - k^2 }
\\
&= \int_0^\infty \frac{q dq}{(2\pi)^3(k_x^2 + q^2 - k^2)}
   \underbrace{\int_0^{2\pi} e^{iq\rho \cos\theta}\,d\theta}
             _{2\pi J_0(q\rho)}
\\
&= \int_0^\infty \frac{q J_0(q\rho) dq}{(2\pi)^2 (k_x^2 + q^2 - k^2)}
\\
&= \frac{1}{4\pi^2} K_0\Big( [k_x^2-k^2]^{1/2} \rho \Big)
\end{align*}
%====================================================================%
where $K_0$ is a Bessel function.

Derivatives:

%====================================================================%
\begin{align*}
 \partial_\rho \wt{G\sups{full}}(k_x; \rho)
&= -\frac{k_t K_1(k_t \rho)}{4\pi^2} 
\\ 
 \partial^2_\rho \wt{G\sups{full}}(k_x; \rho)
&= \frac{k_t^2 \Big[ K_0(k_t \rho) + K_2(k_t\rho)\Big]}
        {8\pi^2}
\end{align*}
where 
$$ k_t^2 = k_x^2 - k^2.$$
%====================================================================%

%%%%%%%%%%%%%%%%%%%%%%%%%%%%%%%%%%%%%%%%%%%%%%%%%%%%%%%%%%%%%%%%%%%%%%
%%%%%%%%%%%%%%%%%%%%%%%%%%%%%%%%%%%%%%%%%%%%%%%%%%%%%%%%%%%%%%%%%%%%%%
%%%%%%%%%%%%%%%%%%%%%%%%%%%%%%%%%%%%%%%%%%%%%%%%%%%%%%%%%%%%%%%%%%%%%%
\subsection{2D}

The Fourier-synthesized form of $G\sups{long}(r)$ at
a point $r=\sqrt{x^2+y^2+z^2}$ is
%====================================================================%
\begin{align*}
 G\sups{long}(r)
   &=G\sups{long}\Big( \sqrt{x^2 + y^2 + z^2} \Big)
\\
   &=\int_{-\infty}^\infty \wt{G\sups{long}}(\vb k; z) e^{i\vb k \cdot \vb x} \, d\vb k
\end{align*}
%$$====================================================================%
where $\vb x=(x,y)$, $\vb k=(k_x, k_y)$, and 
%
%%====================================================================%
%\begin{align*}
% \wt{G\sups{long}}(\vb k; z)
%   &=\frac{1}{(2\pi)^2} \int
%     G\sups{long}\Big( \sqrt{x^2 + y^2 + z^2} \Big) e^{-i\vb k \cdot \vb x} 
%     \, d\vb x.
%\intertext{Insert (\ref{GLong}):}
%   &=\frac{1}{8\pi^{7/2}}
%     \int_{\mc C_1} \, du \, e^{-u^2 z^2 + \frac{k^2}{4u^2}}
%     \underbrace{
%     \Big[\int_{-\infty}^\infty e^{-u^2 x^2 - ik_x x} \, dx\Big]
%                }
%              _{\sqrt{\pi} \cdot \zeta^{-1} \cdot u^{-1} \cdot e^{-ik_x^2 / 4u^2 }}
%     \underbrace{
%     \Big[\int_{-\infty}^\infty e^{iu^2 y^2 - ik_x y} \, dy\Big]
%                }
%              _{\sqrt{\pi} \cdot \zeta^{-1} \cdot u^{-1} \cdot e^{-ik_y^2 / 4u^2 }}
%\\
%&=\frac{\zeta^{-1}}{8\pi^{5/2}} \int_0^\eta \frac{du}{u^2} 
%        e^{iu^2 z^2 + i(k^2 - k_x^2 - k_y^2)/(4u^2)}
%\intertext{Change variables to $s=1/(2u)$:}
%&=\frac{\zeta^{-1}}{4\pi^{5/2}} \int_{ 1/(2\eta) }^\infty 
%   e^{i(k^2 - k_x^2 - k_y^2)s^2 + iz^2/(4s^2)} \, ds
%\\[5pt]
%&=\frac{i}{2\pi} 
%  \cdot \texttt{PH}\left( \frac{1}{2\eta}, \sqrt{k^2 - |\vb k|^2}, \frac{z}{2}\right).
%\end{align*}
%%====================================================================%

%\subsection{Alternate derivation}

\begin{align*}
 \wt{G\sups{long}}(\vb k; z) 
  &=\frac{1}{(2\pi)^2} 
    \int G\sups{long}\Big(\sqrt{x^2 + y^2 + z^2}\Big) e^{-i\vb k \cdot \vb x}
         d\vb x
\\
  &=\frac{1}{(2\pi)^2} 
    \int_0^\infty \, \rho\, d\rho \,
    \int_0^{2\pi} \, d\theta \,
    G\sups{long}\Big(\sqrt{\rho^2 + z^2}\Big) e^{-i|\vb k|\rho \cos\theta}
\\
  &=\frac{1}{2\pi} 
    \int_0^\infty 
    \, 
    \rho J_0(|\vb k|\rho) G\sups{long}\Big(\sqrt{\rho^2 + z^2}\Big) \, d\rho
\\
  &=\frac{1}{4\pi^{5/2}}
    \int_{\mc C_1} \, du \, e^{-u^2 z^2 + k^2/(4u^2)} \,
    \underbrace{ \int_0^\infty \,
                 \rho J_0(|\vb k|\rho) e^{-u^2 \rho^2} \, d\rho
               }_{\frac{1}{2u^2}e^{-|\vb k|^2/(4u^2)}}
\\
  &=\frac{1}{8\pi^{5/2}}\int_{\mc C_1} \, \frac{du}{u^2} \, 
     e^{-u^2 z^2 + (k^2 - |\vb k|^2)/(4u^2)}
\intertext{Change variables to $s=1/(2u)$:}
  &= \frac{1}{4\pi^{5/2}}\int_{\mc C_2}
     e^{-(k^2 - |\vb k|^2)s^2 + z^2/(4s^2)}\,ds
\\
  &= \frac{1}{2\pi} 
     \texttt{PH}\Big(\frac{1}{2\eta}, i\sqrt{k^2-|\vb k|^2}, \frac{z}{2}\Big).
\end{align*}

%%%%%%%%%%%%%%%%%%%%%%%%%%%%%%%%%%%%%%%%%%%%%%%%%%%%%%%%%%%%%%%%%%%%%%
%%%%%%%%%%%%%%%%%%%%%%%%%%%%%%%%%%%%%%%%%%%%%%%%%%%%%%%%%%%%%%%%%%%%%%
%%%%%%%%%%%%%%%%%%%%%%%%%%%%%%%%%%%%%%%%%%%%%%%%%%%%%%%%%%%%%%%%%%%%%%
\newpage
\section{Short-distance behavior of $G\sups{long}$ in real space}

For computations of the ``all-but-3'' or ``all-but-9'' kernels
we need to compute the contributions of the innermost 3 or 9 
lattice cells to $\GB\sups{distant}$ (so that we may subtract
these real-space contributions from the Fourier-space sum 
that computes the sum over all real-space lattice cells).
This involves evaluating the kernel $G\sups{long}$ in real space.
Unfortunately, it seems there is no formula equivalent
to (\ref{GShortConvenient}) for convenient evaluation of 
$G\sups{long}$ in real space. Instead, we compute $G\sups{long}(r)$
as follows:

\begin{enumerate}

\item
For $r$ not close to zero, we simply set
%====================================================================%
\numeq{GLongCheat}
{G\sups{long}(r) = \frac{e^{ikr}}{4\pi r} - G\sups{short}(r)}
%====================================================================%
with $G\sups{short}$ computed by equation (\ref{GShortConvenient}).

\item
In the limit $r\to 0$, both terms in (\ref{GLongCheat}) 
diverge, but the difference tends to a finite constant---that
is to say, $G\sups{long}(r=0)$ is nonzero and finite. With a little
work one obtains the following small-$r$ expansion:

\begin{align*}
  G\sups{long}(r)
&= C_0 + C_2 r^2 + C_4r^4 + O(r^6)
\\[10pt]
C_0 &= \frac{\eta}{2\pi^{3/2}} e^{k^2/(4\eta^2)}
       + \frac{ik}{4\pi}
         \left[1 + \erf\left( \frac{ik}{2\eta}\right) \right]
\\[10pt]
C_2 &= -\frac{\eta (2\eta^2 + k^2)}{12\pi^{3/2}}
            e^{k^2/(4\eta^2)}
       -\frac{ik^3}{24\pi}
         \left[1 + \erf\left( \frac{ik}{2\eta}\right) \right]
\\
C_4 &= \frac{\eta (12\eta^4 + 2\eta^2 k^2 + k^4)}{240\pi^{3/2}}
            e^{k^2/(4\eta^2)}
       +\frac{ik^5}{480\pi}
         \left[1 + \erf\left( \frac{ik}{2\eta}\right) \right]
\end{align*}
where again $\zeta=e^{-i\pi/4}.$

\end{enumerate}

%%%%%%%%%%%%%%%%%%%%%%%%%%%%%%%%%%%%%%%%%%%%%%%%%%%%%%%%%%%%%%%%%%%%%%
%%%%%%%%%%%%%%%%%%%%%%%%%%%%%%%%%%%%%%%%%%%%%%%%%%%%%%%%%%%%%%%%%%%%%%
%%%%%%%%%%%%%%%%%%%%%%%%%%%%%%%%%%%%%%%%%%%%%%%%%%%%%%%%%%%%%%%%%%%%%%
\newpage
\section{Reference identities}

\subsection*{Contour-integral expression for the Helmholtz kernel}
\label{ContourIntegralSection}

%====================================================================%
\numeq{HelmholtzDecomposition}
{
 \frac{e^{ikr}}{4\pi r}
\equiv 
 \frac{1}{2\pi^{3/2}} \int_{\mc C}
  e^{-r^2u^2 + k^2/(4u^2)} \, du
}
%====================================================================%
where $\mc C$ is a contour like that pictured
in Figure \ref{ContourFigure}; the key features 
of this contour are the following:
%====================================================================%
\begin{enumerate}
 \item
 Over the interval $[0,\eta]$ on the real axis,
 the contour dips down into the lower half-plane
 with slope $\gamma$.
 \item
 Over the interval $[\eta,\infty]$ on the real axis,
 the contour pokes up into the upper half-plane.
 \item
 The slope at the origin, $\gamma$, lies in the range
 $ 0 \le \gamma \le \left(\frac{\pi}{4} - \text{arg }k \right)$
 where $\text{arg }k$ is the phase angle of the complex 
 wavenumber (Helmholtz parameter).
 \item
 The variable substitution $z\to 1/z$ transforms integrals over 
 $\mc C_1$ into integrals over $\mc C_2.$
\end{enumerate}
%====================================================================%

Property (3) here is required to ensure that the integrand remains
well-behaved as we approach $u=0$ along $\mc C$. Indeed, in the
vicinity of the origin we have 
\begin{align*}
 u &\approx e^{-i\gamma} t
\intertext{for a real-valued variable $t$ approaching $t\to 0.$ The
           exponent of the integrand in (\ref{HelmholtzDecomposition})
           then approaches}
\text{exponent}\approx \frac{k^2}{4u^2} 
 &\to |k|^2 e^{2(\text{arg }k+\gamma)} \cdot \frac{1}{4t^2}
\end{align*}
and we need the real part of this quantity to tend to 
\textit{negative} infinity so that the exponential as a whole 
tends to zero instead of blowing up, i.e. we require
%====================================================================%
$$ \text{Re }e^{2(\text{arg }k+\gamma)} < 0
   \qquad \Longrightarrow \qquad
   \frac{\pi}{2} < 2(\text{arg }k+\gamma) > \frac{3\pi}{2}
$$ 
%====================================================================%
Since we always have $0\le \text{arg }k\le \pi/2$ and we 
want to choose $\gamma$ in the range $0\le \gamma \le \frac{\pi}{2}$, 
this translates into the requirement that
$$ \gamma > \frac{\pi}{4} - \text{arg }k.$$

\subsection*{Short-ranged Helmholtz kernel}

\begin{align}
  \texttt{PH}(\eta,r,k)
  &\equiv
  \frac{1}{2\pi^{3/2}}
  \int_{\mc C_2} e^{ ir^2 u^2 + i\frac{k^2}{4u^2}}\, du
\nn
  &=
   \frac{1}{8\pi r}
     \Bigg\{ e^{ ikr} \erfc\left[  \eta r
                                 +i\frac{k}{2\eta}\right]
            +e^{-ikr} \erfc\left[  \eta r 
                                 -i\frac{k}{2\eta}\right]
     \Bigg\}
  \label{PHDefinition}
\end{align}

(Here $\mc C_2$ is the portion of the contour that 
covers the real-axis interval $[\eta,\infty]$.)
I call this function the ``partial Helmholtz'' function 
because $\texttt{PH}(\eta, r, k)$ is a sort of partial
version of the Helmholtz kernel $e^{ikr}/(4\pi r).$

%%%%%%%%%%%%%%%%%%%%%%%%%%%%%%%%%%%%%%%%%%%%%%%%%%%%%%%%%%%%%%%%%%%%%%
%%%%%%%%%%%%%%%%%%%%%%%%%%%%%%%%%%%%%%%%%%%%%%%%%%%%%%%%%%%%%%%%%%%%%%
%%%%%%%%%%%%%%%%%%%%%%%%%%%%%%%%%%%%%%%%%%%%%%%%%%%%%%%%%%%%%%%%%%%%%%
\subsection{Example contour [COMPLETE ME]}

One example of such a contour that satisfies the requirements 
enumerated in Section \label{ContourIntegralSection} is
%====================================================================%
$$ \mc C 
    = \{ \text{Re } z, \text{Im } z\}
    = \left\{ t, -4\eta\gamma \sin\left(4\, \atan \frac{t}{\eta}\right)\right\},
      \qquad 0\le t \le \infty
$$
%====================================================================%
where $\mc C_1$ and $\mc C_2$ corresponding to the parameter ranges
$t\in [0,\eta]$ and $t\in[\eta,\infty]$.

To demonstrate that this contour satisfies in particular 
property (4) in Section \ref{ContourIntegralSection}, go like 
this:
%====================================================================%
\begin{align*}
\mathcal{I}=
\int_{\mc C_1} f(z) dz
 &=\int_{0}^\eta
       f\Big[ z(t) \Big] 
       z^\prime(t) \, dt
\end{align*}
%====================================================================%
where 
%====================================================================%
$$ z(t)=t-i\gamma\sin\left(4\,\atan \frac{t}{\eta}\right), 
   \qquad
   z^\prime(t)
   =
   1 - i\frac{16\gamma\eta^2 \cos\left(4\atan\frac{t}{\eta}\right)}
                      {t^2 + \eta^2}.
$$
%====================================================================%
Now change variables to $s=\frac{\eta^2}{t}, -\frac{\eta^2}{s^2}ds=dt.$
The integral becomes
$$\mathcal{I}
  = 
  \int_\eta^\infty f\Big[z(s)\Big] z^\prime(s) \frac{\eta^2}{s^2}ds.
$$
%====================================================================%
First note that 
$$ \atan\frac{t}{\eta} = \atan\frac{\eta}{s} 
   = \frac{\pi}{2}-\atan\frac{s}{\eta}$$
and hence
$$
 \sin\left(4\,\atan\frac{t}{\eta} \right)
= - \sin\left(4\,\atan\frac{s}{\eta}\right), 
\qquad 
 \cos\left(4\,\atan\frac{t}{\eta} \right)
= + \cos\left(4\,\atan\frac{s}{\eta}\right).
$$
%====================================================================%
Thus 
$$ z^\prime(s) 
   = 
   1 - \frac{\eta^2}{s} + i\gamma\left(4\,\atan\frac{s}{\eta}\right)
$$
Also,
$$ z(t) = \frac{\eta^2}{s} + i\gamma\left(4\,\atan\frac{s}{\eta}\right)$$

%====================================================================%
\begin{align*}
 G\sups{long}(\vb r) 
&=\frac{1}{2\pi^{3/2}} 
  \int_0^\eta e^{-u^2(t) r^2 + k^2/(4u^2(t))}
  \left[ 1 - i\frac{4 \gamma \eta \cos\left(4\, \atan \frac{t}{\eta}\right)}
                   {t^2 + \eta^2}
  \right] dt
\\
 G\sups{short}(\vb r) 
&=\frac{1}{2\pi^{3/2}} 
  \int_\eta^\infty e^{-u^2(t) r^2 + k^2/(4u^2(t))}
  \left[ 1 - i\frac{4 \gamma \eta \cos\left(4\, \atan \frac{t}{\eta}\right)}
                   {t^2 + \eta^2}
  \right] dt
\end{align*}
where $u(t)=t-i\gamma\sin\left(4\,\atan\frac{t}{\eta}\right).$
%====================================================================%

%%%%%%%%%%%%%%%%%%%%%%%%%%%%%%%%%%%%%%%%%%%%%%%%%%%%%%%%%%%%%%%%%%%%%%
%%%%%%%%%%%%%%%%%%%%%%%%%%%%%%%%%%%%%%%%%%%%%%%%%%%%%%%%%%%%%%%%%%%%%%
%%%%%%%%%%%%%%%%%%%%%%%%%%%%%%%%%%%%%%%%%%%%%%%%%%%%%%%%%%%%%%%%%%%%%%
\newpage
\section{Derivatives}

\subsection{Derivatives of $\GB\sups{nearby}$}

%====================================================================%
\begin{align*}
 \frac{d}{dx_i} \GB\sups{nearby}(\vb p; \vb x)
& = \frac{x_i}{r} \sum_{\vb L} 
    e^{i \vb p \cdot \vb L}
    \left. \pard{}{r} \texttt{PH}\big(\eta, r, k\big)
    \right|_{r=|\vb r - \vb L|}
\\
\end{align*}
%====================================================================%

\subsection{Derivatives of $\GB\sups{distant}$: 1D}

In this section I assume the lattice basis vector points
in the $\vbhat{x}$ direction so that $\GB\sups{distant}$ is 
is defined by the sum (\ref{GDistantSum1}).
%====================================================================%
\begin{align*}
  \overline{G}\sups{distant}(\vb p, \vb x) 
  =\frac{2\pi}{L_{0x}} \sum_{m} e^{i Q_m x}
   \wt{G\sups{long}}\Big( Q_m; \rho\Big)
\end{align*}
where 
$$Q_m \equiv p_x - \frac{2m\pi}{L_{0x}}, \qquad \rho=\sqrt{x^2 + y^2}.$$
%====================================================================%
\begin{align*}
 \partial_x 
  \overline{G}\sups{distant}(\vb p, \vb x) 
 &=\frac{2\pi}{L_{0x}} \sum_{m} (iQ_m) e^{i Q_m x}
   \wt{G\sups{long}}\Big( Q_m; \rho\Big)
\\
 \partial_y 
  \overline{G}\sups{distant}(\vb p, \vb x) 
 &=\frac{2\pi}{L_{0x}} \pf{y}{\rho} \sum_{m} e^{i Q_m x}
   \partial_\rho \wt{G\sups{long}}\Big( Q_m; \rho\Big)
\\
 \partial_z 
  \overline{G}\sups{distant}(\vb p, \vb x) 
 &=\frac{2\pi}{L_{0x}} \pf{z}{\rho} \sum_{m} e^{i Q_m x}
   \partial_\rho \wt{G\sups{long}}\Big( Q_m; \rho\Big)
\\
 \partial_x \partial_y
  \overline{G}\sups{distant}(\vb p, \vb x) 
 &=\frac{2\pi}{L_{0x}} \pf{y}{\rho} \sum_{m} (iQ_m) e^{i Q_m x}
   \partial_\rho \wt{G\sups{long}}\Big( Q_m; \rho\Big)
\\
 \partial_x \partial_z
  \overline{G}\sups{distant}(\vb p, \vb x) 
 &=\frac{2\pi}{L_{0x}} \pf{z}{\rho} \sum_{m} (iQ_m) e^{i Q_m x}
   \partial_\rho \wt{G\sups{long}}\Big( Q_m; \rho\Big)
\\
 \partial_y \partial_z
  \overline{G}\sups{distant}(\vb p, \vb x) 
 &=\frac{2\pi}{L_{0x}} \pf{yz}{\rho^2} \sum_{m} e^{i Q_m x}
   \Big( \partial^2_{\rho\rho} -\frac{1}{\rho}\partial_\rho\Big)
   \wt{G\sups{long}}\Big( Q_m; \rho\Big)
\\
 \partial_x \partial_y \partial_z
  \overline{G}\sups{distant}(\vb p, \vb x) 
 &=\frac{2\pi}{L_{0x}} \pf{yz}{\rho^2} \sum_{m} (iQ_m) e^{i Q_m x}
   \Big( \partial^2_{\rho\rho} -\frac{1}{\rho}\partial_\rho\Big)
   \wt{G\sups{long}}\Big( Q_m; \rho\Big)
\end{align*}
where 

\subsection{Derivatives of $\GB\sups{distant}$: 2D}
\end{document}
