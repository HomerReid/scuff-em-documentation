\documentclass[letterpaper]{article}

%********************************************************
%* latex shortcuts used for libscuff documentation files
%* 
%* homer reid -- 1994--2011
%**************************************************
\usepackage{graphicx}
\usepackage{color}
\usepackage{dsfont}
\usepackage{bbm}
\usepackage{amsmath}
\usepackage{amssymb}
\usepackage{float}
\usepackage{psfrag}
\usepackage{mathdots}
%\usepackage{algorithm}
%\usepackage{algorithmic}
%\usepackage{listings}
\usepackage{array}
\usepackage{url}
\usepackage{arydshln}
\usepackage{fancybox}
\usepackage{fancyvrb}
\usepackage{verbatim}

%--------------------------------------------------
%- boldface greek letters 
%--------------------------------------------------
\newcommand\vbphi{\mathbf{\phi}}
\newcommand\vbPhi{\mathbf{\Phi}}
\newcommand{\vbDelta}{\boldsymbol{\Delta}}
\newcommand{\vbrho}{\boldsymbol{\rho}}
\newcommand{\vbchi}{\boldsymbol{\chi}}

%--------------------------------------------------
%- colors -----------------------------------------
%--------------------------------------------------
\newcommand{\red}[1]{\textcolor{red}{#1}}
\newcommand{\blue}[1]{\textcolor{blue}{#1}}
\newcommand{\green}[1]{\textcolor{green}{#1}}
\newcommand{\atan}{\text{atan}}

%--------------------------------------------------
% general commands
%--------------------------------------------------
\newcommand\Tr{\hbox{Tr }}
\newcommand\sups[1]{^{\hbox{\scriptsize{#1}}}}
\newcommand\supt[1]{^{\hbox{\tiny{#1}}}}
\newcommand\subs[1]{_{\hbox{\scriptsize{#1}}}}
\newcommand\subt[1]{_{\hbox{\tiny{#1}}}}
\newcommand{\nn}{\nonumber \\}
\newcommand{\vb}[1]{\mathbf{#1}}
\newcommand{\numeq}[2]{\begin{equation} #2 \label{#1} \end{equation}}
\newcommand{\pard}[2]{\frac{\partial #1}{\partial #2}}
\newcommand{\pardn}[3]{\frac{\partial^{#1} #2}{\partial #3^{#1}}}
\newcommand{\pf}[2]{\left(\frac{#1}{#2}\right)}
\newcommand{\pp}{{\prime\prime}}
\newcommand{\vbhat}[1]{\vb{\hat #1}}
\newcommand{\mc}[1]{\mathcal{#1}}
\newcommand{\bmc}[1]{\boldsymbol{\mathcal{#1}}}
\newcommand{\mb}[1]{\mathbb{#1}}
\newcommand{\ds}[1]{\displaystyle{#1}}
\newcommand{\primedsum}{\sideset{}{'}{\sum}}
\newcommand{\pan}{\mathcal{P}}

%--------------------------------------------------
%- special libscuff stuff--------------------------
%--------------------------------------------------
\newcommand\ls{{\sc libscuff}}
\newcommand\lss{{\sc libscuff\,}}
\newcommand{\BG}{\boldsymbol{\Gamma}}
\newcommand{\MInt}{\vb{\hat M}}
\newcommand{\NInt}{\vb{\hat N}}
\newcommand{\MExt}{\vb{\check M}}
\newcommand{\NExt}{\vb{\check N}}
\newcommand{\xInt}{\vb{\hat x}}
\newcommand{\xExt}{\vb{\check x}}

%--------------------------------------------------
%-- superscripts for dyadic green's functions 
%--------------------------------------------------
\newcommand{\EEe}{^{\hbox{\tiny{EE}\scriptsize{,$e$}}}}
\newcommand{\MEe}{^{\hbox{\tiny{ME}\scriptsize{,$e$}}}}
\newcommand{\EMe}{^{\hbox{\tiny{EM}\scriptsize{,$e$}}}}
\newcommand{\MMe}{^{\hbox{\tiny{MM}\scriptsize{,$e$}}}}
\newcommand{\EEr}{^{\hbox{\tiny{EE}\scriptsize{,$r$}}}}
\newcommand{\MEr}{^{\hbox{\tiny{ME}\scriptsize{,$r$}}}}
\newcommand{\EMr}{^{\hbox{\tiny{EM}\scriptsize{,$r$}}}}
\newcommand{\MMr}{^{\hbox{\tiny{MM}\scriptsize{,$r$}}}}
\newcommand{\incr}{^{\text{\scriptsize inc},r}}
\newcommand{\incro}{^{\text{\scriptsize inc},r_1}}
\newcommand{\incrt}{^{\text{\scriptsize inc},r_2}}

%%%%%%%%%%%%%%%%%%%%%%%%%%%%%%%%%%%%%%%%%%%%%%%%%%%%%%%%%%%%%%%%%%%%%%
%%%%%%%%%%%%%%%%%%%%%%%%%%%%%%%%%%%%%%%%%%%%%%%%%%%%%%%%%%%%%%%%%%%%%%
%%%%%%%%%%%%%%%%%%%%%%%%%%%%%%%%%%%%%%%%%%%%%%%%%%%%%%%%%%%%%%%%%%%%%%
\newcommand{\MMExt}{\check{\mathcal{M}}}
\newcommand{\MMInt}{\hat{\mathcal{M}}}
\newcommand{\NNExt}{\check{\mathcal{N}}}
\newcommand{\NNInt}{\hat{\mathcal{N}}}
\newcommand{\BMMExt}{\boldsymbol{\check{\mathcal{M}}}}
\newcommand{\BMMInt}{\boldsymbol{\hat{\mathcal{M}}}}
\newcommand{\BNNExt}{\boldsymbol{\check{\mathcal{N}}}}
\newcommand{\BNNInt}{\boldsymbol{\hat{\mathcal{N}}}}
\newpage

%--------------------------------------------------
%- inner products and operator matrix elements  ---
%--------------------------------------------------
\newcommand{\expval}[1]{ \big\langle #1 \big\rangle}
\newcommand{\Expval}[1]{ \Big\langle #1 \Big\rangle}
\newcommand{\inp}[2]{ \big\langle #1 \big| #2 \big\rangle}
\newcommand{\Inp}[2]{ \Big\langle #1 \Big| #2 \Big\rangle}
\newcommand{\INP}[2]{ \bigg\langle #1 \bigg| #2 \Big\rangle}
\newcommand{\vmv}[3]{ \big\langle #1 \big| #2 \big| #3 \big\rangle}
\newcommand{\VMV}[3]{ \Big\langle #1 \Big| #2 \Big| #3 \Big\rangle}

%------------------------------------------------------------
%- shaded box for source code inclusions 
%------------------------------------------------------------
%\definecolor{lightgrey}{rgb}{0.85,0.85,0.85}
%\newcommand{\SourceBox}[1]
% { \begin{mdframed}[backgroundcolor=lightgrey, linewidth=2pt,
%                    tikzsetting={draw=black, line width=2pt,
%                                 dashed, dash pattern= on 5pt off 5pt}] %
%  \begin{verbatim} #1 \end{verbatim} \end{mdframed} \medskip
%}
%\IfFileExists{mdframed.sty}
% { \usepackage{mdframed}
%   \surroundwithmdframed[backgroundcolor=lightgrey, linewidth=2pt,
%                         framemethod=tikz,
%                         skipabove=10pt,
%                         skipbelow=10pt,
%                         tikzsetting={draw=black, line width=2pt,
%                                      dashed, dash pattern= on 5pt off 5pt}
%                        ]{verbatim}
% }
% {
% }
%--------------------------------------------------
%- shaded verbatim for code inclusions
%--------------------------------------------------
\definecolor{lightgrey}{rgb}{0.85,0.85,0.85}
\newenvironment{verbcode}{\VerbatimEnvironment% 
  \noindent
  %{\columnwidth-\leftmargin-\rightmargin-2\fboxsep-2\fboxrule-4pt} 
  \begin{Sbox} 
  \begin{minipage}{\linewidth}
  \begin{Verbatim}
}{% 
  \end{Verbatim}  
   \end{minipage}
  \end{Sbox} 
  \fcolorbox{black}{lightgrey}{\textcolor{black}{\TheSbox}}
} 


\graphicspath{{figures/}}

%------------------------------------------------------------
%------------------------------------------------------------
%- Special commands for this document -----------------------
%------------------------------------------------------------
%------------------------------------------------------------$
\newcommand{\FSV}{\boldsymbol{\mathcal{F}}}   % field six-vector
\newcommand{\CSV}{\boldsymbol{\mathcal{C}}}   % current six-vector
\newcommand{\BSV}{\boldsymbol{\mathcal{B}}}   % basis-function six-vector
\newcommand{\SSV}{\boldsymbol{\mathcal{S}}}   % source six-vector

\newcommand{\NM}{\boldsymbol{\mathcal{N}}}
\newcommand{\SVDGF}{\boldsymbol{\mathcal{G}}} % six-vector DGF

\newcommand{\SVChi}{\boldsymbol{\chi}}        % six-vector susceptibility

%------------------------------------------------------------
%------------------------------------------------------------
%- Document header  -----------------------------------------
%------------------------------------------------------------
%------------------------------------------------------------
\title {SIE-BEM Formulations in {\sc scuff-em}}
\author {Homer Reid}
\date {March 29, 2013}

%------------------------------------------------------------
%------------------------------------------------------------
%- Start of actual document
%------------------------------------------------------------
%------------------------------------------------------------

\begin{document}
\pagestyle{myheadings}
\markright{Homer Reid: SIE-BEM Formulations in {\sc scuff-em}}
\maketitle

\tableofcontents

Consider a single dielectric object in a medium.   
I define $\vbhat{n}$ to be the outward pointing normal vector
and $\{\vb K, \vb N\}=
\{\vbhat{n}\times \vb H\sups{tot}, -\vbhat{n}\times\vb E\sups{tot}\}$
to be the surface currents seen from the exterior medium.
The quantities $\vbhat{n}, \vb{K}, \vb{N}$ all suffer a sign flip
when reckoned from the interior medium; I will not define
separate notation for the sign-flipped version of these
quantities.

Six-vector notation for fields, surface currents, and dyadic Green's
functions:
$$ \FSV=\left(\begin{array}{c} \vb E \\ \vb H \end{array}\right), \qquad
   \CSV=\left(\begin{array}{c} \vb K \\ \vb N \end{array}\right), \qquad
   \SVDGF=\left(\begin{array}{cc} 
      \BG\supt{EE} & \BG\supt{EM}  \\
      \BG\supt{ME} & \BG\supt{MM} 
          \end{array}\right).
$$
The total fields in the exterior and interior regions are related
to the surface currents and the incident fields (which we assume
to arise from sources in the exterior medium) according to
\begin{subequations}
\begin{align}
 \FSV\subs{ext}\sups{tot} &= \SVDGF\subs{ext} * \CSV + \FSV\sups{inc} \\
 \FSV\subs{int}\sups{tot} &= -\SVDGF\subs{int} * \CSV
\end{align}
\label{FGC}
\end{subequations}
On the other hand, the surface currents are related to the 
total fields in the two regions according to
\begin{subequations}
\begin{align}
 \CSV &=  \NM\FSV\sups{tot}\subs{ext}
\\
 \CSV &= -\NM\FSV\sups{tot}\subs{int}
\intertext{where}
 \NM &=
   \left(\begin{array}{cc} 
   0               & \vbhat{n}\times \\
  -\vbhat{n}\times & 0
   \end{array} \right).
\end{align}
\label{CNF}
\end{subequations}
%
Combining (\ref{CNF}) with (\ref{FGC}) yields 
two distinct six-vector integral equations
relating $\CSV$ to $\FSV\sups{inc}$:
\begin{subequations}
\begin{align}
\end{align}
\end{subequations}


\footnote{This way of understanding the PMCHW and n-Muller
formulations follows P. Yla-Oijala and M. Taskinen,
``Well-conditioned Muller formulation for electromagnetic 
scattering by dielectric objects,'' \textit{IEEE Transactions 
on Antennas and Propagation}, \textbf{53} 3316 (2005).}

%%%%%%%%%%%%%%%%%%%%%%%%%%%%%%%%%%%%%%%%%%%%%%%%%%%%%%%%%%%%%%%%%%%%%%
%%%%%%%%%%%%%%%%%%%%%%%%%%%%%%%%%%%%%%%%%%%%%%%%%%%%%%%%%%%%%%%%%%%%%%
%%%%%%%%%%%%%%%%%%%%%%%%%%%%%%%%%%%%%%%%%%%%%%%%%%%%%%%%%%%%%%%%%%%%%%
\newpage
\end{document}

@ARTICLE{Helsinki2005,
author={Yla-Oijala, P. and Taskinen, M.}, 
journal={Antennas and Propagation, IEEE Transactions on}, 
title={Well-conditioned Muller formulation for electromagnetic scattering by dielectric objects}, 
year={2005}, 
volume={53}, 
number={10}, 
pages={3316-3323}, 
keywords={Galerkin method;dielectric bodies;electric breakdown;electric field integral equations;electromagnetic wave scattering;iterative methods;magnetic field integral equations;matrix algebra;method of moments;surface electromagnetic waves;Galerkin method;MoM;N-Muller;RWG basis function;Rao-Wilton-Glisson;electromagnetic scattering;electrostatic integral equation;homogeneous dielectric object;iterative solution;low-frequency breakdown;magnetostatic integral equation;method of moments;surface integral equation;well-conditioned Muller formulation;well-conditioned matrix equation;Dielectric breakdown;Electric breakdown;Electromagnetic scattering;Electrostatics;Helium;Integral equations;Iterative methods;Magnetostatic waves;Microwave frequencies;Moment methods;Dielectric object;MÜller formulation;electromagnetic scattering;low frequency;method of moments (MoM);surface integral equation}, 
doi={10.1109/TAP.2005.856313}, 
ISSN={0018-926X},}
