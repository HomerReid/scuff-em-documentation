\documentclass{article} 
\usepackage{bbding} 
\input{physcmds}
\graphicspath{{figures/}}

%%%%%%%%%%%%%%%%%%%%%%%%%%%%%%%%%%%%%%%%%%%%%%%%%%%%%%%%%%%%%%%%%%%%%%
% special commands for this document %%%%%%%%%%%%%%%%%%%%%%%%%%%%%%%%% 
%%%%%%%%%%%%%%%%%%%%%%%%%%%%%%%%%%%%%%%%%%%%%%%%%%%%%%%%%%%%%%%%%%%%%%

\newcommand{\bs}[1]{\boldsymbol{#1}}
\newcommand{\BG}{\boldsymbol{\Gamma}}

\newcommand\eee{^{ \text{{\tiny EE}},e}}
\newcommand\eme{^{ \text{{\tiny EM}},e}}
\newcommand\mee{^{ \text{{\tiny ME}},e}}
\newcommand\mme{^{ \text{{\tiny MM}},e}}
\newcommand\een{^{ \text{{\tiny EE}},n}}
\newcommand\emn{^{ \text{{\tiny EM}},n}}
\newcommand\men{^{ \text{{\tiny ME}},n}}
\newcommand\mmn{^{ \text{{\tiny MM}},n}}

\newcommand\eeo{^{ \text{{\tiny EE,1}}}}
\newcommand\emo{^{ \text{{\tiny EM,1}}}}
\newcommand\meo{^{ \text{{\tiny ME,1}}}}
\newcommand\mmo{^{ \text{{\tiny ME,1}}}}
\newcommand\eet{^{ \text{{\tiny EE,2}}}}
\newcommand\emt{^{ \text{{\tiny EM,2}}}}
\newcommand\met{^{ \text{{\tiny ME,2}}}}
\newcommand\mmt{^{ \text{{\tiny ME,2}}}}

\newcommand\eees{^{ \text{{\tiny EE}},e*}}
\newcommand\emes{^{ \text{{\tiny EM}},e*}}
\newcommand\mees{^{ \text{{\tiny ME}},e*}}
\newcommand\mmes{^{ \text{{\tiny MM}},e*}}
\newcommand\eens{^{ \text{{\tiny EE}},n*}}
\newcommand\emns{^{ \text{{\tiny EM}},n*}}
\newcommand\mens{^{ \text{{\tiny ME}},n*}}
\newcommand\mmns{^{ \text{{\tiny MM}},n*}}

\renewcommand{\inp}[2]{ \big\langle #1 \big| #2 \big\rangle}
\newcommand{\INP}[2]{ \Big\langle #1 \Big| #2 \Big\rangle}
\newcommand{\vmv}[3]{ \big\langle #1 \big| #2 \big| #3 \Big\rangle}
\newcommand{\VMV}[3]{ \Big\langle #1 \Big| #2 \Big| #3 \Big\rangle}

%**************************************************
%* Document header info ***************************
%**************************************************
\title{SIE Approach to Fresnel Scattering}
\author {Homer Reid}
\date {May 3, 2012}

%**************************************************
%* Start of actual document ***********************
%**************************************************

\begin{document}
\maketitle

\pagestyle{myheadings}
\markright{Homer Reid: SIE Approach to Fresnel Scattering}

\abstract{I revisit the theory of Fresnel scattering (that is,
the scattering of a plane wave from an infinite planar dielectric
interface)  from the perspective of the surface-integral-equation 
approach to electromagnetic scattering. The idea is to develop
some intuition for the physical significance of effective
electric and magnetic surface currents by demonstrating how
they arise in the simplest analytically-solvable scattering
problem.}

\tableofcontents 

%%%%%%%%%%%%%%%%%%%%%%%%%%%%%%%%%%%%%%%%%%%%%%%%%%%%%%%%%%%%%%%%%%%%%%
%%%%%%%%%%%%%%%%%%%%%%%%%%%%%%%%%%%%%%%%%%%%%%%%%%%%%%%%%%%%%%%%%%%%%%
%%%%%%%%%%%%%%%%%%%%%%%%%%%%%%%%%%%%%%%%%%%%%%%%%%%%%%%%%%%%%%%%%%%%%%
\newpage
\section{Scattering from a Planar Surface: Exact Solution}

\subsection{Normal Incidence}

Consider an infinite dielectric half-space filling the region $z<0,$
and a plane wave normally incident from above.
\numeq{NormalPlaneWave}
{
  \vb E\sups{inc}(\vb x) = E_0 e^{-i k_0 z} \vbhat{x}  \qquad
  \vb H\sups{inc}(\vb x) = -\frac{1}{Z_0} E_0 e^{-i k_0 z} \vbhat{y}
}
The scattered fields above the $xy$ plane, and the total fields below 
the $xy$ plane, are 
%====================================================================%
$$\begin{array}{lcllcl}
 \displaystyle{
  \vb E\sups{scat,above}(\vb x) 
              }
 &=
 &
 \displaystyle{
   A e^{+ik_0z}\vbhat{x}, 
              }
\qquad  
 &\displaystyle{
  \vb H\sups{scat,above}(\vb x) 
               }
 &= 
 &\displaystyle{
  \frac{1}{Z_0} A e^{+ik_0z}\vbhat{y}, 
               }
\\[10pt]
%--------------------------------------------------------------------%
 \displaystyle{
  \vb E\sups{tot,below}(\vb x) 
              }
 &=
 &
 \displaystyle{
   B e^{-ik_1z}\vbhat{x}, 
              }
\qquad  
 &\displaystyle{
  \vb H\sups{tot,below}(\vb x) 
               }
 &= 
 &\displaystyle{
  -\frac{1}{Z_0 Z^1} B e^{-ik_1z}\vbhat{y}
               }
\end{array}$$
%====================================================================%
where $Z_0$ is the impedance of free space, 
$Z^1$ is the dimensionless relative wave impedance 
of the half-space, and $A$ and $B$ are unknown 
coefficients to be determined.

Matching tangential fields at $z=0$ yields
\begin{align*} 
 E_0 + A &= B, \\
 E_0 - A &= \frac{1}{Z^1}B,
\end{align*} 
with solution
$$ A=\frac{Z^1-1}{Z^1+1}E_0, \qquad 
   B=\frac{2Z^1}{Z^1+1} E_0.
$$
For $Z^1=1$ (the dielectric half-space isn't there) 
we have $A=0, B=1$, so there is no reflected wave
and the full wave is transmitted. 

\subsubsection*{Surface Currents}

\numeq{KNExact}
{ \begin{array}{lclcl}
     \vb K
 &=& \displaystyle{
        \vbhat z \times \vb H\sups{tot} 
                  }
 &=& \displaystyle{
    +\frac{2E_0}{Z_0(Z^1+1)}\vbhat{x}
                  }
\\[15pt]
     \vb N
 &=& \displaystyle{
       -\vbhat z \times \vb E\sups{tot}
                  }
 &=& \displaystyle{
      -\frac{2Z^1E_0}{(Z^1+1)}\vbhat{y}
                  }
\end{array}}

%=================================================
%=================================================
%=================================================
\subsection{Non-Normal Incidence, $\vb E$ perpendicular to plane of incidence}

I now rotate the propagation vector through an angle $\theta$ around
the $y$ axis. (For $\theta=0$ I recover the results of the previous
section.)
The wavevectors of the incident, reflected, and transmitted waves
are 
\begin{align*}
 \vbhat{k}^i &= k_0\Big( \sin\theta \vbhat{y} - \cos\theta\vbhat{z} \Big) \\
 \vbhat{k}^r &= k_0\Big( \sin\theta \vbhat{y} + \cos\theta\vbhat{z} \Big) \\
 \vbhat{k}^t &= k_1\Big( \sin\theta^\prime \vbhat{y} + \cos\theta^\prime\vbhat{z} \Big)\\
\end{align*}
where $\theta^\prime$ is related to the incident angle by 
$$ k_0\sin\theta = k_1 \sin \theta^\prime. $$
%
Incident wave:
%
\begin{align}
\vb E\sups{inc}(\vb x) 
 &= E_0 e^{ik_0[y\sin\theta - z\cos\theta ]} \, \vbhat{x} 
\\
\vb H\sups{inc}(\vb x) 
 &= \frac{E_0}{Z_0} 
    e^{ik_0[y\sin\theta  - z\cos\theta]} 
    \Big( -\cos\theta\,\vbhat{y} - \sin\theta\,\vbhat{z}\Big)
\label{NonNormalIncidentEPerp}
\end{align}
%
Reflected wave:
%
\begin{align}
\vb E\sups{refl}(\vb x)
 &= rE_0 e^{ik_0[y\sin\theta  + z\cos\theta]} \, \vbhat{x} 
\\
\vb H\sups{refl}(\vb x)
 &= \frac{rE_0}{Z_0} 
    e^{ik_0[y\sin\theta  + z\cos\theta ]}
    \Big( +\cos\theta\,\vbhat{y} - \sin\theta\,\vbhat{z}\Big)
\label{NonNormalReflectedEPerp}
\end{align}
%
Transmitted wave:
%
\begin{align*}
\vb E\sups{trans}(\vb x)
 &= tE_0 e^{ik_1[y\sin\theta^\prime  + z\cos\theta^\prime]} \, \vbhat{x} 
\\
\vb H\sups{trans}(\vb x)
 &= \frac{tE_0}{Z_1} 
    e^{ik_1[y\sin\theta^\prime  + z\cos\theta^\prime]}
    \Big( -\cos\theta^\prime\,\vbhat{y} - \sin\theta^\prime\,\vbhat{z}\Big)
\label{NonNormalTransmittedEPerp}
\end{align*}

\subsection*{Surface currents}

\begin{align*}
  \vb K &= \frac{E_0}{Z_0} (1-r)\cos\theta e^{ik_0 y \sin \theta} 
            \, \vbhat{x} 
\\
  \vb N &= E_0 (1+r) e^{ik_0 y \sin \theta} \, \vbhat{y} 
\end{align*}
%
Let's check that the normal fields at the surface are 
correctly recovered from the divergence-of-surface-current 
prescription:

\begin{align*}
  E_z &= \frac{\nabla \cdot \vb K}{i\epsilon\omega} = 0
\qquad \text{\CheckmarkBold}
\\
  H_z & = -\frac{\nabla \cdot \vb N}{i\mu\omega} 
        = -\frac{i E_0 k_0 \sin\theta(1+r)}{i\mu\omega} e^{ik_0 y\sin\theta}
        = -\frac{E_0}{Z_0}(1+r) e^{ik_0 y \sin\theta} 
\qquad \text{\CheckmarkBold}
\end{align*}

%=================================================
%=================================================
%=================================================
\subsection{Non-Normal Incidence, $\vb E$ parallel to plane of incidence}

%
Incident wave:
%
\begin{align*}
\vb E\sups{inc}(\vb x) 
 &= E_0 
    e^{ik_0[y\sin\theta  - z\cos\theta ]} 
    \Big( \cos\theta\,\vbhat{y} + \sin\theta\,\vbhat{z}\Big)
\\
\vb H\sups{inc}(\vb x) 
 &= \frac{E_0}{Z_0} 
    e^{ik_0[y\sin\theta -z \cos\theta]} \,\vbhat{x}
\end{align*}
%
Reflected wave:
%
\begin{align*}
\vb E\sups{refl} 
 &= rE_0 e^{ik_0[y\sin\theta + z\cos\theta ]} 
    \Big( -\cos\theta\,\vbhat{y} + \sin\theta\,\vbhat{z}\Big)
\\
\vb H\sups{refl}(\vb x) 
 &=  r \frac{E_0}{Z_0} 
       e^{ik_0[y\sin\theta +z \cos\theta]} \,\vbhat{x}
\end{align*}
%
Transmitted wave:
%
\begin{align*}
\vb E\sups{trans}(\vb x) 
 &= tE_0 e^{ik_1[y\sin\theta^\prime - z\cos\theta^\prime ]} 
    \Big( \cos\theta^\prime\,\vbhat{y} + \sin\theta^\prime\,\vbhat{z}\Big)
\\
\vb H\sups{trans}(\vb x) 
 &= \frac{tE_0}{Z_1} 
    e^{ik_0[y\sin\theta^\prime -z \cos\theta^\prime]} \,\vbhat{x}
\end{align*}

\subsection*{Surface currents}

\begin{align*}
  \vb K &= \frac{E_0}{Z_0} (1+r) e^{ik_0 y \sin \theta} \, \vbhat{y} 
\\
  \vb N &= E_0 (1-r) \cos\theta e^{ik_0 y \sin \theta} \, \vbhat{x} 
\end{align*}
%
Let's check that the normal fields at the surface are 
correctly recovered from the divergence-of-surface-current 
prescription:

\begin{align*}
  E_z &= \frac{\nabla \cdot \vb K}{i\epsilon\omega} = E_0(1+r)\sin\theta
\qquad \text{\CheckmarkBold}
\\
  H_z & = -\frac{\nabla \cdot \vb N}{i\mu\omega} 
        = 0
\qquad \text{\CheckmarkBold}
\end{align*}

%%%%%%%%%%%%%%%%%%%%%%%%%%%%%%%%%%%%%%%%%%%%%%%%%%%%%%%%%%%%%%%%%%%%%%
%%%%%%%%%%%%%%%%%%%%%%%%%%%%%%%%%%%%%%%%%%%%%%%%%%%%%%%%%%%%%%%%%%%%%%
%%%%%%%%%%%%%%%%%%%%%%%%%%%%%%%%%%%%%%%%%%%%%%%%%%%%%%%%%%%%%%%%%%%%%%
\newpage
\section{Scattering from a Planar Surface: SIE solution}

I now re-solve the same problem using the SIE method.

\subsection{Surface-Current Basis Functions}

Notation: 

\begin{itemize}
 \item $\vb x =(x,y)$ is a two-component coordinate vector.
 \item $\vb r =(x,y,z)=(\vb x, z)$ is a three-component coordinate vector.
 \item $\vb p  $ is a two-component Fourier vector.
\end{itemize}

There are two types of surface-current basis functions 
for surface currents at $z=0$:
$$ \vb f_{x\vb p}(\vb x) = e^{i\vb p \cdot \vb x} \vbhat{x}, \qquad
   \vb f_{y\vb p}(\vb x) = e^{i\vb p \cdot \vb x} \vbhat{y}
$$

The electric and magnetic surface currents at $z=0$ are 
expanded in the $\vb f_{\vb p}$ basis:
\begin{align*}
  \vb K(\vb x) 
&= \sum_{\vb p}
   \Big[   K_{x\vb p} \vb f_{x\vb p}(\vb x)  
         + K_{y\vb p} \vb f_{y\vb p}(\vb x)  
   \Big]
\\[3pt]
  \vb N(\vb x) 
&= \sum_{\vb p}
   \Big[   N_{x\vb p} \vb f_{x\vb p}(\vb x)  
         + N_{y\vb p} \vb f_{y\vb p}(\vb x)  10
   \Big]
\end{align*}
where  
$$ \sum_{\vb p} = \int\frac{d^2 \vb p}{(2\pi)^2}. $$

\subsection{Convolutions of surface currents with $\vb G$ and $\vb C$ dyadics:}

I first recall my conventions for the dyadic green's functions: In homogeneous
medium $r$, the fields arising from surface currents are
\begin{align*}
    \vb E(\vb x) 
&= \int\bigg\{ 
   \BG^{\text{\tiny{EE}},r}(\vb x, \vb x^\prime) 
   \cdot 
   \vb K(\vb x^\prime) 
  +\BG^{\text{\tiny{EM}},r}(\vb x, \vb x^\prime) 
   \cdot 
   \vb N(\vb x^\prime) \bigg\} d\vb x^\prime
\\[10pt]
    \vb H(\vb x) 
&= \int\bigg\{ 
   \BG^{\text{\tiny{ME}},r}(\vb x, \vb x^\prime) 
   \cdot 
   \vb K(\vb x^\prime) 
  +\BG^{\text{\tiny{MM}},r}(\vb x, \vb x^\prime) 
   \cdot 
   \vb N(\vb x^\prime) \bigg\} d\vb x^\prime
\end{align*}
where 
%====================================================================%
$$\begin{array}{lcllcl}
 \BG^{\text{\tiny{EE}},r}
 &=&ik_r Z_0 Z^r \vb G, 
 \qquad 
 &\BG^{\text{\tiny{EM}},r}
 &=&+ik_r \vb C, 
\\
%--------------------------------------------------------------------%
 \BG^{\text{\tiny{ME}},r}
 &=&-ik_r \vb C, 
 \qquad 
 &\BG^{\text{\tiny{MM}},r}
 &=&+\frac{ik_r}{Z_0Z^r} \vb G, 
\\
\end{array}$$
%====================================================================%
In the following, I put 
$$ Q \equiv Q(k,\vb p) = \sqrt{k^2 - |\vb p|^2}.$$
Note that $Q$ is imaginary for $|\vb p|>k$. At a temporal
frequency of $\omega=ck,$ the fields arising from surface 
currents with spatial frequency $|\vb p|>k$ are evanescent 
waves.

The inner products I need are 

\begin{align*}
 \int \vb G(k; \vb r, \vb r^\prime) \cdot \vb f_{x\vb p}(\vb r^\prime)
d\vb r^\prime
&=\frac{ ie^{i\vb p \cdot \vb x} e^{iQ|z|}}{2Q}
       \left[   \left(1-\frac{p_x^2}{k^2}\right) \vbhat{x}
               -\frac{p_x p_y}{k^2} \vbhat{y}
               -\frac{iQ p_x}{k^2} \vbhat{z}
       \right]
\\[5pt]
 \int \vb G(k; \vb r, \vb r^\prime) \cdot \vb f_{y\vb p}(\vb r^\prime)
d\vb r^\prime
&=\frac{ ie^{i\vb p \cdot \vb x} e^{iQ|z|}}{2Q}
       \left[  -\frac{p_x p_y}{k^2} \vbhat{x}
               +\left(1-\frac{p_y^2}{k^2} \right) \vbhat{y}  
               -\frac{iQ p_y}{k^2} \vbhat{z}
       \right]
\\[5pt]
 \int \vb C(k; \vb r, \vb r^\prime) \cdot \vb f_{x\vb p}(\vb r^\prime)
d\vb r^\prime
&=   \text{sign }(z)
   \frac{ ie^{i\vb p \cdot \vb x} e^{iQ|z|}}{2k}
   \left[ \vbhat{y} + \frac{ip_y}{Q} \vbhat{z} \right]
\\[5pt]
 \int \vb C(k; \vb r, \vb r^\prime) \cdot \vb f_{y\vb p}(\vb r^\prime)
d\vb r^\prime
&= - \text{sign }(z)
   \frac{ ie^{i\vb p \cdot \vb x} e^{iQ|z|}}{2k}
   \left[ \vbhat{x} - \frac{ip_x}{Q} \vbhat{z} \right]
\end{align*}

Note that the tangential component of the $\vb C$ integrals 
(that is, the magnetic field due to an electric surface current, 
or vice versa) is discontinuous at $z=z^\prime$.

\subsection{Inner Products of $\vb f_{\vb p}$ functions with dyadic GFs}

Note: In what follows I am being a little cavalier. The four-dimensional
integrals in a quantity like 
$\vmv{\vb f_{\vb p}}{\vb G}{\vb f_{\vb p}}$ are actually
infinite: if I restricted each surface integral to a finite 
area $A$, then the matrix element would be proportional to 
$A$ and would diverge with the area of the planar interface.
The proper way to write this is to say
$$  \vmv{\vb f_{\vb p}}{\vb G}{\vb f_{\vb p^\prime}}
  = \overline{\vmv{\vb f_{\vb p}}{\vb G}{\vb f_{\vb p}}}
    \cdot (2\pi)^2 \delta(\vb p-\vb p^\prime)
$$
where the $\delta$ function factor has units of area,
and the barred inner product is finite. In what follows I
will not bother to write out the overbar or the 
$\delta$ function factor, but it's good to keep them
in mind.

I have verified the following calculations numerically:
\begin{align*} 
 \VMV{ \vb f_{x\vb p}(\vb x) }
     { \vb G(\vb x, z; \vb x^\prime, z^\prime) }
     { \vb f_{x\vb p}(\vb x^\prime) }
 &= \frac{ ie^{iQ|z-z^\prime|} }{2Q}\left(1-\frac{p_x^2}{k^2}\right)
\\
 \VMV{ \vb f_{x\vb p}(\vb x) }
     { \vb G(\vb x, z; \vb x^\prime, z^\prime) }
     { \vb f_{y\vb p}(\vb x^\prime) }
 &= -\frac{ ie^{iQ|z-z^\prime|} }{2Q} \cdot \frac{p_x p_y}{k^2}
\\
 \VMV{ \vb f_{y\vb p}(\vb x) }
     { \vb G(\vb x, z; \vb x^\prime, z^\prime) }
     { \vb f_{x\vb p}(\vb x^\prime) }
 &= -\frac{ ie^{iQ|z-z^\prime|} }{2Q} \cdot \frac{p_x p_y}{k^2}
\\
 \VMV{ \vb f_{y\vb p}(\vb x) }
     { \vb G(\vb x, z; \vb x^\prime, z^\prime) }
     { \vb f_{y\vb p}(\vb x^\prime) }
 &= \frac{ ie^{iQ|z-z^\prime|} }{2Q}\left(1-\frac{p_y^2}{k^2}\right)
\end{align*} 
\begin{align*} 
 \VMV{ \vb f_{x\vb p}(\vb x) }
     { \vb C(\vb x, z; \vb x^\prime, z^\prime) }
     { \vb f_{y\vb p}(\vb x^\prime) }
 &=  \text{sign}(z-z^\prime) \frac{ i e^{iQ|z-z^\prime|} }{2k}
\\
 \VMV{ \vb f_{y\vb p}(\vb x) }
     { \vb C(\vb x, z; \vb x^\prime, z^\prime) }
     { \vb f_{x\vb p}(\vb x^\prime) }
 &= -\text{sign}(z-z^\prime) \frac{ i e^{iQ|z-z^\prime|} }{2k}
\end{align*} 
%
Note that, at $z\to z^\prime$, the $\vmv{\vb f}{\vb C}{\vb f}$
inner products approach a nonzero constant whose sign depends 
on whether $z\to z^{\prime +}$ or $z\to z^{\prime -}$. This 
subtlety turns out to be unimportant for setting up and 
solving scattering problems using the SIE formalism, but 
crucial for understanding the compact power formulas discussed
in the next section.

%=====================================================================
%=====================================================================
%=====================================================================
\newpage
\subsection{SIE Matrices}

In this section I omit the $\vb p$ subscript.

I order the expansion coefficients and the incident-field inner
products into 4-dimensional vectors thusly:
\numeq{SIEVectors}
{ \left(\begin{array}{c}
 K_x \\[3pt] K_y \\[3pt] N_x \\[3pt] N_y
 \end{array}\right),
\qquad
 \left(\begin{array}{c}
 \inp{\vb f_{x}}{\vb E\sups{inc}} \\[3pt]
 \inp{\vb f_{y}}{\vb E\sups{inc}} \\[3pt]
 \inp{\vb f_{x}}{\vb H\sups{inc}} \\[3pt]
 \inp{\vb f_{y}}{\vb H\sups{inc}}
 \end{array}\right).
}
Then the SIE matrix has the following structure:
$$ \vb M=
   \left(\begin{array}{cccc}
   \vmv{ \vb f_{x} } { \BG\supt{EE} } { \vb f_x }
  &\vmv{ \vb f_{x} } { \BG\supt{EE} } { \vb f_y }
  &\vmv{ \vb f_{x} } { \BG\supt{EM} } { \vb f_x }
  &\vmv{ \vb f_{x} } { \BG\supt{EM} } { \vb f_y }
 \\[3pt]
   \vmv{ \vb f_{y} } { \BG\supt{EE} } { \vb f_x }
  &\vmv{ \vb f_{y} } { \BG\supt{EE} } { \vb f_y }
  &\vmv{ \vb f_{y} } { \BG\supt{EM} } { \vb f_x }
  &\vmv{ \vb f_{y} } { \BG\supt{EM} } { \vb f_y }
 \\[3pt]
   \vmv{ \vb f_{x} } { \BG\supt{ME} } { \vb f_x }
  &\vmv{ \vb f_{x} } { \BG\supt{ME} } { \vb f_y }
  &\vmv{ \vb f_{x} } { \BG\supt{MM} } { \vb f_x }
  &\vmv{ \vb f_{x} } { \BG\supt{MM} } { \vb f_y }
 \\[3pt]
   \vmv{ \vb f_{y} } { \BG\supt{ME} } { \vb f_x }
  &\vmv{ \vb f_{y} } { \BG\supt{ME} } { \vb f_y }
  &\vmv{ \vb f_{y} } { \BG\supt{MM} } { \vb f_x }
  &\vmv{ \vb f_{y} } { \BG\supt{MM} } { \vb f_y }
   \end{array}\right).
$$
The contribution of medium $m$ to the SIE matrix is 
%====================================================================%
$$ \vb M^{(m)}=
   ik_m 
   \left(\begin{array}{cccc}
    Z_0 Z^m
    \vmv{ \vb f_{x} } { \vb G } { \vb f_x }
  & Z_0 Z^m
    \vmv{ \vb f_{x} } { \vb G } { \vb f_y }
  & \vmv{ \vb f_{x} } { \vb C } { \vb f_x }
  & \vmv{ \vb f_{x} } { \vb C } { \vb f_y }
 \\[3pt]
%--------------------------------------------------------------------%
    Z_0 Z^m
    \vmv{ \vb f_{y} } { \vb G } { \vb f_x }
  & Z_0 Z^m
    \vmv{ \vb f_{y} } { \vb G } { \vb f_y }
  & \vmv{ \vb f_{y} } { \vb C } { \vb f_x }
  & \vmv{ \vb f_{y} } { \vb C } { \vb f_y }
 \\[3pt]
%--------------------------------------------------------------------%
    -\vmv{ \vb f_{x} } { \vb C } { \vb f_x }
  & -\vmv{ \vb f_{x} } { \vb C } { \vb f_y }
  & \frac{1}{Z_0 Z^m} \vmv{ \vb f_{x} } { \vb G } { \vb f_x }
  & \frac{1}{Z_0 Z^m} \vmv{ \vb f_{x} } { \vb G } { \vb f_y }
 \\[3pt]
%--------------------------------------------------------------------%
    -\vmv{ \vb f_{y} } { \vb C } { \vb f_x }
  & -\vmv{ \vb f_{y} } { \vb C } { \vb f_y }
  & \frac{1}{Z_0 Z^m} \vmv{ \vb f_{y} } { \vb G } { \vb f_x }
  & \frac{1}{Z_0 Z^m} \vmv{ \vb f_{y} } { \vb G } { \vb f_y }
   \end{array}\right).
$$
%====================================================================%
Inserting the results of the last section, the exterior and interior
contributions to the SIE matrix are 
%====================================================================%
$$ \vb M^{(0)}=
   -\frac{k_0}{2}
   \left(\begin{array}{cccc}
    \frac{Z_0}{Q_0}\left(1-\frac{p_x^2}{k_0^2}\right)
  &
    \frac{Z_0}{Q_0}\left(-\frac{p_xp_y}{k_0^2}\right)
  & 0 
  & \frac{1}{k_0}
 \\[6pt]
%--------------------------------------------------------------------%
    \frac{Z_0}{Q_0}\left(-\frac{p_x p_y}{k_0^2}\right)
  &
    \frac{Z_0}{Q_0}\left(1-\frac{p_y^2}{k_0^2}\right)
  & -\frac{1}{k_0}
  & 0 
 \\[6pt]
%--------------------------------------------------------------------%
    0
  & -\frac{1}{k_0}
  & \frac{1}{Q_0 Z_0} \left(1-\frac{p_x^2}{k_0^2}\right)
  & \frac{1}{Q_0 Z_0} \left(-\frac{p_xp_y }{k_0^2}\right)
 \\[6pt]
%--------------------------------------------------------------------%
    \frac{1}{k_0}
  & 0
  & \frac{1}{Q_0 Z_0} \left(-\frac{p_xp_y}{k_0^2}\right)
  & \frac{1}{Q_0 Z_0} \left(1-\frac{p_y^2}{k_0^2}\right)
   \end{array}\right).
$$
%====================================================================%
%====================================================================%
$$ \vb M^{(1)}=
   -\frac{k_1}{2}
    \left(\begin{array}{cccc}
    \frac{Z_0Z^1}{Q_1}\left(1-\frac{p_x^2}{k_1^2}\right)
  &
    \frac{Z_0Z^1}{Q_1}\left(-\frac{p_xp_y}{k_1^2}\right)
  & 0 
  & -\frac{1}{k_1}
 \\[6pt]
%--------------------------------------------------------------------%
    \frac{Z_0Z^1}{Q_1}\left(-\frac{p_x^2}{k_1^2}\right)
  &
    \frac{Z_0Z^1}{Q_1}\left(1-\frac{p_y^2}{k_1^2}\right)
  & +\frac{1}{k_1}
  & 0 
 \\[6pt]
%--------------------------------------------------------------------%
    0
  & +\frac{1}{k_1}
  & \frac{1}{Q_1 Z_0 Z^1} \left(1-\frac{p_x^2}{k_1^2}\right)
  & \frac{1}{Q_1 Z_0 Z^1} \left(-\frac{p_xp_y }{k_1^2}\right)
 \\[6pt]
%--------------------------------------------------------------------%
   -\frac{1}{k_1}
  & 0
  & \frac{1}{Q_1 Z_0 Z^1} \left(-\frac{p_xp_y}{k_1^2}\right)
  & \frac{1}{Q_1 Z_0 Z^1} \left(1-\frac{p_y^2}{k_1^2}\right)
   \end{array}\right).
$$
%====================================================================%

%=====================================================================
%=====================================================================
%=====================================================================
\subsection{SIE Scattering Solution: Normal Incidence}
For the normally incident plane wave of equation 
(\ref{NormalPlaneWave}), the RHS vector of the SIE system 
[again being somewhat cavalier about neglecting 
factors of $\delta(\vb p-\vb p^\prime)$] is 
$$-
 \left(\begin{array}{c}
 \inp{\vb f_{x}}{\vb E\sups{inc}} \\[3pt]
 \inp{\vb f_{y}}{\vb E\sups{inc}} \\[3pt]
 \inp{\vb f_{x}}{\vb H\sups{inc}} \\[3pt]
 \inp{\vb f_{y}}{\vb H\sups{inc}}
 \end{array}\right)
\,=\,
  -E_0 
 \left(\begin{array}{c}
 1 \\ 
 0 \\ 
 0 \\ 
 -\frac{1}{Z_0}
 \end{array}\right).
$$
Out of the four unknown coefficients in the unknown
vector (\ref{SIEVectors}), only $K_x$ and $N_y$ are 
nonzero, and then only for $\vb p=0$, in which
case I have simply $Q=k$.
The SIE system reduces to 

$$ 
 -\frac{1}{2}
 \left(\begin{array}{cc}
 \displaystyle{ \vphantom{\frac{1}{Z_0}} Z_0(Z^1+1) } & 0 \\
 0 & \displaystyle{\frac{1}{Z_0(\frac{1}{Z^1} + 1)}}
 \end{array} \right)
 \left(\begin{array}{c}
 \displaystyle{ \vphantom{\frac{1}{Z_0}} K_x } 
 \\
 \displaystyle{ \vphantom{\frac{1}{Z_0}} N_y } 
 \end{array}\right)
=
 -E_0
 \left(\begin{array}{c}
 \displaystyle{ \vphantom{\frac{1}{Z_0}} 1 }
 \\
 \displaystyle{ -\frac{1}{Z_0} }
 \end{array}\right)
$$ 
with solution
$$ K_x = \frac{2E_0}{Z_0(Z^1+1)}, \qquad 
   N_y = -\frac{2 Z^1 E_0 }{(Z^1+1)}.
$$
This reproduces the exact result (\ref{KNExact}).

%%%%%%%%%%%%%%%%%%%%%%%%%%%%%%%%%%%%%%%%%%%%%%%%%%%%%%%%%%%%%%%%%%%%%%
%%%%%%%%%%%%%%%%%%%%%%%%%%%%%%%%%%%%%%%%%%%%%%%%%%%%%%%%%%%%%%%%%%%%%%
%%%%%%%%%%%%%%%%%%%%%%%%%%%%%%%%%%%%%%%%%%%%%%%%%%%%%%%%%%%%%%%%%%%%%%
\newpage
\section{An Important Subtlety in the SIE matrices}

Before proceeding, I pause to note the following important subtlety
in the SIE matrices. The subtlety is easiest to see 
for the case $\vb p=0$, although it is present for $\vb p\ne 0$  
as well.

By the results of the previous section, the exterior and
interior contributions to the SIE matrices at $\vb p=0$ 
are as follows:
%====================================================================%
\begin{align*}
 \vb M^{(0)}
&=\left(\begin{array}{cccc}
   \hspace{0.07in} -\frac{Z_0}{2}\hspace{0.07in}
  &
    0
  & 0 
  & -\frac{1}{2}
 \\[6pt]
%--------------------------------------------------------------------%
    0
  &\hspace{0.07in} -\frac{Z_0}{2}\hspace{0.07in}
  & \frac{1}{2}
  & 0 
 \\[6pt]
%--------------------------------------------------------------------%
    0
  &  \frac{1}{2}
  & \hspace{0.07in} -\frac{1}{2Z_0}\hspace{0.07in}
  & -\frac{1}{2Z_0}
 \\[6pt]
%--------------------------------------------------------------------%
    -\frac{1}{2}
  & 0
  & 0
  & \hspace{0.07in} -\frac{1}{2Z_0}\hspace{0.07in}
   \end{array}\right).
\\[10pt]
%--------------------------------------------------------------------%
\vb M^{(1)}
&=
   \left(\begin{array}{cccc}
    -\frac{Z_0Z^1}{2}
  &
    0
  & 0 
  & \frac{1}{2}
 \\[6pt]
%--------------------------------------------------------------------%
    0
  & -\frac{Z_0Z^1}{2}
  & -\frac{1}{2}
  & 0 
 \\[6pt]
%--------------------------------------------------------------------%
    0
  & -\frac{1}{2}
  & -\frac{1}{2Z_0Z^1}
  & 0
 \\[6pt]
%--------------------------------------------------------------------%
    \frac{1}{2}
  & 0
  & 0
  & -\frac{1}{2Z_0Z^1}
   \end{array}\right).
\end{align*}
%====================================================================%
The total SIE matrix is the sum of the exterior and interior contributions:
\begin{align*}
 \vb M
&=
   \left(\begin{array}{cccc}
    -\frac{Z_0}{2}(1+Z^1)
  &
    0
  & 0 
  & 0
 \\[6pt]
%--------------------------------------------------------------------%
    0
  & -\frac{Z_0}{2}(1+Z^1)
  & 0
  & 0 
 \\[6pt]
%--------------------------------------------------------------------%
    0
  & 0
  & -\frac{1}{2Z_0}(1+\frac{1}{Z^1})
  &
 \\[6pt]
%--------------------------------------------------------------------%
    0
  & 0
  & 0
  & -\frac{1}{2Z_0}(1+\frac{1}{Z^1})
   \end{array}\right).
\end{align*}
Note something interesting here: The off-diagonal elements of 
$\pm \frac{1}{2}$, which are present in the \textit{individual}
contributions to the SIE matrix, cancel when we form the sum 
and are \textit{absent} from the final SIE matrix.

\subsubsection*{Origin of the $\pm\frac{1}{2}$ factors}

The factors of $\pm \frac{1}{2}$ come from a $\delta$ function
in the $\vb C$ dyadic Green's function. The proper way to 
write inner products involving the $\vb C$ function is something
like this:
\numeq{CFinite}
{
   \VMV{\vb f}{\vb C}{\vb g} 
   = 
     \VMV{\vb f}{\vb C\sups{finite}}{\vb g} 
     \,\pm\, 
     \frac{1}{2ik}\cdot \VMV{\vb f}{\vbhat{n}\times}{\vb g} 
}
The first term here gives zero inner for coplanar surface-current 
basis functions $\vb f$ and $\vb g$, but the second term 
is nonvanishing whenever $\vb f$ and $\vb g$ have crossed
overlap (in fact, this term is proportional to what Steven
calls the ''$T$ matrix.'') The choice of $\pm$ sign is 
determined by whether we are taking the limit as $\vb f$ approaches
$\vb g$ from the \textit{exterior} or the \textit{interior}. 

\subsubsection*{Implications for concise power formulas in {\sc scuff-em}}

The additional $\delta$ function term here is
\textbf{not taken into account by {\sc scuff-em.}}
Indeed, as noted above, it is perfectly acceptable 
to ignore this term as long as we are only ever computing
the \textit{total} SIE matrices. 
However, when we need to compute \textit{individual} 
contributions to the SIE matrices, we need to account for
the extra $\delta$ function term. 

Computationally, this is actually easy, because the crossed-overlap
integrals $\vmv{\vb f}{\vbhat{n}\times}{\vb g}$ can be computed in 
closed form for the RWG basis functions.

%%%%%%%%%%%%%%%%%%%%%%%%%%%%%%%%%%%%%%%%%%%%%%%%%%%%%%%%%%%%%%%%%%%%%%
%%%%%%%%%%%%%%%%%%%%%%%%%%%%%%%%%%%%%%%%%%%%%%%%%%%%%%%%%%%%%%%%%%%%%%
%%%%%%%%%%%%%%%%%%%%%%%%%%%%%%%%%%%%%%%%%%%%%%%%%%%%%%%%%%%%%%%%%%%%%%
\newpage
\section{Comparison of Power Formulas}

\subsection{Exact Results for Transmitted, Scattered, and Total Power}

It is easy to read off expressions for the various power quantities 
from the exact solution of Section 1.

\subsubsection*{Transmitted (Absorbed) Power}
The total Poynting vector at $z=0$ is 
$$
 \vb P\sups{tot}
=\frac{1}{2}\,\text{Re }
 \Big[\vb E^{\text{\scriptsize{tot}}*} \times \vb H\sups{tot}\Big]
=-\text{Re } \frac{|B|^2}{2Z_0 Z^1}\vbhat{z}
$$
so the power per unit area transmitted through the infinite planar 
surface is 
\numeq{Ptrans}
{ P\sups{trans}
  =\frac{4 \cdot (\text{Re }Z_1)}{|Z_1+1|^2} \cdot \frac{|E_0|^2}{2Z_0}.
}
This quantity is the analogue of what we normally compute as 
the absorbed power for a finite scatterer; I call it $P\sups{trans}$
instead of $P\sups{abs}$ because it contains both the power absorbed 
by the dielectric (if it is lossy) and the flux of the plane wave 
travelling through the dielectric toward $z\to-\infty$. (In 
particular, it is nonzero even for a lossless dielectric, which
would not be the case for a finite scatterer.)

\subsubsection*{Scattered Power}
On the other hand, the scattered Poynting vector at $z=0$ is 
$$
 \vb P\sups{scat}
=\frac{1}{2}\,\text{Re }
 \Big[\vb E^{\text{\scriptsize{scat}}*} \times \vb H\sups{scat}\Big]
=+\text{Re } \frac{|A|^2}{2Z_0}\vbhat{z}
$$
so the power scattered per unit area by the infinite planar 
surface is 
\numeq{PScat}
{ P\sups{scat}
  =\left|\frac{Z^1-1}{Z^1 + 1}\right|^2 \cdot \frac{|E_0|^2}{2Z_0}.
}

\subsubsection*{Total Power}
The analogue of what we normally call the ``total power''
(or the ``extinction'') is the sum of the transmitted and 
scattered powers:
\numeq{Ptot}
{
  P\sups{tot} = P\sups{scat} + P\sups{abs}
=\frac{|E_0|^2}{2Z_0}.
}
Note that this is actually equal to the entire flux of 
the incident plane wave, i.e. 100\% of its power its
``extinguished'' by the interaction with the dielectric
half-space. This just reflects the fact that, in contrast
to the case of a finite scatterer, in this case there is
no way for a portion of the incident power to flow ``around''
the scatterer; all power is either reflected or transmitted.

\subsection{Concise SIE Formulas for Transmitted, Scattered, and Total Power}

\subsubsection{Transmitted power by HR formula}

The HR formula for the transmitted power is, in SGJ notation,
$$ P^{\text{{\scriptsize trans},{\sc hr} }}=\frac{1}{4}x^* T x $$
where the elements of the $T$ matrix are the crossed-overlaps
of the basis functions:
$$ T_{\alpha\beta} = \VMV{\vb f_\alpha}{\vbhat{n}\times}{\vb f_\beta}.$$
In the present case this works out to 
\begin{align*}
 P^{\text{{\scriptsize trans},{\sc hr} }}
& = -\frac{1}{2}\text{Re } K^*_x N_y \VMV{\vb f_x}{\vbhat{z}\times}{\vb f_y}
\\
& = \frac{4\cdot(\text{Re }Z^1)}{|Z^1+1|^2}\cdot \frac{|E_0|^2}{2Z_0}
\end{align*}
in agreement with the exact solution (\ref{Ptrans}).

\subsubsection{Absorbed power by SGJ formula}

The SGJ formula for the absorbed power reads (in SGJ notation)
\numeq{PabsSGJ1}
{
 P\sups{abs,{\sc sgj}}
 =\frac{1}{4}\text{Re }\Big[ x^* G^0 x + x^* s^0 \Big]
}
The first term here is 
%====================================================================%
\begin{align}
 \frac{1}{4}\,\text{Re }\,x^* G^0 x 
&= 
 -\frac{1}{8}\text{ Re }
 \left(\begin{array}{c}
 \displaystyle{ \frac{2E_0}{Z_0(Z^1+1)} }
 \\[5pt]
 \displaystyle{ -\frac{2Z^1 E_0}{Z^1+1} }
 \end{array}\right)^\dagger 
%--------------------------------------------------------------------%
 \left(\begin{array}{cc}
 \displaystyle{ \vphantom{\frac{1}{Z_0}} Z_0 } & -\frac{1}{2}
 \\[5pt]
 -\frac{1}{2} & \displaystyle{\frac{1}{Z_0}}
 \end{array} \right)
%--------------------------------------------------------------------%
 \left(\begin{array}{c}
 \displaystyle{ \frac{2E_0}{Z_0(Z^1+1)} }
 \\[5pt]
 \displaystyle{ -\frac{2Z^1 E_0}{Z^1+1} }
 \end{array}\right)
\nonumber\\[10pt]
%--------------------------------------------------------------------%
&= -\frac{|E_0|^2}{8Z_0}\text{ Re }
   \bigg\{   \underbrace{ \frac{4}{|Z^1+1|^2} }_{\text{EE}} 
          \,+\,\underbrace{ \frac{2Z^1}{|Z^1+1|^2} }_{\text{EM}} 
          \,+\,\underbrace{ \frac{2Z^{1*}}{|Z^1+1|^2} }_{\text{ME}}
          \,+\,\underbrace{ \frac{4|Z^1|^2}{|Z^1+1|^2} }_{\text{MM}}
   \bigg\}
\nonumber\\[10pt]
&= -\frac{|E_0|^2}{2Z_0}
   \bigg\{ 1 - \frac{4(\text{Re} Z^1)}{|Z^1+1|^2} \bigg\}
\label{xGxo4}
\end{align}
%====================================================================%
The second term is 
\begin{align}
&= \frac{1}{4}\text{Re }
 \left(\begin{array}{c}
 \displaystyle{ \frac{2E_0}{Z_0(Z^1+1)} }
 \\[8pt]
 \displaystyle{ -\frac{2Z^1 E_0}{Z^1+1} }
 \end{array}\right)^\dagger 
 \cdot 
 \left(\begin{array}{c}
 \displaystyle{ \vphantom{\frac{1}{Z_0}} E_0 }
 \\[8pt]
 \displaystyle{ -\frac{E_0}{Z_0} }
 \end{array}\right)
\nn
&= \frac{|E_0|^2}{2Z_0}
\label{xso4}
\end{align}
Inserting (\ref{xGxo4}) and (\ref{xso4}) into (\ref{PabsSGJ1}), 
I find
\begin{align*}
 P\sups{abs,{\sc sgj}}
 &=  \frac{1}{4}\text{Re }\Big[ x^* G^0 x \Big]
   +\frac{1}{4}\text{Re }\Big[ x^* s^0 \Big]
\\
 &= +\frac{4\cdot(\text{Re }Z^1)}{|Z^1+1|^2}\cdot \frac{|E_0|^2}{2Z_0}
\end{align*}
again reproducing the correct result.

\subsubsection{Contribution of $\delta$-function terms to SGJ formula}

Let's now investigate what would happen if the weird off-diagonal
factors of $\frac{1}{2}$ were missing from the individual SIE
matrices (as is the case in {\sc scuff-em}). This would correspond
to eliminating the terms marked ``EM'' and ``ME'' from the sum in
equation (\ref{xGxo4}). The contribution made by these terms to the 
sum is 
\begin{align*}
\Delta P\sups{abs,{\sc sgj}} 
  &= -\frac{\text{Re }Z^1}{|Z^1+1|^2}\cdot \frac{|E_0|^2}{2Z_0}
\\
  &=\frac{1}{4}P\sups{abs,{\sc sgj}}.
\end{align*}
So eliminating these terms from the sum would result in an answer
that is off by a factor of $\frac{3}{4}$, not $\frac{1}{2}$ as I seem
to observe in practice.


%%%%%%%%%%%%%%%%%%%%%%%%%%%%%%%%%%%%%%%%%%%%%%%%%%%%%%%%%%%%%%%%%%%%%%
%%%%%%%%%%%%%%%%%%%%%%%%%%%%%%%%%%%%%%%%%%%%%%%%%%%%%%%%%%%%%%%%%%%%%%
%%%%%%%%%%%%%%%%%%%%%%%%%%%%%%%%%%%%%%%%%%%%%%%%%%%%%%%%%%%%%%%%%%%%%%
\newpage
\section{Comparison of Force Formulas}

\subsection*{Exact force for $\vb E-$field perpendicular to plane}

In general, the time average of the $i$-directed force per unit 
area at a point ($\vb x$) on a surface is related to the Maxwell 
stress tensor at $\vb x$ according to 
\begin{align}
 F_i 
&= -\frac{1}{2}\text{Re }T_{ij}(\vb x) \hat n_j (\vb x)
\nonumber
\intertext{(where $\vbhat{n}$ is the \textit{outward}-pointing 
            surface normal)} 
&= -\frac{1}{2}\text{Re }\bigg\{ 
     \epsilon E_i^* \big[\vb E \cdot \vbhat{n}\big] 
    +\mu      H_i^* \big[\vb H \cdot \vbhat{n}\big] 
    -\frac{\hat n_i}{2}\Big[ \epsilon|\vb E|^2 + \mu|\vb H|^2\Big]
   \bigg\}.
\label{Fi}
\end{align}
For the particular case of the non-normally-incident 
plane wave of (\ref{NonNormalIncidentEPerp}), we have
$\vbhat{n}=+\vbhat{z}$ and the fields at $z=0$ are 
$$ 
   \left(\begin{array}{c}
   E_x \\ E_y \\ E_z \\ H_x \\ H_y \\ H_z
   \end{array}\right)
   =
   E_0 e^{ik_0 y\sin\theta}
   \left(\begin{array}{c}
   (1+r) \\ 0 \\ 0 \\ 0 \\ -\frac{1}{Z_0}(1-r)\cos\theta \\ -\frac{1}{Z_0}(1+r)\sin\theta
   \end{array}\right)
$$
with
$$ \vb E\cdot \vbhat{n}=0, 
   \qquad 
   \vb H\cdot \vbhat{n}=-\frac{E_0}{Z_0}(1+r)\sin\theta e^{ik_0 y\sin\theta}, 
$$
$$
   |\vb E|^2=|E_0|^2 |1+r|^2, 
   \qquad 
   |\vb H|^2=\frac{|E_0|^2}{Z_0^2}\Big[ |1-r|^2 \cos^2\theta + |1+r|^2 \sin^2 \theta\Big]
$$
so from (\ref{Fi}) we find
\begin{align*}
 F_x &= 0
\\
 F_y &= -\frac{\epsilon_0 |E_0|^2}{2}(1-|r|^2)\cos\theta\sin\theta
\\
 F_z &= -\frac{\epsilon_0 |E_0|^2}{2}\text{Re }
         \bigg\{   |1+r|^2\sin^2\theta 
                 - \frac{1}{2}|1+r|^2 
                 - \frac{1}{2}\Big[|1-r|^2 \cos^2 \theta + |1+r|^2 \sin^2\theta\Big]
         \bigg\}
\\
&= -\frac{\epsilon_0 |E_0|^2}{4}\cos^2 \theta \, \text{Re }\Big[ |1+r|^2 - |1-r|^2\Big]
\\
&= -\epsilon_0 |E_0|^2\cos^2 \theta \cdot\Big(\text{Re }r\Big).
\end{align*}
The force is proportional to the real part of the reflection coefficient,
and it vanishes for an incident field propagating in the direction parallel
to the interface.

\subsection*{FSC Force for $\vb E$-field perpendicular to plane}

The formula for the normally-directed stress tensor in terms of
the surface currents is 
%====================================================================%
\begin{align}
 T_{ij}n_j 
&= \frac{1}{i\omega}
   \bigg\{ (\vbhat{n} \times \vb N^*)_i(\nabla \cdot \vb K)
           +
           (\vbhat{n} \times \vb K^*)_i(\nabla \cdot \vb N)
   \bigg\}
\nn
&\qquad
 -\frac{\hat{n}_i}{2}
  \bigg\{      \mu\Big[ |\vb K|^2 - \frac{|\nabla \cdot \vb K|^2}{k^2}\Big]
         +\epsilon\Big[ |\vb N|^2 - \frac{|\nabla \cdot \vb N|^2}{k^2}\Big]
  \bigg\}
\label{TijNjKN}
\end{align}
%====================================================================%
In this case we have 
$$ K_x = \frac{E_0}{Z_0}(1-r)\cos\theta e^{ik_0 y \sin \theta},
   \qquad
   N_y = E_0(1+r)e^{ik_0 y \sin \theta}
$$ 
with all other components of the surface currents vanishing.
Also, 
$$ \nabla \cdot \vb K =0, 
   \qquad 
   |\vb K|^2 = \frac{|E_0|^2}{Z_0^2}|1-r|^2\cos^2\theta, 
$$
$$
   \nabla \cdot \vb N = ik_0 \sin\theta E_0 (1+r)e^{ik_0 y\sin\theta}, 
   \qquad 
   |\vb N|^2 = |E_0|^2|1+r|^2.
$$
The components of equation (\ref{TijNjKN}) then read
\begin{align*}
 T_{xj} n_{j}
&=0
\\
 T_{yj} n_{j}
&=\frac{1}{i\omega}\bigg\{ \Big[\frac{E_0^*}{Z_0}(1-r)^* \cos\theta \Big]
                           \Big[ik_0 E_0(1+r) \sin\theta\Big]\bigg\}
\\
&=\epsilon_0 |E_0|^2 (1-r)^*(1+r) \cos\theta\sin\theta
\\
 T_{zj} n_{j}
&=-\frac{\epsilon_0 |E_0|^2}{2}
   \Big\{  |1-r|^2 \cos^2\theta
          +|1+r|^2 
          -|1+r|^2 \sin^2\theta \Big\}
\\
&=-\frac{\epsilon_0 |E_0|^2}{2}\cos^2\theta 
   \Big\{  |1-r|^2 + 
          +|1+r|^2 
          -|1+r|^2 \sin^2\theta 
   \Big\}
\end{align*}
 
\end{document}
