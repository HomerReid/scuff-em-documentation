\documentclass[letterpaper]{article}
\usepackage[square,sort,comma,numbers]{natbib}
%====================================================================%
\input{scufftex}

\newcommand\supsstar[1]{^{\hbox{\scriptsize{#1}}*}}
\newcommand\suptstar[1]{^{\hbox{\scriptsize{#1}}*}}
\newcommand{\iD}{_{i\text{\tiny D}}}
\newcommand{\jS}{_{j\text{\tiny S}}}

\newcommand{\OP}{\vb O\sups{power}}
\newcommand{\OG}{\overline{G}}
\newcommand{\SG}{\overline{\vb G}}
\newcommand{\IF}{^{i\text{\scriptsize F}}}
\newcommand{\IT}{^{i\text{\scriptsize T}}}
\newcommand{\lm}{_{\ell m}}
\newcommand{\lmp}{_{\ell^\prime m^\prime}}
\newcommand{\OutD}{^{\hbox{\scriptsize{out}}\dagger}}

\renewcommand{\hslash}{\,\backslash\hspace{-0.06in}h}
\newcommand{\jslash}{\backslash\hspace{-0.05in}j}
\newcommand{\Rbar}{\overline{R}}
\newcommand{\Rslash}{\backslash\hspace{-0.08in}R}

\newcommand{\regstar}{^{\hbox{\scriptsize reg}*}}

\graphicspath{{figures/}}

%------------------------------------------------------------
%------------------------------------------------------------
%- Special commands for this document -----------------------
%------------------------------------------------------------
%------------------------------------------------------------

%------------------------------------------------------------
%------------------------------------------------------------
%- Document header  -----------------------------------------
%------------------------------------------------------------
%------------------------------------------------------------
\title {Electromagnetism in the Spherical-Wave Basis: \\
        {\large A (Somewhat Random) Compendium of Reference Formulas}
       }
\author {Homer Reid}
\date {August 1, 2016}

%------------------------------------------------------------
%------------------------------------------------------------
%- Start of actual document
%------------------------------------------------------------
%------------------------------------------------------------

\begin{document}
\pagestyle{myheadings}
\markright{Homer Reid: E\&M in the Spherical-Wave Basis}
\maketitle

\begin{abstract}
This memo consolidates and collects for reference
a somewhat random hodgepodge of formulas and results
in the spherical-wave approach to electromagnetism
that I have found useful over the years in developing
and testing {\sc scuff-em} and {\sc buff-em}.
\end{abstract}

\tableofcontents

%%%%%%%%%%%%%%%%%%%%%%%%%%%%%%%%%%%%%%%%%%%%%%%%%%%%%%%%%%%%%%%%%%%%%%
%%%%%%%%%%%%%%%%%%%%%%%%%%%%%%%%%%%%%%%%%%%%%%%%%%%%%%%%%%%%%%%%%%%%%%
%%%%%%%%%%%%%%%%%%%%%%%%%%%%%%%%%%%%%%%%%%%%%%%%%%%%%%%%%%%%%%%%%%%%%%
\newpage
\section{Vector Spherical Wave Solutions to Maxwell's Equations}
\label{SphericalMaxwellAppendix}

Many authors define pairs of three-vector-valued functions
$\{\vb M_{\ell m}(\vb x), \vb N_{\ell m}(\vb x)\}$
describing exact solutions of Maxwell's equations 
in spherical coordinates for 
a homogeneous medium with wavenumber $k$, i.e.
%====================================================================%
$$\Big[\nabla \times \nabla \times + k^2 \Big]
  \left\{\begin{array}{c} \vb M \\ \vb N\end{array}\right\} = 0.$$
%====================================================================%
These functions always involve spherical Bessel functions and
spherical harmonics, but the precise definitions (including
sign conventions and normalization factors) vary from author
to author. In this section I set down the particular
conventions that I use. In the next section I give
explicit closed-form expressions for small $\ell$.

\paragraph{Vector spherical harmonics}
%====================================================================%
\begin{align*}
  \vb X\lm(\theta, \varphi) 
&= \frac{i}{\ell(\ell+1)}\nabla \times 
     \Big\{Y\lm(\theta, \varphi) \vbhat{r}\Big\}
\\
%--------------------------------------------------------------------%
  \vb Z\lm(\theta, \varphi) &= \vbhat{r} \times \vb X\lm(\theta, \varphi)
\end{align*}
%====================================================================%

\noindent More explicitly, the components of $\vb X$ and $\vb Z$ are
%====================================================================%
\begin{align*}
\vb X\lm(\theta, \phi) 
&= 
   \frac{i}{\sqrt{\ell(\ell+1)}}
   \left[  \frac{im}{\sin\theta}Y_{\ell m} \vbhatt{\theta}
           -\pard{Y_{\ell m}}{\theta}\vbhatt{\varphi}
   \right]
\\
\vb Z\lm(\theta, \phi) 
&=
   \frac{i}{\sqrt{\ell(\ell+1)}}
   \left[   \pard{Y_{\ell m}}{\theta}\vbhatt{\theta}
           +\frac{im}{\sin\theta}Y_{\ell m} \vbhatt{\varphi}
   \right].
\end{align*}
%====================================================================%
Their divergences are:

\paragraph{Radial functions}
%====================================================================%
\begin{align*}
 R_\ell \sups{outgoing}(kr)  &= h^{(1)}_\ell(kr) \\
 R_\ell \sups{incoming}(kr)  &= h^{(2)}_\ell(kr) \\
 R_\ell \sups{regular}(kr)   &= j_\ell(kr).
\end{align*}
%====================================================================%
I also define the shorthand symbols
%====================================================================%
$$ \overline{R}_\ell(kr)
   \equiv 
   \frac{1}{kr}\left|R_\ell(x) + \frac{d}{dx}R_\ell(x)\right|_{x=kr}
   \qquad
   \Rslash_\ell(kr) = -\frac{\sqrt{l(l+1)}}{kr} R_\ell(kr).
$$
%====================================================================%

\paragraph{Vector spherical wave functions}
%====================================================================%
\begin{align*} 
 \vb M\lm(k; \vb r) 
&\equiv R_\ell(kr) \vb X\lm(\Omega) 
\\
%--------------------------------------------------------------------%
 \vb N\lm(k; \vb r) &\equiv 
i\overline{R}_\ell(kr) \vb Z\lm(\Omega) + \Rslash_\ell(kr) Y\lm(\Omega) \vbhat{r} 
\end{align*} 
%====================================================================%

%%%%%%%%%%%%%%%%%%%%%%%%%%%%%%%%%%%%%%%%%%%%%%%%%%%%%%%%%%%%%%%%%%%%%%
%%%%%%%%%%%%%%%%%%%%%%%%%%%%%%%%%%%%%%%%%%%%%%%%%%%%%%%%%%%%%%%%%%%%%%
%%%%%%%%%%%%%%%%%%%%%%%%%%%%%%%%%%%%%%%%%%%%%%%%%%%%%%%%%%%%%%%%%%%%%%
\section{Explicit expression for small $\ell$}

\paragraph{The first few radial functions}

%====================================================================%
$$\begin{array}{lclclcl}
% R\sups{regular}_0(kr) 
%&=&
% \displaystyle{ \frac{\sin kr}{kr} }
%&\qquad&
% \displaystyle{ \overline{R}\sups{regular}_0(kr)  }
%&=&
% \displaystyle{ \frac{i\cos kr}{kr} }
%\\[10pt]
%%--------------------------------------------------------------------%
% R\sups{outgoing}_0(kr) 
%&=&
% \displaystyle{ \frac{-i e^{ikr}}{kr} }
%&\qquad&
% \displaystyle{ \overline{R}\sups{outgoing}_0(kr) }
%&=&
% \displaystyle{ \frac{ie^{ikr}}{kr} }
%\\[10pt]
%%--------------------------------------------------------------------%
% R\sups{incoming}_0(kr) 
%&=&
% \displaystyle{ \frac{e^{-ikr}}{kr} }
%&\qquad&
% \displaystyle{ \overline{R}\sups{outgoing}_0(kr) }
%&=&
% \displaystyle{ \frac{e^{-ikr}}{kr} } 
%\\[10pt]
%--------------------------------------------------------------------%
 R\sups{regular}_1(kr) 
&=&
 \displaystyle{ -\frac{i(ikr)\cos(kr) + \sin (kr)}{(ikr)^2} }
&\qquad&
 \displaystyle{ \overline{R}\sups{regular}_1(kr)  }
&=&
 \displaystyle{ \frac{(ikr)\cos(kr) - \Big[-1-(ikr)^2\Big] \sin(kr)}
                     {k^3 r^3}
              }
\\[10pt]
%--------------------------------------------------------------------%
 R\sups{outgoing}_1(kr) 
&=&
 \displaystyle{ \Big[-1 + ikr - (ikr)^2\Big] \frac{e^{ikr}}{k^3 r^3} }
&\qquad&
 \displaystyle{ \overline{R}\sups{outgoing}_1(kr) }
&=&
 \displaystyle{ \Big[-ikr + (ikr)^2\Big]\frac{e^{ikr}}{k^3r^3} }
\end{array}$$
%====================================================================%

\paragraph{The first few outgoing functions}

In what follows, the $Q_n$ are dimensionless polynomial factors:
%====================================================================%
\begin{subequations}
\begin{align}
 Q_1(x) &= 1-x \\
 Q_{2a}(x) &= 1-x+x^2 \\
 Q_{2b}(x) &= 3-3x+x^2 \\
 Q_3(x) &= 6-6x+3x^2-x^3
\end{align}
\end{subequations}
%====================================================================%
\begin{align*}
 \vb M\sups{outgoing}_{1,\pm 1}(\vb r)
  &=\sqrt\frac{3}{16\pi}\pf{e^{ikr}}{k^2 r^2} e^{\pm i\phi}
    \left(\begin{array}{c}
    0          \\
    -iQ_1(ikr) \\
    \pm Q_1(ikr) \cos\theta 
    \end{array}\right)
\\[5pt]
%--------------------------------------------------------------------%
 \vb M\sups{outgoing}_{1,0}(\vb r)
  &=\sqrt\frac{3}{8\pi}\pf{e^{ikr}}{k^2 r^2}
    \left(\begin{array}{c}
    0       \\
    0       \\
    Q_1(ikr) \sin\theta 
    \end{array}\right)
\\
%--------------------------------------------------------------------%
 \vb N\sups{outgoing}_{1,\pm 1}(\vb r)
  &=\sqrt\frac{3}{16\pi}\pf{e^{ikr}}{k^3 r^3} e^{\pm i\phi}
    \left(\begin{array}{c}
    \mp 2iQ_1(ikr)\sin \theta 			\\
    \pm iQ_{2a}(ikr) \cos\theta			\\
    -Q_{2a}(ikr)
    \end{array}\right)
\\
%--------------------------------------------------------------------%
 \vb N\sups{outgoing}_{1,0}(\vb r)
  &=\sqrt\frac{3}{8\pi}\pf{e^{ikr}}{k^3 r^3}
    \left(\begin{array}{c}
    2iQ_1(ikr)\cos \theta		\\
    +iQ_{2a}(ikr)\sin\theta	\\
    0
    \end{array}\right)
\\[5pt]
%--------------------------------------------------------------------%
 \vb M\sups{outgoing}_{2,\pm 2}(\vb r)
  &=\sqrt\frac{5}{16\pi}\pf{e^{ikr}}{k^3 r^3} e^{\pm 2i\phi}
    \left(\begin{array}{c}
    0 				  \\
   \pm  iQ_{2b}(ikr)\sin\theta	  \\
    -Q_{2b}(ikr)\cos\theta\sin\theta
    \end{array}\right)
\\
%--------------------------------------------------------------------%
 \vb M\sups{outgoing}_{2,\pm 1}(\vb r)
  &=\sqrt\frac{5}{16\pi}\pf{e^{ikr}}{k^3 r^3} e^{\pm i\phi}
    \left(\begin{array}{c}
    0                            \\
    -iQ_{2b}(ikr)\cos\theta  \\
    \pm Q_{2b}(ikr)\cos 2\theta
    \end{array}\right)
\\
%--------------------------------------------------------------------%
 \vb M\sups{outgoing}_{2,0}(\vb r)
  &=\sqrt\frac{15}{8\pi}\pf{e^{ikr}}{k^3 r^3}
    \left(\begin{array}{c}
    0 \\ 
    0 \\ 
    -Q_{2b}(ikr)\cos \theta \sin\theta
    \end{array}\right)
\\[5pt]
%--------------------------------------------------------------------%
 \vb N\sups{outgoing}_{2,\pm 2}(\vb r)
  &=\sqrt\frac{5}{16\pi}\pf{e^{ikr}}{k^4 r^4} e^{\pm 2i\phi}
    \left(\begin{array}{c}
   3iQ_{2b}(ikr) \sin^2 \theta                          \\
   -iQ_3(ikr) \cos\theta \sin\theta   \\
   \pm Q_3(ikr) \sin\theta 
    \end{array}\right)
\\
%--------------------------------------------------------------------%
 \vb N\sups{outgoing}_{2,\pm 1}(\vb r)
  &=\sqrt\frac{5}{16\pi}\pf{e^{ikr}}{k^4 r^4} e^{\pm i\phi}
    \left(\begin{array}{c}
   \mp 3iQ_{2b}(ikr) \sin 2\theta               \\
   \pm iQ_3(ikr) \cos 2\theta  \\
   -Q_3(ikr) \cos\theta
    \end{array}\right)
\\
%--------------------------------------------------------------------%
 \vb N\sups{outgoing}_{2,0}(\vb r)
  &=\sqrt\frac{15}{8\pi}\pf{e^{ikr}}{k^4 r^4}
    \left(\begin{array}{c}
   iQ_{2b}(ikr)(3\cos^2 \theta -1)                    \\
   iQ_3(ikr) \cos \theta \sin\theta \\
   0
    \end{array}\right).
\end{align*}
\end{document}
