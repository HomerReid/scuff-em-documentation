\documentclass[letterpaper]{article}

%********************************************************
%* latex shortcuts used for libscuff documentation files
%* 
%* homer reid -- 1994--2011
%**************************************************
\usepackage{graphicx}
\usepackage{color}
\usepackage{dsfont}
\usepackage{bbm}
\usepackage{amsmath}
\usepackage{amssymb}
\usepackage{float}
\usepackage{psfrag}
\usepackage{mathdots}
%\usepackage{algorithm}
%\usepackage{algorithmic}
%\usepackage{listings}
\usepackage{array}
\usepackage{url}
\usepackage{arydshln}
\usepackage{fancybox}
\usepackage{fancyvrb}
\usepackage{verbatim}

%--------------------------------------------------
%- boldface greek letters 
%--------------------------------------------------
\newcommand\vbphi{\mathbf{\phi}}
\newcommand\vbPhi{\mathbf{\Phi}}
\newcommand{\vbDelta}{\boldsymbol{\Delta}}
\newcommand{\vbrho}{\boldsymbol{\rho}}
\newcommand{\vbchi}{\boldsymbol{\chi}}

%--------------------------------------------------
%- colors -----------------------------------------
%--------------------------------------------------
\newcommand{\red}[1]{\textcolor{red}{#1}}
\newcommand{\blue}[1]{\textcolor{blue}{#1}}
\newcommand{\green}[1]{\textcolor{green}{#1}}
\newcommand{\atan}{\text{atan}}

%--------------------------------------------------
% general commands
%--------------------------------------------------
\newcommand\Tr{\hbox{Tr }}
\newcommand\sups[1]{^{\hbox{\scriptsize{#1}}}}
\newcommand\supt[1]{^{\hbox{\tiny{#1}}}}
\newcommand\subs[1]{_{\hbox{\scriptsize{#1}}}}
\newcommand\subt[1]{_{\hbox{\tiny{#1}}}}
\newcommand{\nn}{\nonumber \\}
\newcommand{\vb}[1]{\mathbf{#1}}
\newcommand{\numeq}[2]{\begin{equation} #2 \label{#1} \end{equation}}
\newcommand{\pard}[2]{\frac{\partial #1}{\partial #2}}
\newcommand{\pardn}[3]{\frac{\partial^{#1} #2}{\partial #3^{#1}}}
\newcommand{\pf}[2]{\left(\frac{#1}{#2}\right)}
\newcommand{\pp}{{\prime\prime}}
\newcommand{\vbhat}[1]{\vb{\hat #1}}
\newcommand{\mc}[1]{\mathcal{#1}}
\newcommand{\bmc}[1]{\boldsymbol{\mathcal{#1}}}
\newcommand{\mb}[1]{\mathbb{#1}}
\newcommand{\ds}[1]{\displaystyle{#1}}
\newcommand{\primedsum}{\sideset{}{'}{\sum}}
\newcommand{\pan}{\mathcal{P}}

%--------------------------------------------------
%- special libscuff stuff--------------------------
%--------------------------------------------------
\newcommand\ls{{\sc libscuff}}
\newcommand\lss{{\sc libscuff\,}}
\newcommand{\BG}{\boldsymbol{\Gamma}}
\newcommand{\MInt}{\vb{\hat M}}
\newcommand{\NInt}{\vb{\hat N}}
\newcommand{\MExt}{\vb{\check M}}
\newcommand{\NExt}{\vb{\check N}}
\newcommand{\xInt}{\vb{\hat x}}
\newcommand{\xExt}{\vb{\check x}}

%--------------------------------------------------
%-- superscripts for dyadic green's functions 
%--------------------------------------------------
\newcommand{\EEe}{^{\hbox{\tiny{EE}\scriptsize{,$e$}}}}
\newcommand{\MEe}{^{\hbox{\tiny{ME}\scriptsize{,$e$}}}}
\newcommand{\EMe}{^{\hbox{\tiny{EM}\scriptsize{,$e$}}}}
\newcommand{\MMe}{^{\hbox{\tiny{MM}\scriptsize{,$e$}}}}
\newcommand{\EEr}{^{\hbox{\tiny{EE}\scriptsize{,$r$}}}}
\newcommand{\MEr}{^{\hbox{\tiny{ME}\scriptsize{,$r$}}}}
\newcommand{\EMr}{^{\hbox{\tiny{EM}\scriptsize{,$r$}}}}
\newcommand{\MMr}{^{\hbox{\tiny{MM}\scriptsize{,$r$}}}}
\newcommand{\incr}{^{\text{\scriptsize inc},r}}
\newcommand{\incro}{^{\text{\scriptsize inc},r_1}}
\newcommand{\incrt}{^{\text{\scriptsize inc},r_2}}

%%%%%%%%%%%%%%%%%%%%%%%%%%%%%%%%%%%%%%%%%%%%%%%%%%%%%%%%%%%%%%%%%%%%%%
%%%%%%%%%%%%%%%%%%%%%%%%%%%%%%%%%%%%%%%%%%%%%%%%%%%%%%%%%%%%%%%%%%%%%%
%%%%%%%%%%%%%%%%%%%%%%%%%%%%%%%%%%%%%%%%%%%%%%%%%%%%%%%%%%%%%%%%%%%%%%
\newcommand{\MMExt}{\check{\mathcal{M}}}
\newcommand{\MMInt}{\hat{\mathcal{M}}}
\newcommand{\NNExt}{\check{\mathcal{N}}}
\newcommand{\NNInt}{\hat{\mathcal{N}}}
\newcommand{\BMMExt}{\boldsymbol{\check{\mathcal{M}}}}
\newcommand{\BMMInt}{\boldsymbol{\hat{\mathcal{M}}}}
\newcommand{\BNNExt}{\boldsymbol{\check{\mathcal{N}}}}
\newcommand{\BNNInt}{\boldsymbol{\hat{\mathcal{N}}}}
\newpage

%--------------------------------------------------
%- inner products and operator matrix elements  ---
%--------------------------------------------------
\newcommand{\expval}[1]{ \big\langle #1 \big\rangle}
\newcommand{\Expval}[1]{ \Big\langle #1 \Big\rangle}
\newcommand{\inp}[2]{ \big\langle #1 \big| #2 \big\rangle}
\newcommand{\Inp}[2]{ \Big\langle #1 \Big| #2 \Big\rangle}
\newcommand{\INP}[2]{ \bigg\langle #1 \bigg| #2 \Big\rangle}
\newcommand{\vmv}[3]{ \big\langle #1 \big| #2 \big| #3 \big\rangle}
\newcommand{\VMV}[3]{ \Big\langle #1 \Big| #2 \Big| #3 \Big\rangle}

%------------------------------------------------------------
%- shaded box for source code inclusions 
%------------------------------------------------------------
%\definecolor{lightgrey}{rgb}{0.85,0.85,0.85}
%\newcommand{\SourceBox}[1]
% { \begin{mdframed}[backgroundcolor=lightgrey, linewidth=2pt,
%                    tikzsetting={draw=black, line width=2pt,
%                                 dashed, dash pattern= on 5pt off 5pt}] %
%  \begin{verbatim} #1 \end{verbatim} \end{mdframed} \medskip
%}
%\IfFileExists{mdframed.sty}
% { \usepackage{mdframed}
%   \surroundwithmdframed[backgroundcolor=lightgrey, linewidth=2pt,
%                         framemethod=tikz,
%                         skipabove=10pt,
%                         skipbelow=10pt,
%                         tikzsetting={draw=black, line width=2pt,
%                                      dashed, dash pattern= on 5pt off 5pt}
%                        ]{verbatim}
% }
% {
% }
%--------------------------------------------------
%- shaded verbatim for code inclusions
%--------------------------------------------------
\definecolor{lightgrey}{rgb}{0.85,0.85,0.85}
\newenvironment{verbcode}{\VerbatimEnvironment% 
  \noindent
  %{\columnwidth-\leftmargin-\rightmargin-2\fboxsep-2\fboxrule-4pt} 
  \begin{Sbox} 
  \begin{minipage}{\linewidth}
  \begin{Verbatim}
}{% 
  \end{Verbatim}  
   \end{minipage}
  \end{Sbox} 
  \fcolorbox{black}{lightgrey}{\textcolor{black}{\TheSbox}}
} 


\graphicspath{{figures/}}

%------------------------------------------------------------
%------------------------------------------------------------
%- Special commands for this document -----------------------
%------------------------------------------------------------
%------------------------------------------------------------

%------------------------------------------------------------
%------------------------------------------------------------
%- Document header  -----------------------------------------
%------------------------------------------------------------
%------------------------------------------------------------
\title {Efficient Evaluation of Matrix Elements \\
        between Distant Basis Functions in \ls}
\author {Homer Reid}
\date {April 16, 2013}

%------------------------------------------------------------
%------------------------------------------------------------
%- Start of actual document
%------------------------------------------------------------
%------------------------------------------------------------

\begin{document}
\pagestyle{myheadings}
\markright{Homer Reid: \lss Implementation and Technical Details}
\maketitle

\tableofcontents

%%%%%%%%%%%%%%%%%%%%%%%%%%%%%%%%%%%%%%%%%%%%%%%%%%%%%%%%%%%%%%%%%%%%%%
%%%%%%%%%%%%%%%%%%%%%%%%%%%%%%%%%%%%%%%%%%%%%%%%%%%%%%%%%%%%%%%%%%%%%%
%%%%%%%%%%%%%%%%%%%%%%%%%%%%%%%%%%%%%%%%%%%%%%%%%%%%%%%%%%%%%%%%%%%%%%
\newpage
\section{Overview}

%%%%%%%%%%%%%%%%%%%%%%%%%%%%%%%%%%%%%%%%%%%%%%%%%%%%%%%%%%%%%%%%%%%%%%
%%%%%%%%%%%%%%%%%%%%%%%%%%%%%%%%%%%%%%%%%%%%%%%%%%%%%%%%%%%%%%%%%%%%%%
%%%%%%%%%%%%%%%%%%%%%%%%%%%%%%%%%%%%%%%%%%%%%%%%%%%%%%%%%%%%%%%%%%%%%%
\newpage
\section{Cartesian Multipole Technique}

$$ G_{\mu\nu}(\vb r) = 
     G_{\mu\nu}(\vb r_0) 
   + (\vb r - \vb r_0)_\rho G_{\mu\nu\rho}(\vb r_0) 
   + \frac{1}{2}(\vb r - \vb r_0)_\rho (\vb r - \vb r_0)_\sigma 
     G_{\mu\nu\rho\sigma}(\vb r_0) + \cdots
$$
\begin{align*}
&\hspace{-0.1in} 
\int \int f_{m\mu}(\vb x) G_{\mu\nu}(\vb x-\vb x^\prime) f_{n\nu}(\vb x^\prime)
  d\vb x d d\vb x^\prime
\\
&=
 G_{\mu\nu}^0
 \underbrace{\left[ \int f_{m\mu}(\vb x) \, d\vb x \right]}
           _{\mc M_{m\mu}}
 \underbrace{\left[ \int f_{n\nu}(\vb x^\prime) \, d\vb x^\prime \right]}
           _{\mc M_{n\nu}}
\\
&+
 G_{\mu\nu\rho}^0
 \left\{ \underbrace{
            \left[ \int (\vb x - \vb x_0)_\rho f_{m\mu}(\vb x) \, d\vb x \right]
                    }_{\mc M_{m\mu\rho}}
         \underbrace{
         \left[ \int f_{n\nu}(\vb x^\prime) \, d\vb x^\prime \right]
                    }_{\mc M_{n\nu}}
        -
         \underbrace{
         \left[ \int f_{m\mu}(\vb x) \, d\vb x \right]
                    }_{\mc M_{m\mu}}
         \underbrace{
         \left[ \int (\vb x^\prime-\vb x^\prime_0)_\rho 
                     f_{n\nu}(\vb x^\prime) \, d\vb x^\prime \right]
                    }_{\mc M_{n\nu\rho}}
 \right\}     
\\
&+\cdots
\end{align*}

\subsection*{Multipole moments of RWG basis functions}

\begin{align*}
 \mc M_{m\mu}     
&= \frac{l_m}{3}\Big(\vb Q^-_m - \vb Q^+_m\Big)_\mu
\\
 \mc M_{m\mu\rho} 
&= \frac{l_m}{12}\Big[ \vb A^-_\mu \vb A^-_\rho 
                      -\vb A^+_\mu \vb A^+_\rho \Big]
  -\frac{1}{8}\Big[ \vb B_\mu \mc M_{m\rho} + \mc M_{m\mu} \vb B_\rho \Big]
\end{align*}

\subsection*{Cartesian Components of Dyadic Green's functions}

\begin{align*}
  G_{\mu\nu}(\vb r) 
&= \left[P_1(ikr)\delta_{\mu\nu} + P_2(ikr)\frac{r_\mu r_\nu}{r^2}\right]\Phi(r)
\\
  C_{\mu\nu}(\vb r) 
&= ik P_3(ikr) \Phi(r) \varepsilon_{\mu\nu\rho} r_\rho
\end{align*}

\begin{align*}
 \Phi(r) &=\frac{e^{ikr}}{4\pi (ik)^2 r^3}
\\
P_1(x)&=1-x-x^2 
\\
P_2(x)&=-3 + 3x - x^2
\\
P_3(x)&=-1+x
\\
\end{align*}

\subsection*{First derivatives}

\begin{align*}
  G_{\mu\nu\rho}(\vb r)
&=\frac{d}{dr_\rho} G_{\mu\nu}(\vb r)
&= \left[P_1(ikr)\delta_{\mu\nu} + P_2(ikr)\frac{r_\mu r_\nu}{r^2}\right]\Phi(r)
\\
  C_{\mu\nu}(\vb r) 
&= ik P_3(ikr) \Phi(r) \varepsilon_{\mu\nu\rho} r_\rho
\end{align*}

\begin{align*}
 \Phi(r) &=\frac{e^{ikr}}{4\pi (ik)^2 r^3}
\\
P_1(x)&=1-x-x^2 
\\
P_2(x)&=-3 + 3x - x^2
\\
P_3(x)&=-1+x
\\
\end{align*}

\end{document}
