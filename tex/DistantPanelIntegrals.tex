\documentclass[letterpaper]{article}

\input{scufftex}

\graphicspath{{figures/}}

%------------------------------------------------------------
%------------------------------------------------------------
%- Special commands for this document -----------------------
%------------------------------------------------------------
%------------------------------------------------------------

%------------------------------------------------------------
%------------------------------------------------------------
%- Document header  -----------------------------------------
%------------------------------------------------------------
%------------------------------------------------------------
\title {Efficient Evaluation of Matrix Elements \\
        between Distant Basis Functions in \ls}
\author {Homer Reid}
\date {April 16, 2013}

%------------------------------------------------------------
%------------------------------------------------------------
%- Start of actual document
%------------------------------------------------------------
%------------------------------------------------------------

\begin{document}
\pagestyle{myheadings}
\markright{Homer Reid: \lss Implementation and Technical Details}
\maketitle

\tableofcontents

%%%%%%%%%%%%%%%%%%%%%%%%%%%%%%%%%%%%%%%%%%%%%%%%%%%%%%%%%%%%%%%%%%%%%%
%%%%%%%%%%%%%%%%%%%%%%%%%%%%%%%%%%%%%%%%%%%%%%%%%%%%%%%%%%%%%%%%%%%%%%
%%%%%%%%%%%%%%%%%%%%%%%%%%%%%%%%%%%%%%%%%%%%%%%%%%%%%%%%%%%%%%%%%%%%%%
\newpage
\section{Overview}

%%%%%%%%%%%%%%%%%%%%%%%%%%%%%%%%%%%%%%%%%%%%%%%%%%%%%%%%%%%%%%%%%%%%%%
%%%%%%%%%%%%%%%%%%%%%%%%%%%%%%%%%%%%%%%%%%%%%%%%%%%%%%%%%%%%%%%%%%%%%%
%%%%%%%%%%%%%%%%%%%%%%%%%%%%%%%%%%%%%%%%%%%%%%%%%%%%%%%%%%%%%%%%%%%%%%
\newpage
\section{Cartesian Multipole Technique}

$$ G_{\mu\nu}(\vb r) = 
     G_{\mu\nu}(\vb r_0) 
   + (\vb r - \vb r_0)_\rho G_{\mu\nu\rho}(\vb r_0) 
   + \frac{1}{2}(\vb r - \vb r_0)_\rho (\vb r - \vb r_0)_\sigma 
     G_{\mu\nu\rho\sigma}(\vb r_0) + \cdots
$$
\begin{align*}
&\hspace{-0.1in} 
\int \int f_{m\mu}(\vb x) G_{\mu\nu}(\vb x-\vb x^\prime) f_{n\nu}(\vb x^\prime)
  d\vb x d d\vb x^\prime
\\
&=
 G_{\mu\nu}^0
 \underbrace{\left[ \int f_{m\mu}(\vb x) \, d\vb x \right]}
           _{\mc M_{m\mu}}
 \underbrace{\left[ \int f_{n\nu}(\vb x^\prime) \, d\vb x^\prime \right]}
           _{\mc M_{n\nu}}
\\
&+
 G_{\mu\nu\rho}^0
 \left\{ \underbrace{
            \left[ \int (\vb x - \vb x_0)_\rho f_{m\mu}(\vb x) \, d\vb x \right]
                    }_{\mc M_{m\mu\rho}}
         \underbrace{
         \left[ \int f_{n\nu}(\vb x^\prime) \, d\vb x^\prime \right]
                    }_{\mc M_{n\nu}}
        -
         \underbrace{
         \left[ \int f_{m\mu}(\vb x) \, d\vb x \right]
                    }_{\mc M_{m\mu}}
         \underbrace{
         \left[ \int (\vb x^\prime-\vb x^\prime_0)_\rho 
                     f_{n\nu}(\vb x^\prime) \, d\vb x^\prime \right]
                    }_{\mc M_{n\nu\rho}}
 \right\}     
\\
&+\cdots
\end{align*}

\subsection*{Multipole moments of RWG basis functions}

\begin{align*}
 \mc M_{m\mu}     
&= \frac{l_m}{3}\Big(\vb Q^-_m - \vb Q^+_m\Big)_\mu
\\
 \mc M_{m\mu\rho} 
&= \frac{l_m}{12}\Big[ \vb A^-_\mu \vb A^-_\rho 
                      -\vb A^+_\mu \vb A^+_\rho \Big]
  -\frac{1}{8}\Big[ \vb B_\mu \mc M_{m\rho} + \mc M_{m\mu} \vb B_\rho \Big]
\end{align*}

\subsection*{Cartesian Components of Dyadic Green's functions}

\begin{align*}
  G_{\mu\nu}(\vb r) 
&= \left[P_1(ikr)\delta_{\mu\nu} + P_2(ikr)\frac{r_\mu r_\nu}{r^2}\right]\Phi(r)
\\
  C_{\mu\nu}(\vb r) 
&= ik P_3(ikr) \Phi(r) \varepsilon_{\mu\nu\rho} r_\rho
\end{align*}

\begin{align*}
 \Phi(r) &=\frac{e^{ikr}}{4\pi (ik)^2 r^3}
\\
P_1(x)&=1-x-x^2 
\\
P_2(x)&=-3 + 3x - x^2
\\
P_3(x)&=-1+x
\\
\end{align*}

\subsection*{First derivatives}

\begin{align*}
  G_{\mu\nu\rho}(\vb r)
&=\frac{d}{dr_\rho} G_{\mu\nu}(\vb r)
&= \left[P_1(ikr)\delta_{\mu\nu} + P_2(ikr)\frac{r_\mu r_\nu}{r^2}\right]\Phi(r)
\\
  C_{\mu\nu}(\vb r) 
&= ik P_3(ikr) \Phi(r) \varepsilon_{\mu\nu\rho} r_\rho
\end{align*}

\begin{align*}
 \Phi(r) &=\frac{e^{ikr}}{4\pi (ik)^2 r^3}
\\
P_1(x)&=1-x-x^2 
\\
P_2(x)&=-3 + 3x - x^2
\\
P_3(x)&=-1+x
\\
\end{align*}

\end{document}
