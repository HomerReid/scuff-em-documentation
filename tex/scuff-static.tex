\documentclass[letterpaper]{article}

%********************************************************
%* latex shortcuts used for libscuff documentation files
%* 
%* homer reid -- 1994--2011
%**************************************************
\usepackage{graphicx}
\usepackage{color}
\usepackage{dsfont}
\usepackage{bbm}
\usepackage{amsmath}
\usepackage{amssymb}
\usepackage{float}
\usepackage{psfrag}
\usepackage{mathdots}
%\usepackage{algorithm}
%\usepackage{algorithmic}
%\usepackage{listings}
\usepackage{array}
\usepackage{url}
\usepackage{arydshln}
\usepackage{fancybox}
\usepackage{fancyvrb}
\usepackage{verbatim}

%--------------------------------------------------
%- boldface greek letters 
%--------------------------------------------------
\newcommand\vbphi{\mathbf{\phi}}
\newcommand\vbPhi{\mathbf{\Phi}}
\newcommand{\vbDelta}{\boldsymbol{\Delta}}
\newcommand{\vbrho}{\boldsymbol{\rho}}
\newcommand{\vbchi}{\boldsymbol{\chi}}

%--------------------------------------------------
%- colors -----------------------------------------
%--------------------------------------------------
\newcommand{\red}[1]{\textcolor{red}{#1}}
\newcommand{\blue}[1]{\textcolor{blue}{#1}}
\newcommand{\green}[1]{\textcolor{green}{#1}}
\newcommand{\atan}{\text{atan}}

%--------------------------------------------------
% general commands
%--------------------------------------------------
\newcommand\Tr{\hbox{Tr }}
\newcommand\sups[1]{^{\hbox{\scriptsize{#1}}}}
\newcommand\supt[1]{^{\hbox{\tiny{#1}}}}
\newcommand\subs[1]{_{\hbox{\scriptsize{#1}}}}
\newcommand\subt[1]{_{\hbox{\tiny{#1}}}}
\newcommand{\nn}{\nonumber \\}
\newcommand{\vb}[1]{\mathbf{#1}}
\newcommand{\numeq}[2]{\begin{equation} #2 \label{#1} \end{equation}}
\newcommand{\pard}[2]{\frac{\partial #1}{\partial #2}}
\newcommand{\pardn}[3]{\frac{\partial^{#1} #2}{\partial #3^{#1}}}
\newcommand{\pf}[2]{\left(\frac{#1}{#2}\right)}
\newcommand{\pp}{{\prime\prime}}
\newcommand{\vbhat}[1]{\vb{\hat #1}}
\newcommand{\mc}[1]{\mathcal{#1}}
\newcommand{\bmc}[1]{\boldsymbol{\mathcal{#1}}}
\newcommand{\mb}[1]{\mathbb{#1}}
\newcommand{\ds}[1]{\displaystyle{#1}}
\newcommand{\primedsum}{\sideset{}{'}{\sum}}
\newcommand{\pan}{\mathcal{P}}

%--------------------------------------------------
%- special libscuff stuff--------------------------
%--------------------------------------------------
\newcommand\ls{{\sc libscuff}}
\newcommand\lss{{\sc libscuff\,}}
\newcommand{\BG}{\boldsymbol{\Gamma}}
\newcommand{\MInt}{\vb{\hat M}}
\newcommand{\NInt}{\vb{\hat N}}
\newcommand{\MExt}{\vb{\check M}}
\newcommand{\NExt}{\vb{\check N}}
\newcommand{\xInt}{\vb{\hat x}}
\newcommand{\xExt}{\vb{\check x}}

%--------------------------------------------------
%-- superscripts for dyadic green's functions 
%--------------------------------------------------
\newcommand{\EEe}{^{\hbox{\tiny{EE}\scriptsize{,$e$}}}}
\newcommand{\MEe}{^{\hbox{\tiny{ME}\scriptsize{,$e$}}}}
\newcommand{\EMe}{^{\hbox{\tiny{EM}\scriptsize{,$e$}}}}
\newcommand{\MMe}{^{\hbox{\tiny{MM}\scriptsize{,$e$}}}}
\newcommand{\EEr}{^{\hbox{\tiny{EE}\scriptsize{,$r$}}}}
\newcommand{\MEr}{^{\hbox{\tiny{ME}\scriptsize{,$r$}}}}
\newcommand{\EMr}{^{\hbox{\tiny{EM}\scriptsize{,$r$}}}}
\newcommand{\MMr}{^{\hbox{\tiny{MM}\scriptsize{,$r$}}}}
\newcommand{\incr}{^{\text{\scriptsize inc},r}}
\newcommand{\incro}{^{\text{\scriptsize inc},r_1}}
\newcommand{\incrt}{^{\text{\scriptsize inc},r_2}}

%%%%%%%%%%%%%%%%%%%%%%%%%%%%%%%%%%%%%%%%%%%%%%%%%%%%%%%%%%%%%%%%%%%%%%
%%%%%%%%%%%%%%%%%%%%%%%%%%%%%%%%%%%%%%%%%%%%%%%%%%%%%%%%%%%%%%%%%%%%%%
%%%%%%%%%%%%%%%%%%%%%%%%%%%%%%%%%%%%%%%%%%%%%%%%%%%%%%%%%%%%%%%%%%%%%%
\newcommand{\MMExt}{\check{\mathcal{M}}}
\newcommand{\MMInt}{\hat{\mathcal{M}}}
\newcommand{\NNExt}{\check{\mathcal{N}}}
\newcommand{\NNInt}{\hat{\mathcal{N}}}
\newcommand{\BMMExt}{\boldsymbol{\check{\mathcal{M}}}}
\newcommand{\BMMInt}{\boldsymbol{\hat{\mathcal{M}}}}
\newcommand{\BNNExt}{\boldsymbol{\check{\mathcal{N}}}}
\newcommand{\BNNInt}{\boldsymbol{\hat{\mathcal{N}}}}
\newpage

%--------------------------------------------------
%- inner products and operator matrix elements  ---
%--------------------------------------------------
\newcommand{\expval}[1]{ \big\langle #1 \big\rangle}
\newcommand{\Expval}[1]{ \Big\langle #1 \Big\rangle}
\newcommand{\inp}[2]{ \big\langle #1 \big| #2 \big\rangle}
\newcommand{\Inp}[2]{ \Big\langle #1 \Big| #2 \Big\rangle}
\newcommand{\INP}[2]{ \bigg\langle #1 \bigg| #2 \Big\rangle}
\newcommand{\vmv}[3]{ \big\langle #1 \big| #2 \big| #3 \big\rangle}
\newcommand{\VMV}[3]{ \Big\langle #1 \Big| #2 \Big| #3 \Big\rangle}

%------------------------------------------------------------
%- shaded box for source code inclusions 
%------------------------------------------------------------
%\definecolor{lightgrey}{rgb}{0.85,0.85,0.85}
%\newcommand{\SourceBox}[1]
% { \begin{mdframed}[backgroundcolor=lightgrey, linewidth=2pt,
%                    tikzsetting={draw=black, line width=2pt,
%                                 dashed, dash pattern= on 5pt off 5pt}] %
%  \begin{verbatim} #1 \end{verbatim} \end{mdframed} \medskip
%}
%\IfFileExists{mdframed.sty}
% { \usepackage{mdframed}
%   \surroundwithmdframed[backgroundcolor=lightgrey, linewidth=2pt,
%                         framemethod=tikz,
%                         skipabove=10pt,
%                         skipbelow=10pt,
%                         tikzsetting={draw=black, line width=2pt,
%                                      dashed, dash pattern= on 5pt off 5pt}
%                        ]{verbatim}
% }
% {
% }
%--------------------------------------------------
%- shaded verbatim for code inclusions
%--------------------------------------------------
\definecolor{lightgrey}{rgb}{0.85,0.85,0.85}
\newenvironment{verbcode}{\VerbatimEnvironment% 
  \noindent
  %{\columnwidth-\leftmargin-\rightmargin-2\fboxsep-2\fboxrule-4pt} 
  \begin{Sbox} 
  \begin{minipage}{\linewidth}
  \begin{Verbatim}
}{% 
  \end{Verbatim}  
   \end{minipage}
  \end{Sbox} 
  \fcolorbox{black}{lightgrey}{\textcolor{black}{\TheSbox}}
} 


\graphicspath{{figures/}}

%------------------------------------------------------------
%------------------------------------------------------------
%- Special commands for this document -----------------------
%------------------------------------------------------------
%------------------------------------------------------------
\newcommand{\YY}{\mc Y}
\newcommand{\BE}{\begin{equation}}
\newcommand{\EE}{\end{equation}}

%------------------------------------------------------------
%------------------------------------------------------------
%- Document header  -----------------------------------------
%------------------------------------------------------------
%------------------------------------------------------------
\title {{\sc scuff-static}: Pure Electrostatics in {\sc scuff-em}}
\author {Homer Reid}
\date {May 17, 2013}


%------------------------------------------------------------
%------------------------------------------------------------
%- Start of actual document
%------------------------------------------------------------
%------------------------------------------------------------

\begin{document}
\pagestyle{myheadings}
\markright{Homer Reid: Electrostatics in \ls}
\maketitle

\tableofcontents

%%%%%%%%%%%%%%%%%%%%%%%%%%%%%%%%%%%%%%%%%%%%%%%%%%%%%%%%%%%%%%%%%%%%%%
%%%%%%%%%%%%%%%%%%%%%%%%%%%%%%%%%%%%%%%%%%%%%%%%%%%%%%%%%%%%%%%%%%%%%%
%%%%%%%%%%%%%%%%%%%%%%%%%%%%%%%%%%%%%%%%%%%%%%%%%%%%%%%%%%%%%%%%%%%%%%
\newpage
\section{Overview}

The core {\sc scuff-em} library is designed to solve
electromagnetism problems at nonzero frequencies.
In principle, we can use it to solve electrostatics
problems simply by working at low nonzero frequencies
and extrapolating to the $\omega\to 0$ limit. This
approach works adequately in many cases of interest
(as, for example, in the ``electrostatics of a spherical
dielectric shell'' example on the {\sc scuff-em}
website.\footnote{\url{http://homerreid.com/scuff-EM/scuff-scatter/scuffScatterExamples.shtml#SphericalShell}.})

However, there are at least two drawbacks to such an 
approach:

\begin{itemize}
 \item The nonzero-frequency BEM formulations
       used by \lss are solving for surface
       \textit{currents} (both electric and 
       magnetic currents), whereas for 
       electrostatics problems the appropriate 
       unknowns are electric \textit{charges} 
       with no magnetic unknowns. Because there 
       are multiple current distributions whose 
       divergence yields
       the same charge density, the full-wave
       formulations in \lss become ill-conditioned
       in the extreme DC limit. Even for 
       low-but-not-extremely-low frequencies
       at which the full-wave formulation is 
       not particularly badly behaved, it is 
       still computationally \textit{wasteful} 
       to solve for electric and magnetic currents
       when we only need electric charges.
       Indeed, for a dielectric object represented
       by a surface mesh of $N$ triangular panels,
       the full-wave formulation of \lss
       involves roughly $3N$ unknowns, whereas a
       pure-electrostatics formulation
       requires exactly $N$ unknowns, as discussed below.

 \item In electrostatics we frequently encounter
       boundary conditions consisting of fixed
       potentials on conductor surfaces. This 
       kind of boundary condition is unwieldy 
       to support in core \lss
       (although the {\sc scuff-rf} module 
       does support a version of it in the 
       definition of port voltages).

\end{itemize}
%
To address these difficulties, {\sc scuff-em} includes 
a pure electrostatics module known as {\sc scuff-static}.
This code reuses much of the existing {\sc libscuff}
infrastructure
to implement an electrostatic BEM formulation.
It is designed to be easily incorporated into 
existing {\sc scuff-em} workflows; in particular, it
reads the same \texttt{.scuffgeo} files for describing
geometries.\footnote{One distinction: {\sc scuff-static}
does not presently support {\texttt{LATTICE}} statements,
i.e. extended objects with periodic boundary conditions.}
The following section of this memo describes the 
(standard) electrostatic BEM formulation used by 
{\sc scuff-em}, and subsequent sections discuss
the implementation of the various types of calculation
you can request in a {\sc scuff-static} run.

\subsection*{Implementation of $\lambda$-surfaces}

In addition to the usual electrostatic boundary conditions
for perfect conductors and dielectrics, {\sc scuff-static} 
supports an exotic type of boundary condition that I will 
call ``$\lambda$-conditions,'' which arises in theoretical
treatments of entanglement entropy. At a point $\vb x$ on
the surface of a body 
satisfying $\lambda$-conditions (which I will call a 
``$\lambda$-surface''), the boundary condition on the
electrostatic potential is 
%====================================================================%
\numeq{LambdaCondition0}
{
    \left|\frac{d\phi}{d\vbhat{n}}\right|_{\vb x^+} 
   -\left|\frac{d\phi}{d\vbhat{n}}\right|_{\vb x^-} 
   =\frac{1}{\lambda} \phi(\vb x)
}
%====================================================================%
where $\lambda$ is a material property of the surface.
[Noting that the LHS of (\ref{LambdaCondition0}) is the
jump in normal electric field across the two sides of
a surface in vacuum, it is 
tempting to interpret the RHS as a surface charge
density, in which case we can think of $1/\lambda$ as 
a sort of linear semiconducting susceptibility, i.e.
the surface develops a local charge density proportional to 
the local electrostatic potential with proportionality
constant $1/\lambda$.] $\lambda$-surfaces are 
described in {\sc scuff-em} geometry files as 
infinitesimally thin objects whose dielectric 
constant has zero real part and negative imaginary part;
the absolute value of the imaginary part is taken
as the value of $\lambda.$

%%%%%%%%%%%%%%%%%%%%%%%%%%%%%%%%%%%%%%%%%%%%%%%%%%%%%%%%%%%%%%%%%%%%%%
%%%%%%%%%%%%%%%%%%%%%%%%%%%%%%%%%%%%%%%%%%%%%%%%%%%%%%%%%%%%%%%%%%%%%%
%%%%%%%%%%%%%%%%%%%%%%%%%%%%%%%%%%%%%%%%%%%%%%%%%%%%%%%%%%%%%%%%%%%%%%
\newpage
\section{Theory}

As in the usual \ls, we imagine geometries to consist
of homogeneous regions $\{\mc R_r\}$. Homogeneous region 
$\{\mc R_r\}$ has relative dielectric permittivity 
$\epsilon_r.$ 

The boundary of $\mc R_r$ is denoted $\partial \mc R_r$; 
it may consist of a single closed surface or a union of
surfaces, each of which may be individually open.
Thus we write 
%====================================================================%
\numeq{RrSs} { \partial \mc R_r = \cup \, \mathcal{S}_s }
%====================================================================%
Each surface $\mc S_s$ bounds precisely two regions; 
thus $\mc S_s$ appears on the RHS of equation (\ref{RrSs})
for two different values of $r$. 

On each surface lives a surface charge density 
$\sigma(\vb x)$.

Surface $\mc S_s$ may optionally be PEC, in which case it
must be assigned a fixed potential $V_s$ before the problem
can be solved. Physically, a PEC surface corresponds to
an infinitesimally thin conducting layer at the interface
between two dielectric regions, and the surface charge
on such a surface represents free charges supplied to the
surface by whatever batteries initially charged them up to
their specified  potentials $V_s.$

On the other hand, a non-PEC surface simply desribes the 
interface between two dielectric regions, and the 
surface charge on such a surface represents the divergence
of the bound volume polarization density in the dielectric  
region. Among other things, this means that the total
surface charge integrated over the boundary of a dielectric
region must vanish. There is no such constraint on 
the total surface charge on PEC surfaces.

%====================================================================%
%====================================================================%
%====================================================================%
\subsection*{Potentials and fields from charge densities: 
             continuous forms} 

The electrostatic potential and field at an arbitrary point
$\vb x$ are obtained by summing contributions from
surface charges on all surfaces, plus the contributions of
any external field sources that may be present:
%====================================================================%
\begin{subequations}
\begin{align}
\phi(\vb x)  
&= \phi\sups{ext}(\vb x) 
  +\frac{1}{4\pi\epsilon_0} 
   \sum_{s} \oint_{\mc S_s}
   \frac{1}{|\vb x - \vb x^\prime|}
           \,\sigma(\vb x^\prime) \, d\vb x^\prime
\\
%--------------------------------------------------------------------%
\vb E(\vb x) 
&= \vb E\sups{ext}(\vb x)
  +\frac{1}{4\pi\epsilon_0}
   \sum_{s} \oint_{\mc S_s}
   \frac{(\vb x-\vb x^\prime)}
        {|\vb x - \vb x^\prime|^3}
        \,\sigma(\vb x^\prime)\,d\vb x^\prime.
\end{align}
\label{ContinuousEquations1}
\end{subequations}
%====================================================================%
\noindent 
where $\phi\sups{ext}, \vb E\sups{ext}$ are the potential and field 
due to external field sources.\footnote{In a full-wave scattering 
problem these would be the ``incident'' fields, but in an 
electrostatic problem the word ``incident'' doesn't quite make 
sense, so we call them the ``external'' fields instead.}

There are several important distinctions between these 
equations and the corresponding equations in the full-wave
formulations: 
%====================================================================%
\begin{itemize}
 %--------------------------------------------------------------------%
 \item
 The integrals in (\ref{ContinuousEquations1}) are over 
 \textit{all} surfaces in the problem, not just the surfaces 
 bounding the region in which the evaluation point lies. 
 This is in contrast to the full-wave case, in which the 
 corresponding surface integrals range only over the surfaces 
 bounding the medium in question.
 %--------------------------------------------------------------------%
 \item
 Similarly, $\phi\sups{ext}, \vb E\sups{ext}$ contribute 
 to the total fields at \textit{all} points in space, not 
 just points living in the same regions as the external 
 field sources. This is again in contrast to the full-wave 
 case, where the incident field sources contribute only
 to fields at points in the same region as those sources.
 %--------------------------------------------------------------------%
 \item
 Equations (\ref{ContinuousEquations1}) do \textit{not}
 involve the dielectric constants of the various regions.
 Instead, surface charges on all surfaces contribute to 
 the potential as they would in vacuum. The dielectric 
 constants of the various materials enter only through
 the boundary conditions, discussed below.
 %--------------------------------------------------------------------%
 \item
 Another distinction with the full-wave case which is not 
 apparent from (\ref{ContinuousEquations1}) is that we can 
 have nonzero induced charge densities $\sigma$ even in 
 the absence of any external fields. This is possible
 if the problem involves conducting surfaces maintained
 at nonzero potentials.
\end{itemize}
%====================================================================%

%====================================================================%
%====================================================================%
%====================================================================%
\subsection*{Surface charge density expansion}

Now imagine approximating surface $\mc S_s$ as the union
of $N_s\supt{P}$ flat triangular panels, 
$$ \mc S_{s} = \cup \, \mc P_{sa}$$
where $a=1,\cdots,N_s\supt{P}$.
Let $P_{sa}$ have area $A_{sa}$ and surface normal
$\vbhat{n}_{sa}$. (We will worry about the direction
of $\vbhat{n}_{sa}$ later.)

To panel $\mc P_{sa}$ we assign a scalar-valued 
``pulse'' basis function $b_{sa}(\vb x)$ that is 
1 on the panel and 0 elsewhere:
$$ b_{sa}(\vb x) = 
   \begin{cases}
    1, \qquad &\vb x \in \mc P_{sa} \\
    0, \qquad &\text{otherwise}.
  \end{cases}
$$
We approximate the surface charge density on $\mc S_s$
as an expansion in the $b_{sa}$ functions:
$$ \frac{\sigma(\vb x)}{\epsilon_0} 
   \approx \sum_{a=1}^{N_s\supt{P}} \sigma_{sa} b_{sa}(\vb x)
   \qquad 
   \text{for }\vb x\in \mc S_s.
$$
Note that my $\sigma_{sa}$ unknowns have the dimensions
of 
$$\frac{\text{surface charge density}}{\text{permittivity}}=
  \frac{\text{volts}}{\text{length}}.
$$
In what follows, I will frequently use the collective
subscript $n=(sa)$ with $\sum_n=\sum_s \sum_a$, i.e.
a sum over $n$ runs over all panels on all surfaces.

%%%%%%%%%%%%%%%%%%%%%%%%%%%%%%%%%%%%%%%%%%%%%%%%%%%%%%%%%%%%%%%%%%%%%%
%%%%%%%%%%%%%%%%%%%%%%%%%%%%%%%%%%%%%%%%%%%%%%%%%%%%%%%%%%%%%%%%%%%%%%
%%%%%%%%%%%%%%%%%%%%%%%%%%%%%%%%%%%%%%%%%%%%%%%%%%%%%%%%%%%%%%%%%%%%%%
\subsection*{Potentials and fields from charge densities: 
             discretized forms} 

The electrostatic potential and field at $\vb x$ are 
%====================================================================%
\begin{subequations}
\begin{align}
\phi(\vb x) 
&= \phi_r\sups{ext}(\vb x)
   +
   \sum_{n} \sigma_n 
   \int_{\mc P_n} \frac{d\vb x^\prime}{4\pi |\vb x - \vb x^\prime|}
\\
%--------------------------------------------------------------------%
\vb E(\vb x) 
&= \vb E_r\sups{ext}(\vb x)
   +
   \sum_{n} \sigma_n 
   \int_{\mc P_n} \frac{(\vb x-\vb x^\prime)d\vb x^\prime}
                       {4\pi |\vb x - \vb x^\prime|^3}
\end{align}
\label{DiscreteEquations1}
\end{subequations}
where the sum is over all panels on all surfaces in the problem.
%====================================================================%

%%%%%%%%%%%%%%%%%%%%%%%%%%%%%%%%%%%%%%%%%%%%%%%%%%%%%%%%%%%%%%%%%%%%%%
%%%%%%%%%%%%%%%%%%%%%%%%%%%%%%%%%%%%%%%%%%%%%%%%%%%%%%%%%%%%%%%%%%%%%%
%%%%%%%%%%%%%%%%%%%%%%%%%%%%%%%%%%%%%%%%%%%%%%%%%%%%%%%%%%%%%%%%%%%%%%
\subsection*{Conditions on potentials and fields}

\subsubsection*{Conditions at PEC surfaces}

At PEC surfaces we impose the condition that the electrostatic
potential equal the specified potential for that conductor:
%====================================================================%
\begin{align}
 \phi(\vb x) &= V_s, \qquad \text{for } \vb x \in \mc S_s.
\nonumber
\intertext{Galerkin-testing with the expansion functions for 
           surface $\mc S_s$, we find }
 \int_{\mc P_{sa}} \phi(\vb x) d\vb x &= A_{sa} V_s,
 \qquad \text{for all panels $\mc P_{sa}$ on surface }\mc S_s.
\nonumber
 \intertext{Inserting (\ref{DiscreteEquations1}a), this reads }
 \sum_{n} \mathcal{I}\supt{(1)}_{mn} \sigma_n
 &= A_{m} V_m - \int_{\mc P_m} \phi\sups{ext}(\vb x)d\vb x
\label{PECPanelEquation}
\end{align}
where $V_m$ is the potential at which the conducting surface
containing panel $\mc P_m$ is held and
$$ \mathcal{I}\supt{(1)}_{mn} 
   \equiv 
   \int_{\mc P_m} \int_{\mc P_n} 
   \frac{1}{4\pi|\vb x - \vb x^\prime|} \, d\vb x^\prime \,d\vb x.
$$ 
%====================================================================%

\subsubsection*{Conditions at dielectric surfaces}

At non-PEC surfaces we impose the condition that the normal 
electric field exhibit the requisite discontinuity. 
If $\vb x$ is a point on a surface $\mc S_s$ lying between
regions $\mc R_r$ and $\mc R_{r^\prime}$, the condition is 
\numeq{EFieldDiscontinuity}
{
 \epsilon_r \left|\pard{\phi}{\vbhat{n}}\right|_{\vb x^+}
=
 \epsilon_{r^\prime} \left|\pard{\phi}{\vbhat{n}}\right|_{\vb x^-}
}
where $\vbhat{n}$ is the surface normal pointing 
away from $\mc R_r$ into $\mc R_{r^\prime}$, and
where $x^+$ and $x^-$ are points lying
infinitesimally displaced from $\vb x$ along
$\vbhat{n}$ into $\mc R_r$ and $\mc R_{r^\prime}$.

When we seek to enforce condition (\ref{EFieldDiscontinuity}) 
at a point $\vb x$ lying within a panel $\mc P_{sa}$
on $\mc S_s$, we find the following dichotomy:
%====================================================================%
\begin{enumerate}
  \item Surface charges on $\mc P_{sa}$ contribute to 
        the two sides of (\ref{EFieldDiscontinuity})
        with \textit{opposite} signs.
  \item Surface charges on all other panels, as well as the 
        external field sources, contribute
        to the two sides of (\ref{EFieldDiscontinuity})
        with the \textit{same} sign.
\end{enumerate}
Equation (\ref{EFieldDiscontinuity}) thus reads, for a 
point $\vb x$ on $\mc P_m$,
%====================================================================%
\begin{align*}
&\epsilon_r 
   \left[ \frac{\sigma_m}{2} 
          + \sum_{n\ne m} \sigma_n \int_{\mc P_{n}}
            \frac{ \vbhat{n}_m \cdot (\vb x-\vb x^\prime)}
                 { 4\pi |\vb x-\vb x^\prime|^3 } \, d\vb x^\prime
          + \vbhat{n}_m \cdot \vb E\sups{ext}(\vb x)
   \right]
\\
&\qquad
=\epsilon_{r^\prime}
   \left[  -\frac{\sigma_m}{2} 
          + \sum_{n\ne m} \sigma_n \int_{\mc P_{n}}
            \frac{ \vbhat{n}_m \cdot (\vb x-\vb x^\prime)}
                 { 4\pi |\vb x-\vb x^\prime|^3 } \, d\vb x^\prime
          + \vbhat{n}_m \cdot \vb E\sups{ext}(\vb x)
   \right]
\end{align*}
or 
\numeq{DielectricCondition}
{ \sigma_m 
  + \Delta_{rr^\prime}
    \sum_{n\ne m} \sigma_n \int_{\mc P_{n}}
    \frac{ \vbhat{n}_m \cdot (\vb x-\vb x^\prime)}
         { 4\pi |\vb x-\vb x^\prime|^3 } \, d\vb x^\prime
   = -\Delta_{rr^\prime}\vbhat{n}_m \cdot \vb E\sups{ext}(\vb x)
}
with
\numeq{DeltaDef}
{\Delta_{rr^\prime} = 2\frac{\epsilon_r-\epsilon_r^\prime}{\epsilon_r + \epsilon_r^\prime}.}
Now Galerkin-test (\ref{DielectricCondition}) with the pulse basis
function associated with panel $m$:
\numeq{DielectricPanelEquation}
{
   A_{m} \sigma_m 
   + \Delta_{rr^\prime} \sum_{n\ne m} \mathcal{I}\sups{(2)}_{mn} \sigma_n
   = -\Delta_{rr^\prime} \int_{\mc P_m} \vbhat{n}_m \cdot \vb E\sups{ext}(\vb x) d\vb x
}
where 
$$ \mathcal{I}\supt{(2)}_{mn} 
   \equiv 
   \int_{\mc P_m} \int_{\mc P_n} 
   \frac{\vbhat{n}_m \cdot (\vb x-\vb x^\prime)}
        {4\pi|\vb x - \vb x^\prime|^3} \, d\vb x^\prime \,d\vb x.
$$ 

\subsubsection*{Conditions at $\lambda-$ surfaces}

As a generalization of the usual electrostatics
for PEC and dielectric bodies, {\sc scuff-static} 
also supports a modified type of boundary conditions
that we will call $\lambda$-\textit{conditions}.
The $\lambda$ boundary condition is defined by
\numeq{LambdaCondition}
{
  \lambda\bigg\{
    \left|\frac{d\phi}{d\vbhat{n}}\right|_{\vb x^+} 
   -\left|\frac{d\phi}{d\vbhat{n}}\right|_{\vb x^-} 
          \bigg\}
   =\phi(\vb x).
}
Surfaces satisfying $\lambda$ conditions will be 
known as $\lambda$-\textit{surfaces}. We may think 
of $\lambda$-surfaces as a type of semiconducting
surface on which reside a surface charge density
proportional to the local electrostatic potential.

It's easy to write down the Galerkin-tested version
of (\ref{LambdaCondition}). Consider testing each
side with the pulse basis function associated with
panel $m$. By the dichotomy discussed above, the 
contribution of all other panels $\mc P_{n\ne m}$ to 
the quantity in curly brackets vanishes, as does the 
contribution of the external field. The contribution
of $\mc P_m$ may be evaluated in a way similar
to our evaluation of the diagonal matrix element
in the dielectric case discussed above. The Galerkin
test of the RHS proceeds similarly to our discussion
of the PEC case above. We obtain
%====================================================================%
$$ \lambda A_m \sigma_m - \sum_{mn} \mc{I}_{mn}^{(1)} \sigma_{n}
   =\int_{\mc P_m} \phi\sups{ext}(\vb x)\,d\vb x.
$$
%====================================================================%


\subsubsection*{BEM system}

Assembling equations (\ref{PECPanelEquation}) for PEC panels,
(\ref{DielectricPanelEquation}) for all dielectric panels,
and (\ref{LambdaCondition}) for all $\lambda$ panels 
into a big linear system, we have 
%====================================================================%
\numeq{BEMSystem}{ \vb M \boldsymbol{\sigma} = \vb v }
%====================================================================%
where the $m$th entry of the unknown vector $\sigma$ is the 
surface charge density (divided by $\epsilon_0$) on the $m$th
panel, and where the elements of the BEM matrix and the RHS vector are
%====================================================================%
\paragraph{If panel $m$ is on a PEC surface:}
$$
  M_{mn} = \mathcal{I}\sups{(1)}_{mn}, 
   \hspace{1.2in}
   v_m = A_{m} V_m - \int_{\mc P_m} \phi\sups{ext}(\vb x) \, d\vb x.
$$
\paragraph{If panel $m$ is on a dielectric surface:}
$$ M_{mn} = 
   \begin{cases} 
     A_m, \qquad &m=n \\
     \Delta_{rr^\prime} \mathcal{I}\sups{(2)}_{mn}, \qquad &m\ne n
   \end{cases} 
   \qquad 
   v_m = -\Delta_{rr^\prime} \int_{\mc P_m} \vbhat{n}_m \cdot \vb E\sups{ext}(\vb x) \, d\vb x.
$$
\paragraph{If panel $m$ is on a $\lambda$-surface:}
$$ M_{mn} = 
     \lambda A_{m}\delta_{mn} - \mathcal{I}\sups{(1)}_{mn},
   \qquad 
   v_m = + \int_{\mc P_m} \phi\sups{ext}(\vb x) \, d\vb x.
$$

%====================================================================%
In these equations, 
\begin{itemize}
 \item $V_m$ denotes the potential at which the conducting surface
       containing panel $\mc P_m$ is held.
 \item $\Delta_{rr^\prime}$ is quantity (\ref{DeltaDef}) with $\epsilon_r$
       and $\epsilon_{r^\prime}$ the permittivities of the regions 
       \textit{exterior} and \textit{interior} to the surface containing
       $\mc P_m,$ respectively.
\end{itemize}

%%%%%%%%%%%%%%%%%%%%%%%%%%%%%%%%%%%%%%%%%%%%%%%%%%%%%%%%%%%%%%%%%%%%%%
%%%%%%%%%%%%%%%%%%%%%%%%%%%%%%%%%%%%%%%%%%%%%%%%%%%%%%%%%%%%%%%%%%%%%%
%%%%%%%%%%%%%%%%%%%%%%%%%%%%%%%%%%%%%%%%%%%%%%%%%%%%%%%%%%%%%%%%%%%%%%
\newpage
\section{Output Modules}

In this section we discuss the implementation of some of the 
{\sc scuff-static} output modules.

\subsection{Capacitance matrix}

Consider a geometry consisting of $N$ conducting bodies
(possibly in the presence of any number of additional
dielectric bodies). The capacitance matrix is an $N\times N$
matrix whose $m,n$ entry $C_{mn}$ gives the charge 
induced on body $m$ when conductor $n$ is maintained at 
a potential of 1 volt and all other conductors are maintained
at zero volts. This is computed in {\sc scuff-static}
by performing $N$ separate electrostatic calculations---that 
is, solving equation (\ref{BEMSystem}) for $N$ separate RHS 
vectors but the same matrix in each case---with the 
$n$th RHS vector $\vb v_n$ corresponding to the case 
in which the conductor potentials are 
$$V_m = (1\text{ volt})\cdot \delta_{mn}, \qquad 
  m=\{1,\cdots,N\}.$$
After solving the system for $\vbsigma$,
we compute the total induced charge on body $m$ by 
simply summing the charges of all panels on that body:
$$ C_{mn} 
   = 
   \epsilon_0 \cdot \sum_{p\in \text{panels on $\mc S_m$}} A_p \sigma_p 
$$
where $A_p$ is the panel area.

\subsection{Polarizability}

\subsection{C-matrix}

The ``$C$-matrix'' (not to be confused with the capacitance
matrix) is a sort of DC version of the ``T-matrix'' familiar
from full-wave electromagnetic scattering 
theory.\footnote{Recall that {\sc scuff-em} includes a code
for computing full-wave T-matrices for objects of arbitrary 
shapes and materials:\url{http://homerreid.com/scuff-EM/scuff-tmatrix}.}
The rows and columns of the $C$-matrix are indexed by 
spherical harmonic indices $(\ell m)$. In brief, the matrix entry 
$C_{\ell^\prime m^\prime, \ell m}$ is the coefficient
of the $(\ell^\prime m^\prime)$ spherical multipole term
in the ``scattered'' or ``outgoing'' electrostatic potential 
due to an object exposed to an ``incoming'' 
$(\ell, m)-$spherical multipole field.

\subsubsection*{Real-valued spherical harmonics}
{\sc scuff-static} works in a basis of \textit{real-valued}
spherical harmonics $\YY_{\ell m}$. These are defined in 
terms of the usual complex-valued spherical harmonics 
$Y_{\ell m}$ by simply replacing the $e^{im\phi}$ factor 
in $Y$ with a $\cos m\phi$ factor for positive $m$
and a $\sin m\phi$ factor for negative $m$ 
(and then adjusting the overall normalization such that
the new functions are orthonormal).
More specifically, we put
%====================================================================%
\numeq{YYDefinition}
{  \YY_{\ell, m} \equiv
   \begin{cases}
     %--------------------------------------------------------------------%
     \frac{1}{\sqrt{2}}\Big[ Y_{\ell,m} + (-1)^{m} Y_{\ell,-m} \Big], \qquad 
     &m>0
     \\
     %--------------------------------------------------------------------%
     Y_{\ell,m} \qquad 
     &m=0
     \\
     %--------------------------------------------------------------------%
     \frac{1}{i\sqrt{2}}\Big[ Y_{\ell,m} - (-1)^m Y_{\ell,-m} \Big], \qquad
     &m<0.
   \end{cases}
}
The $\YY$ functions satisfy the orthogonality relation
$$ \int \YY_{\ell m}(\theta, \phi)                 \,
        \YY_{\ell^\prime m^\prime}(\theta, \phi)   \,
        d\Omega
   =\delta_{\ell, \ell^\prime} \delta_{m,m^\prime}
$$
%====================================================================%
It is convenient to introduce a single index $\alpha$ 
that runs over $(\ell,m)$ tuples according to 
%====================================================================%
$$\alpha(\ell,m) = \ell(\ell + 1) + m. $$ 
%====================================================================%
The following table lists the $\alpha$ indices and 
$\YY$ functions for the first few $(\ell,m)$ tuples:
%====================================================================%
\renewcommand{\arraystretch}{2.0}
$$\begin{array}{|c|c|l|}\hline
\alpha & (\ell, m) & \hspace{0.5in} \YY_{\ell m}          \\\hline
0      & (   0, 0) & +\sqrt\frac{1}{4\pi} \\\hline
%--------------------------------------------------------------------%
1      & (   1,-1) & -\sqrt{\frac{3}{4\pi}} \sin\theta\sin\phi  \\\hline
2      & (   1, 0) & +\sqrt\frac{3}{4\pi} \cos\theta          \\\hline
3      & (   1,+1) & -\sqrt{\frac{3}{4\pi}} \sin\theta\cos\phi  \\\hline
4      & (   2,-2) & -\sqrt\frac{15}{16\pi} \sin^2\theta \sin 2\phi \\ \hline
5      & (   2,-1) & -\sqrt\frac{15}{16\pi} \cos\theta \sin\theta \sin\phi \\ \hline
6      & (   2, 0) & +\sqrt\frac{5}{16\pi}(1+3\cos 2\theta) \\ \hline
7      & (   2,+1) & -\sqrt\frac{15}{16\pi} \cos\theta \sin\theta \cos\phi \\ \hline
8      & (   2,+2) & +\sqrt\frac{15}{16\pi} \sin^2\theta \cos 2\phi \\ \hline
\end{array}$$
\renewcommand{\arraystretch}{1.0}
%====================================================================%

\subsubsection*{Definition of $C$-matrix}

The entries of the $C$-matrix are defined as the coefficients
in a spherical-wave expansion of the ``outgoing'' or 
``scattered'' potential due to a given ``incident'' or
``external'' spherical-wave potential. More generally,
we consider an object subject to a given external potential 
$\phi\sups{ext}(\vb x)$. The total potential is the sum
of the external potential plus a ``scattered'' contribution
arising from the induced charge density on the object:
%====================================================================%
\begin{equation}
\phi(\vb x) = \phi\sups{ext}(\vb x) + \phi\sups{scat}(\vb x).
\end{equation}
%====================================================================%
To define the $C$-matrix, we take the external potential 
to be a unit-strength ``incoming'' $(\ell, m)$ spherical wave:
%====================================================================%
\begin{equation}
\phi\sups{ext}(\vb x)=r^{\ell} \YY_{\ell m}(\theta, \phi).
\end{equation}
%====================================================================%
Then the ``scattered'' potential can quite generally be 
represented as a sum of ``outgoing'' spherical waves, and the 
coefficients in this expansion are the elements of the 
$C$-matrix:
%====================================================================%
\numeq{CMatrixDefinition}
{\phi\sups{scat}(\vb x) 
   = \sum_{\ell^\prime, m^\prime}
     C_{\ell^\prime m^\prime, \ell m} 
           \frac{\YY_{\ell^\prime m^\prime}(\theta, \phi)}
                {r^{\ell^\prime + 1 }}.
}
%====================================================================%

\subsubsection*{$C$-matrix elements from charge-density projections}
The elements of the $C$-matrix may be computed from the induced
surface charge density $\sigma(\vb x)$ on the surfaces of 
the objects in a scattering problem. To do so, first note that
the ``scattered'' potential may be obtained from $\sigma$ 
in the form of a surface integral:
%====================================================================%
\begin{align}
 \phi\sups{scat}(\vb x) 
&= 
 \int G(\vb x-\vb x^\prime) \sigma(\vb x^\prime) d\vb x^\prime
\intertext{where $G(\vb r)=\frac{1}{4\pi |\vb r|}$ is the 
usual electrostatic Green's function. Now insert the 
spherical-wave expansion of $G$:}
&=\int \bigg\{
   \sum_{l^\prime m^\prime}
   \frac{1}{(2l^\prime + 1)} 
   \frac{ (r^\prime)^{\ell^\prime} }
        { r^{\ell^\prime+1} }
   \YY_{\ell^\prime m^\prime}(\theta,\phi)
   \YY_{\ell^\prime m^\prime}(\theta^\prime,\phi^\prime)
      \bigg\}
      \sigma(\vb x^\prime)d\vb x^\prime
\\[5pt]
&= \sum_{l^\prime m^\prime}
   \bigg[\underbrace{\frac{1}{(2l^\prime + 1)}
               \int (r^\prime)^{\ell^\prime}
                    \YY_{\ell^\prime m^\prime}
                        (\theta^\prime,\phi^\prime)
                    \sigma(\vb x^\prime)
                    d\vb x^\prime
              }_{=C_{\ell^\prime m^\prime, \ell m}}
   \bigg]
   \frac{ \YY_{\ell^\prime m^\prime}(\theta,\phi) }
        { r^{\ell^\prime+1} }.
\end{align}
Comparing to (\ref{CMatrixDefinition}), we have the relation
between the $C$-matrix coefficients and the charge density
induced by the incoming $(\ell,m)-$wave potential: 
%====================================================================%
\numeq{CFromSigma}
{ C_{\ell^\prime m^\prime, \ell m}
  = \frac{1}{(2l^\prime + 1)}
    \int (r^\prime)^{\ell^\prime}
         \YY_{\ell^\prime m^\prime}(\theta^\prime,\phi^\prime)
         \sigma(\vb x^\prime) d\vb x^\prime.
}

%====================================================================%

\subsubsection*{Computation of $C$-matrix in {\sc scuff-static}}

The computation of the $C$-matrix now proceeds in analogy
to the capacitance calculation outlined above:
%====================================================================%
\begin{enumerate}
 \item We solve the BEM electrostatics problem (\ref{BEMSystem})
       with the RHS vector corresponding to an $(\ell, m)$
       external field:
       %====================================================================%      
       \numeq{PhiExtCMatrix} 
         { \phi\sups{ext}(\vb x) = r^\ell \YY_{\ell m}(\theta, \phi).}
       %====================================================================%      
 \item Then, for each $(\ell^\prime, m^\prime),$ we compute
       the $(\ell^\prime, m^\prime)$ spherical-multipole
       moment of the induced charge distribution:
       %====================================================================%      
       \begin{align*}
        C_{\ell^\prime m^\prime, \ell m} 
         &=\frac{1}{2l^\prime+1}\oint d\vb x \, r^{\ell^\prime} \YY_{\ell^\prime m^\prime}(\vbhat{x}) \, \sigma(\vb x)
         \intertext{where the surface integral ranges over all surfaces in the problem.
                    We evaluate the integral approximately via a one-point cubature
                    that pretends the integrand is constant over the surface of 
                    each panel and equal to its value at the panel centroid:}
         &\approx \sum_p A_p \sigma_p |\vb x_p|^{\ell^\prime} \YY_{\ell^\prime m^\prime} (\vbhat{x}_p)
       \end{align*}
       where the sum is over all panels on all object surfaces and
       $\vb x_p$ is the centroid of panel $\mc P_p$.
       %====================================================================%      
\end{enumerate}
%====================================================================%
(Note that the $C$-matrix computed is the $C$-matrix for the entire
geometry including all objects).

%====================================================================%

\subsubsection*{Comparison to the usual $C$-matrix} 
 
How do the elements of our $C$-matrix, defined in terms
of real-valued spherical harmonics, compare with the 
usual definition in terms of the usual complex-valued
spherical harmonics? (We will use the notation 
$\overline{C}_{\ell^\prime m^\prime, \ell m}$ to 
indicate the matrix elements as defined by this usual 
definition.) In the usual definition, we take 
the ``incident'' field to be
%%%%%%%%%%%%%%%%%%%%%%%%%%%%%%%%%%%%%%%%%%%%%%%%%%%%%%%%%%%%%%%%%%%%%%
\numeq{UsualPhiExt}
{ \phi\sups{ext}(\vb x) =  r^\ell Y_{\ell m}(\theta, \phi) }
%%%%%%%%%%%%%%%%%%%%%%%%%%%%%%%%%%%%%%%%%%%%%%%%%%%%%%%%%%%%%%%%%%%%%%
and then define the $\overline{C}$-matrix entries according to
%%%%%%%%%%%%%%%%%%%%%%%%%%%%%%%%%%%%%%%%%%%%%%%%%%%%%%%%%%%%%%%%%%%%%%
\numeq{UsualCMatrixDefinition}
{\phi\sups{scat}(\vb x) 
   = \sum_{\ell^\prime, m^\prime}
     \overline{C}_{\ell^\prime m^\prime, \ell m} 
           \frac{Y_{\ell^\prime m^\prime}(\theta, \phi)}
                {r^{\ell^\prime + 1 }}.
}
Here we are using the symbol $\overline{C}$ for the 
entries of the $C$-matrix.
%%%%%%%%%%%%%%%%%%%%%%%%%%%%%%%%%%%%%%%%%%%%%%%%%%%%%%%%%%%%%%%%%%%%%%

To obtain the relation between the $\vb C$ and $\overline{\vb C}$
matrices, consider an external field consisting of a superposition
of incoming real-valued spherical waves:
%====================================================================%
\begin{align}
 \phi\sups{ext} 
  &= \sum_{\ell, m} a_{\ell, m} r^\ell \mc Y_{\ell,m}(\theta,\phi)
\intertext{and write the external field as a superposition
of outgoing real-valued spherical waves:}
 \phi\sups{scat}
  &= \sum_{\ell, m} b_{\ell, m} \frac{\mc Y_{\ell,m}(\theta,\phi)}{r^{\ell+1}}.
\end{align}
%====================================================================%
Then the $a_{\ell,m}$ and $b_{\ell,m}$ coefficients are related
according to 
%%%%%%%%%%%%%%%%%%%%%%%%%%%%%%%%%%%%%%%%%%%%%%%%%%%%%%%%%%%%%%%%%%%%%%
\numeq{MyBCA} { \vb b=\vb C\vb a }
%%%%%%%%%%%%%%%%%%%%%%%%%%%%%%%%%%%%%%%%%%%%%%%%%%%%%%%%%%%%%%%%%%%%%%
where $\vb a,\vb b$ are vectors of $a$ and $b$ coefficients
and $\vb C$ is the $C$-matrix as defined above with real-valued
spherical harmonics.

On the other hand, we could just as easily have expressed
the \textit{same} external and scattered fields in terms of 
complex-valued spherical harmonics,
%%%%%%%%%%%%%%%%%%%%%%%%%%%%%%%%%%%%%%%%%%%%%%%%%%%%%%%%%%%%%%%%%%%%%%
\begin{equation}
 \phi\sups{ext} 
  = \sum_{\ell, m} \overline{a}_{\ell, m} r^\ell Y_{\ell,m}(\theta,\phi), 
 \qquad  
 \phi\sups{scat}
  = \sum_{\ell, m} \overline{b}_{\ell, m} 
                   \frac{Y_{\ell,m}(\theta,\phi)}{r^{\ell+1}}
\end{equation}
%%%%%%%%%%%%%%%%%%%%%%%%%%%%%%%%%%%%%%%%%%%%%%%%%%%%%%%%%%%%%%%%%%%%%%
in which case the vectors of $\overline{b}$ and $\overline{a}$
coefficients would be related by the conventionally-defined
$C$-matrix,
%%%%%%%%%%%%%%%%%%%%%%%%%%%%%%%%%%%%%%%%%%%%%%%%%%%%%%%%%%%%%%%%%%%%%%
\numeq{UsualBCA}
 {\overline{\vb b}=\overline{\vb C} \overline{\vb a}.}
%%%%%%%%%%%%%%%%%%%%%%%%%%%%%%%%%%%%%%%%%%%%%%%%%%%%%%%%%%%%%%%%%%%%%%
Finally, the $a$ and $b$ coefficients may be related to
the $\overline{a}$ and $\overline{b}$ coefficients using
the definition (\ref{YYDefinition}), which reads
%%%%%%%%%%%%%%%%%%%%%%%%%%%%%%%%%%%%%%%%%%%%%%%%%%%%%%%%%%%%%%%%%%%%%%
$$
  \boldsymbol{\Gamma}\vb a=\overline{\vb a},
  \qquad
  \boldsymbol{\Gamma}\vb b=\overline{\vb b}
$$
%%%%%%%%%%%%%%%%%%%%%%%%%%%%%%%%%%%%%%%%%%%%%%%%%%%%%%%%%%%%%%%%%%%%%%
where the elements of the $\boldsymbol{\Gamma}$ matrix are 
%%%%%%%%%%%%%%%%%%%%%%%%%%%%%%%%%%%%%%%%%%%%%%%%%%%%%%%%%%%%%%%%%%%%%%
\BE
 \Gamma_{\ell m, \ell^\prime m^\prime}
  = \begin{cases}
    \frac{ \delta_{\ell\ell^\prime} } { \sqrt{2} }
    \big[ \delta_{m,m^\prime} + (-1)^m\delta_{m,-m^\prime}\big], 
    \qquad &m^\prime>0 
    \\[5pt]
    \delta_{\ell\ell^\prime}\delta_{m0},
    \qquad &m^\prime=0 
    \\[5pt]
    \frac{ \delta_{\ell\ell^\prime} } { i\sqrt{2} }
    \big[ \delta_{m,m^\prime} - (-1)^m\delta_{m,-m^\prime}\big], 
    \qquad &m^\prime<0.
    \end{cases}
\EE
%%%%%%%%%%%%%%%%%%%%%%%%%%%%%%%%%%%%%%%%%%%%%%%%%%%%%%%%%%%%%%%%%%%%%%
(Note that $\boldsymbol \Gamma$ is unitary, 
$\boldsymbol{\Gamma}^\dagger=\boldsymbol{\Gamma}^{-1}$.)
Using these equations, I can rewrite (\ref{MyBCA}) in the form
\begin{align}
 \boldsymbol{\Gamma}^\dagger \overline{\vb b} 
&= \vb C \boldsymbol{\Gamma}^\dagger \overline{\vb a}
\nonumber\intertext{or}
 \overline{\vb b} 
&= \boldsymbol{\Gamma} \vb C \boldsymbol{\Gamma}^\dagger \overline{\vb a}.
\label{NewBCA}
\end{align}
Comparing (\ref{NewBCA}) to (\ref{UsualBCA}) yields the 
transformation matrix from the real-valued spherical-wave
$C$ matrix to the conventionally-normalized $\overline{C}$-matrix:
\BE 
   \overline{\vb C} 
   =
   \boldsymbol{\Gamma} \vb C \boldsymbol{\Gamma}^\dagger.
\EE
\end{document}

%%%%%%%%%%%%%%%%%%%%%%%%%%%%%%%%%%%%%%%%%%%%%%%%%%%%%%%%%%%%%%%%%%%%%%
%%%%%%%%%%%%%%%%%%%%%%%%%%%%%%%%%%%%%%%%%%%%%%%%%%%%%%%%%%%%%%%%%%%%%%
%%%%%%%%%%%%%%%%%%%%%%%%%%%%%%%%%%%%%%%%%%%%%%%%%%%%%%%%%%%%%%%%%%%%%%
\section*{Fields of individual panels}

\begin{align*}
 \phi(\vb x) &= \sum_n \sigma_n \mathbb{P}_n(\vb x) \\
 \vb E(\vb x) &= \sum_n \sigma_n \mathbb{E}_n(\vb x) \\
\end{align*}

$$ \vb x^\prime = V_0 + u\vb A + v\vb B $$
%====================================================================%
$$ \vb x - \vb x^\prime = u\vb A + v\vb B + (\vb V_0 - \vb x) $$
%====================================================================%
$$ |\vb x - \vb x^\prime|^2 = 
    C_{20} u^2 + C_{11} uv + C_{02} v^2 + 
    C_{10} u + C_{01} v + C_{00}
$$
%====================================================================%
$$
\begin{array}{cclcccl}
  C_{20} &=& |\vb A|^2 
  &\qquad &
  C_{02} &=& |\vb B|^2 
\\
  C_{11} &=& 2\vb A \cdot \vb B
  &\qquad &
  C_{10} &=& 2\vb A \cdot (\vb V_0-\vb x)
\\
  C_{01} &=& 2\vb B \cdot (\vb V_0-\vb x)
  &\qquad &
  C_{00} &=& |\vb V_0-\vb x|^2
\end{array}
$$
%====================================================================%
It is convenient to write the distance between points in the form
$$ |\vb x - \vb x^\prime|^2 =
    C_{02} \Big[ (v+v_0)^2 + \xi^2 \Big]
$$
where
$$ v_0=\frac{1}{2}\big(C_{11}^\prime u + C_{01}^\prime\big), 
   \qquad
   \xi^2 = \Big[ C_{20}^\prime - \frac{1}{4}C_{11}^{\prime 2} \Big]u^2
          +\Big[ C_{10}^\prime - \frac{1}{2}C_{11}^{\prime}C_{01}^\prime\Big]u
          +\Big[ C_{00}^\prime - \frac{1}{4}C_{01}^{\prime 2}\Big]
$$
with the shorthand
$$C_{ij}^\prime = \frac{C_{ij}}{C_{02}}.$$
Then we have
%====================================================================%
%====================================================================%
\begin{align*}
 \mathbb{P}_n(\vb x) &= 
 \int_{\mc P_n} \frac{d\vb x^\prime}{|\vb x-\vb x^\prime|}
\\
&=
 \int_0^1 du \, \int_0^u \, dv 
    \frac{1}
         {\sqrt{ C_{20}u^2 + C_{11}uv + C_{02}v^2 + C_{10}u + C_{01}v + C_{00}}}
\intertext{Substitute $v=u\alpha, dv=ud\alpha:$}
&=
 \int_0^1 du \, \int_0^1 \, d\alpha
    \frac{u}
         {\sqrt{  \Big(C_{20}+C_{11}\alpha+C_{02}\alpha^2\Big)u^2
                 +\Big(C_{10}+C_{01}\alpha\Big)u + C_{00}}} 
\intertext{Evaluate the $u$ integral:}
 \int_0^1 du \, \int_0^u \, dv 
\end{align*}
%====================================================================%
