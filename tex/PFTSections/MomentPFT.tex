%%%%%%%%%%%%%%%%%%%%%%%%%%%%%%%%%%%%%%%%%%%%%%%%%%%%%%%%%%%%%%%%%%%%%%
%%%%%%%%%%%%%%%%%%%%%%%%%%%%%%%%%%%%%%%%%%%%%%%%%%%%%%%%%%%%%%%%%%%%%%
%%%%%%%%%%%%%%%%%%%%%%%%%%%%%%%%%%%%%%%%%%%%%%%%%%%%%%%%%%%%%%%%%%%%%%
\newpage
\section{Cartesian-multipole-moment PFT}
\label{MomentPFTSection}

The fact that $\text{Im }\mb G(\vb x, \vb x^\prime)$
in equations (\ref{EMTPFTScattering1}) is nonsingular
at $x=x^\prime$ suggests the possibility of a
multipole expansion relating PFTs to the low-order
moments of the current distribution. In this section 
we investigate this possibility for the nonmagnetic
case ($\vb M=0$).

Inserting the short-distance expansion (\ref{ImGExpansion})
into (\ref{EMTPFTScattering1}a) and keeping only terms of lowest
order in $k$ yields
%====================================================================%
\begin{align}
  P\sups{scat}
&\approx
\frac{k^2 Z_0}{12 \pi}
  \underbrace{ \int J_i^*(\vb x) d\vb x }_{i\omega p^*_i}
  \underbrace{ \int J_i  (\vb x^\prime) d\vb x^\prime }_{-i\omega p_i}
\\
&=-\frac{c^2 k^4 Z_0}{12 \pi}|\vb p|^2
\label{PMultipole}
\end{align}
%====================================================================%
where $\vb p=-\frac{1}{i\omega} \int \vb J dV$ is the
dipole moment of the induced current distribution.
Equation (\ref{PMultipole}) is just the usual result for 
the total power radiated by a 
point dipole~\cite{Jackson1999}.

Proceeding similarly for the force, from (\ref{ImGExpansion})
one first finds
%====================================================================%
$$ \partial_i G_{jk} = \frac{k^3}{60\pi}
   \Big( r_j \delta_{ik} + r_k \delta_{ij} - 4r_i \delta_{jk} \Big)
   + O(k^5)
$$
%====================================================================%
whereupon (\ref{EMTPFTScattering1}b) reads
%====================================================================%
\begin{align}
  F_i\sups{scat}
&\approx
-\frac{k^3 Z_0}{120 \pi c}
  \text{Im }
  \int \bigg\{  J^*_j(\vb x)(\vb x-\vb x^\prime)_j J_i(\vb x^\prime)
\\
&\hspace{1in}
               +J^*_i(\vb x)(\vb x-\vb x^\prime)_j J_j(\vb x^\prime)
\nn
&\hspace{1in}
              -4J^*_j(\vb x)(\vb x-\vb x^\prime)_i J_j(\vb x^\prime)
       \bigg\}\,d\vb x \, d\vb x^\prime
\nn
&= -\frac{k^3 Z_0}{60 \pi c}\text{Im }
   \Big[  \mc M^*_{jj} \mc M_i  + \mc M^*_{ij} \mc M_j
        -4\mc M^*_{ji} \mc M_j \Big]
\label{ForceMultipole1}
\end{align}
where I defined 
%====================================================================%
$$ \mc M_{i}  \equiv \int J_i(\vb x)     \, dV, \qquad
   \mc M_{ij} \equiv \int J_i(\vb x) x_j \, dV.
$$
%====================================================================%
The quantity $\mc M_i$ is related to the electric dipole moment $\vb p$ by 
$$\vb p_i = -\frac{1}{i\omega} \mc M_i.$$
On the other hand, $\mc M_{ij}$ is related to the magnetic
dipole and electric quadrupole moments; in essence,
the magnetic dipole moment $\vb m$ 
is the antisymmetric part of $\mc M_{ij}$,
while the electric quadrupole moment $Q_{ij}$ 
is the symmetric part.
In terms of the standard definitions~\cite{Jackson1999} one finds
%====================================================================%
\begin{align*}
 m_i 
&= \frac{1}{2}\varepsilon_{ijk}\int x_i J_j(\vb x) dV 
\\
&= \frac{1}{2}\varepsilon_{ijk}\mc M_{ji}
\\
&= -\frac{1}{2}\varepsilon_{ijk}\mc M_{ij}
\\
 Q_{ij} 
&=
 -\frac{1}{i\omega} 
 \int \Big\{ 3J_i x_j + 3x_i J_j - 2J_k x_k \delta_{ij} \Big\} \, dV
\\
&=
 -\frac{1}{i\omega}
  \Big[ 3\mc M_{ij} + 3\mc M_{ji} - 2\delta_{ij} \mc M_{kk} \Big].
\end{align*}
%====================================================================%
Using these definitions in (\ref{ForceMultipole1}) and performing
some algebra, one finds the lowest-order terms in the multipole 
expansion of the self-force:
%====================================================================%
\begin{align}
  F_i\sups{scat}
\label{ForceMultipole1}
&\approx
\frac{k^4 Z_0}{12\pi}\text{Re}\Big( \vb m^* \times \vb p \Big)_i
+
\frac{c k^5 Z_0}{120\pi}\text{Im}\Big( \vb Q^* \vb p \Big)_i
\end{align}
%====================================================================%
The quantity in the second term involves the matrix-vector product 
of the 3$\times $3 matrix $\vb Q^*$ with the 3-vector $\vb p$.

Finally, for the torque, inserting (\ref{ImGExpansion}) into
(\ref{EMTPFTScattering1}b) yields
%====================================================================%
\begin{align}
  \mc T_i\sups{scat}
&\approx
-\frac{k Z_0}{12 \pi c}\varepsilon_{ijk}
  \text{Im }\left\{ 
\left[\int \vb J^*_j dV\right] \left[\int \vb J_k dV\right]
            \right\}
\nn
&= -\frac{c k^3 Z_0}{6 \pi}
    \Big(\text{Re }\vb p \times \text{Im }\vb p\Big)_i.
\label{TorqueMultipole}
\end{align}
%====================================================================%

