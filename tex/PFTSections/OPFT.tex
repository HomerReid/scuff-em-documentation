%%%%%%%%%%%%%%%%%%%%%%%%%%%%%%%%%%%%%%%%%%%%%%%%%%%%%%%%%%%%%%%%%%%%%%
%%%%%%%%%%%%%%%%%%%%%%%%%%%%%%%%%%%%%%%%%%%%%%%%%%%%%%%%%%%%%%%%%%%%%%
%%%%%%%%%%%%%%%%%%%%%%%%%%%%%%%%%%%%%%%%%%%%%%%%%%%%%%%%%%%%%%%%%%%%%%
\newpage
\section{Overlap PFT (OPFT)}

In the DSIPFT approach, the field components needed to compute
the Poynting vector (PV) or Maxwell stress tensor (MST) at a point
$\vb x$ on a bounding surface are obtained from the surface
currents by linear convolution. In fact, since the fields 
enter these expressions quadratically, evaluating the PV
or MST at a single point requires \textit{two} separate 
convolutions. Thus, for example, equation
(\ref{PFluxVMVP}) for the Poynting flux at $\vb x$ involves
a double sum over surface-current expansion coefficients.
This means that the DSIPFT approach is computationally
costly.

We achieve a considerable simplification by allowing the 
bounding surface $\mc S$ to coincide with the body surface.
In this case, computing the PV and MST requires only 
knowledge of the fields at the object surface.
But the fields at the object surface may be obtained
\textit{directly} from the surface currents---with no
convolution---simply by appealing to the definition
of the surface currents:
%====================================================================%
$$
 \vb E = \vbhat{n} \times \vb N 
         + \frac{\nabla\cdot\vb K}{i\omega \epsilon}\vbhat{n}, 
\qquad
 \vb H = -\vbhat{n} \times \vb K
         + \frac{\nabla\cdot\vb N}{i\omega \mu}\vbhat{n}.
$$
%====================================================================%
Thus the double sum in e.g. equation (\ref{PFluxVMVP}) for the Poynting 
flux at a point $\vb x$ on the body surface collapses to just a 
few terms; indeed, only pairs of basis functions $(\alpha,\beta)$
that are both nonvanishing at $\vb x$ contribute.
(For the RWG basis there are just 8 such pairs for each point,
independent of the total number of functions in the discretization.)

(Insert OPFT formulas here)
