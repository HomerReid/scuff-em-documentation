\documentclass[letterpaper]{article}

%********************************************************
%* latex shortcuts used for libscuff documentation files
%* 
%* homer reid -- 1994--2011
%**************************************************
\usepackage{graphicx}
\usepackage{color}
\usepackage{dsfont}
\usepackage{bbm}
\usepackage{amsmath}
\usepackage{amssymb}
\usepackage{float}
\usepackage{psfrag}
\usepackage{mathdots}
%\usepackage{algorithm}
%\usepackage{algorithmic}
%\usepackage{listings}
\usepackage{array}
\usepackage{url}
\usepackage{arydshln}
\usepackage{fancybox}
\usepackage{fancyvrb}
\usepackage{verbatim}

%--------------------------------------------------
%- boldface greek letters 
%--------------------------------------------------
\newcommand\vbphi{\mathbf{\phi}}
\newcommand\vbPhi{\mathbf{\Phi}}
\newcommand{\vbDelta}{\boldsymbol{\Delta}}
\newcommand{\vbrho}{\boldsymbol{\rho}}
\newcommand{\vbchi}{\boldsymbol{\chi}}

%--------------------------------------------------
%- colors -----------------------------------------
%--------------------------------------------------
\newcommand{\red}[1]{\textcolor{red}{#1}}
\newcommand{\blue}[1]{\textcolor{blue}{#1}}
\newcommand{\green}[1]{\textcolor{green}{#1}}
\newcommand{\atan}{\text{atan}}

%--------------------------------------------------
% general commands
%--------------------------------------------------
\newcommand\Tr{\hbox{Tr }}
\newcommand\sups[1]{^{\hbox{\scriptsize{#1}}}}
\newcommand\supt[1]{^{\hbox{\tiny{#1}}}}
\newcommand\subs[1]{_{\hbox{\scriptsize{#1}}}}
\newcommand\subt[1]{_{\hbox{\tiny{#1}}}}
\newcommand{\nn}{\nonumber \\}
\newcommand{\vb}[1]{\mathbf{#1}}
\newcommand{\numeq}[2]{\begin{equation} #2 \label{#1} \end{equation}}
\newcommand{\pard}[2]{\frac{\partial #1}{\partial #2}}
\newcommand{\pardn}[3]{\frac{\partial^{#1} #2}{\partial #3^{#1}}}
\newcommand{\pf}[2]{\left(\frac{#1}{#2}\right)}
\newcommand{\pp}{{\prime\prime}}
\newcommand{\vbhat}[1]{\vb{\hat #1}}
\newcommand{\mc}[1]{\mathcal{#1}}
\newcommand{\bmc}[1]{\boldsymbol{\mathcal{#1}}}
\newcommand{\mb}[1]{\mathbb{#1}}
\newcommand{\ds}[1]{\displaystyle{#1}}
\newcommand{\primedsum}{\sideset{}{'}{\sum}}
\newcommand{\pan}{\mathcal{P}}

%--------------------------------------------------
%- special libscuff stuff--------------------------
%--------------------------------------------------
\newcommand\ls{{\sc libscuff}}
\newcommand\lss{{\sc libscuff\,}}
\newcommand{\BG}{\boldsymbol{\Gamma}}
\newcommand{\MInt}{\vb{\hat M}}
\newcommand{\NInt}{\vb{\hat N}}
\newcommand{\MExt}{\vb{\check M}}
\newcommand{\NExt}{\vb{\check N}}
\newcommand{\xInt}{\vb{\hat x}}
\newcommand{\xExt}{\vb{\check x}}

%--------------------------------------------------
%-- superscripts for dyadic green's functions 
%--------------------------------------------------
\newcommand{\EEe}{^{\hbox{\tiny{EE}\scriptsize{,$e$}}}}
\newcommand{\MEe}{^{\hbox{\tiny{ME}\scriptsize{,$e$}}}}
\newcommand{\EMe}{^{\hbox{\tiny{EM}\scriptsize{,$e$}}}}
\newcommand{\MMe}{^{\hbox{\tiny{MM}\scriptsize{,$e$}}}}
\newcommand{\EEr}{^{\hbox{\tiny{EE}\scriptsize{,$r$}}}}
\newcommand{\MEr}{^{\hbox{\tiny{ME}\scriptsize{,$r$}}}}
\newcommand{\EMr}{^{\hbox{\tiny{EM}\scriptsize{,$r$}}}}
\newcommand{\MMr}{^{\hbox{\tiny{MM}\scriptsize{,$r$}}}}
\newcommand{\incr}{^{\text{\scriptsize inc},r}}
\newcommand{\incro}{^{\text{\scriptsize inc},r_1}}
\newcommand{\incrt}{^{\text{\scriptsize inc},r_2}}

%%%%%%%%%%%%%%%%%%%%%%%%%%%%%%%%%%%%%%%%%%%%%%%%%%%%%%%%%%%%%%%%%%%%%%
%%%%%%%%%%%%%%%%%%%%%%%%%%%%%%%%%%%%%%%%%%%%%%%%%%%%%%%%%%%%%%%%%%%%%%
%%%%%%%%%%%%%%%%%%%%%%%%%%%%%%%%%%%%%%%%%%%%%%%%%%%%%%%%%%%%%%%%%%%%%%
\newcommand{\MMExt}{\check{\mathcal{M}}}
\newcommand{\MMInt}{\hat{\mathcal{M}}}
\newcommand{\NNExt}{\check{\mathcal{N}}}
\newcommand{\NNInt}{\hat{\mathcal{N}}}
\newcommand{\BMMExt}{\boldsymbol{\check{\mathcal{M}}}}
\newcommand{\BMMInt}{\boldsymbol{\hat{\mathcal{M}}}}
\newcommand{\BNNExt}{\boldsymbol{\check{\mathcal{N}}}}
\newcommand{\BNNInt}{\boldsymbol{\hat{\mathcal{N}}}}
\newpage

%--------------------------------------------------
%- inner products and operator matrix elements  ---
%--------------------------------------------------
\newcommand{\expval}[1]{ \big\langle #1 \big\rangle}
\newcommand{\Expval}[1]{ \Big\langle #1 \Big\rangle}
\newcommand{\inp}[2]{ \big\langle #1 \big| #2 \big\rangle}
\newcommand{\Inp}[2]{ \Big\langle #1 \Big| #2 \Big\rangle}
\newcommand{\INP}[2]{ \bigg\langle #1 \bigg| #2 \Big\rangle}
\newcommand{\vmv}[3]{ \big\langle #1 \big| #2 \big| #3 \big\rangle}
\newcommand{\VMV}[3]{ \Big\langle #1 \Big| #2 \Big| #3 \Big\rangle}

%------------------------------------------------------------
%- shaded box for source code inclusions 
%------------------------------------------------------------
%\definecolor{lightgrey}{rgb}{0.85,0.85,0.85}
%\newcommand{\SourceBox}[1]
% { \begin{mdframed}[backgroundcolor=lightgrey, linewidth=2pt,
%                    tikzsetting={draw=black, line width=2pt,
%                                 dashed, dash pattern= on 5pt off 5pt}] %
%  \begin{verbatim} #1 \end{verbatim} \end{mdframed} \medskip
%}
%\IfFileExists{mdframed.sty}
% { \usepackage{mdframed}
%   \surroundwithmdframed[backgroundcolor=lightgrey, linewidth=2pt,
%                         framemethod=tikz,
%                         skipabove=10pt,
%                         skipbelow=10pt,
%                         tikzsetting={draw=black, line width=2pt,
%                                      dashed, dash pattern= on 5pt off 5pt}
%                        ]{verbatim}
% }
% {
% }
%--------------------------------------------------
%- shaded verbatim for code inclusions
%--------------------------------------------------
\definecolor{lightgrey}{rgb}{0.85,0.85,0.85}
\newenvironment{verbcode}{\VerbatimEnvironment% 
  \noindent
  %{\columnwidth-\leftmargin-\rightmargin-2\fboxsep-2\fboxrule-4pt} 
  \begin{Sbox} 
  \begin{minipage}{\linewidth}
  \begin{Verbatim}
}{% 
  \end{Verbatim}  
   \end{minipage}
  \end{Sbox} 
  \fcolorbox{black}{lightgrey}{\textcolor{black}{\TheSbox}}
} 

\usepackage{mathtools}

\graphicspath{{figures/}}

%------------------------------------------------------------
%------------------------------------------------------------
%- Special commands for this document -----------------------
%------------------------------------------------------------
%------------------------------------------------------------
\newcommand{\vbsigma}{\boldsymbol{\sigma}}
\newcommand{\YY}{\mc Y}
\newcommand{\BE}{\begin{equation}}
\newcommand{\EE}{\end{equation}}
\newcommand{\wt}{\widetilde}
\newcommand{\wh}{\widehat}
\newcommand{\whmb}[1]{\widehat{\mb #1}}
\newcommand{\GTiab}{\wt{\mb G}_{i;ab}}
\newcommand{\CTiab}{\wt{\mb C}_{i;ab}}

\newcommand{\ImBig}[1]
 {\left[ \text{Im }\left(\vphantom{\Big(} #1 \right)\right]}
\newcommand{\ReBig}[1]
 {\left[ \text{Re }\left(\vphantom{\Big(} #1 \right)\right]}

%------------------------------------------------------------
%------------------------------------------------------------
%- Document header  -----------------------------------------
%------------------------------------------------------------
%------------------------------------------------------------
\title {Evaluation of four-dimensional integrals for
        matrix elements in {\sc scuff-em}}
\author {Homer Reid}
\date {November 24, 2015}

%------------------------------------------------------------
%------------------------------------------------------------
%- Start of actual document
%------------------------------------------------------------
%------------------------------------------------------------

\begin{document}
\pagestyle{myheadings}
\markright{Homer Reid: 4D integrals for {\sc scuff-em} matrix elements}
\maketitle 

\tableofcontents

%====================================================================%
%====================================================================%
%====================================================================%
\newpage
\section{BEM Matrix Elements}

\subsection{Edges, basis functions, and surface-current expansions}

{\sc scuff-em} approximates the electric and magnetic surface currents
$\vb K, \vb N$ on each surface in a geometry as an expansion in
a set of $N\subt{BF}$ basis functions. In six-vector notation we have
%====================================================================%
\begin{align*}
   \left(\begin{array}{c} \vb K(\vb x) \\ \vb N(\vb x) \end{array}\right)
&= \sum_{\alpha=1}^{N\subt{BF}} c_{\alpha} \bmc B_{\alpha}(\vb x)
\end{align*}
%====================================================================%
$$
\bmc B_{2a}(\vb x)=
 \left(\begin{array}{c} \vb b_a(\vb x) \\ 0 \end{array}\right),
\qquad
\bmc B_{2a+1}(\vb x)=
 \left(\begin{array}{c} 0 \\ \vb b_a(\vb x) \end{array}\right) \\
$$
%====================================================================%
\begin{align*}
  \left(\begin{array}{c} \vb K(\vb x) \\ \vb N(\vb x) \end{array}\right)
&= \sum_{a=1}^{N\subt{E}}
  \left(
  \begin{array}{r}      k_a \vb b_a (\vb x) \\
                        n_a \vb b_a (\vb x) 
  \end{array}\right)
\end{align*}
%====================================================================%
(The actual coefficients computed by {\sc scuff-em} are
$\hat{n}_a \equiv -Z_0 n_a$.)

Here $\alpha$ runs over all basis functions in the {\sc scuff-em}
calculation (of which there are $N\subt{BF}$ in total),
$a$ runs over all internal edges on all surfaces
(of which there are $N\subt{E}$ in total), and $\vb b_a(\vb x)$ 
is the RWG basis function associated with the $a$th interior
edge.


\subsection{BEM Matrix Elements}

If two basis functions $\bmc B_\alpha, \bmc B_\beta$ lie on
the same surface and this surface is a dielectric interface,
then the $(\alpha,\beta)$ element of the BEM matrix receives
contributions from both the media exterior and interior to the 
surface. Otherwise ($\bmc B_\alpha, \bmc B_\beta$ on the same
PEC surface or on different surfaces) there is only a contribution
from the exterior medium:\footnote{More accurately, there is only
a contribution from the medium common to the surfaces on 
which $\bmc B_\alpha, \bmc B_\beta$ lie. In the case of 
nested surfaces this will be the interior medium for one of the 
surfaces.}
%====================================================================%
$$
 M_{\alpha\beta}(\omega)
=\begin{dcases}
  M_{\alpha\beta}\sups{ext}(\omega)
 +M_{\alpha\beta}\sups{int}(\omega),
   \quad &\bmc B_\alpha, \bmc B_\beta \quad \text{on same (dielectric) surface}
\\
  M_{\alpha\beta}\sups{ext}(\omega)
   \quad &\bmc B_\alpha, \bmc B_\beta \quad \text{on same (PEC) surface or different surfaces}
 \end{dcases}
$$
%====================================================================%
Each pair of interior edges $(E_a,E_b)$ contributes a $2\times 2$ block
of matrix elements to the BEM matrix for medium $r$:
%====================================================================%
\numeq{MabrOmega}
{ M_{ab}^r(\omega)
 =i\frac{\omega}{c_0}
   \left(\begin{array}{cc}
       \mu_r \mb G_{ab}(k_r) & -n_r \mb C_{ab}(k_r) \\[5pt]
       -n_r \mb C_{ab}(k_r) & -\epsilon_r \mb G_{ab}(k_r)
   \end{array}\right)
}
%====================================================================%
Here $c_0$ is the vacuum speed of light, $\{\epsilon_r, \mu_r\}$ are 
the relative permittivity and permeability of medium $r$ at 
frequency $\omega$, and 
%====================================================================%
$$ n_r=\sqrt{\epsilon_r \mu_r}, \qquad k_r=n_r \frac{\omega}{c_0}.$$
%====================================================================%
The $\mb G, \mb C$ matrix elements and their $k$ derivatives are 
%====================================================================%
\begin{subequations}
\begin{align}
\mb G_{ab}(k)
&=
 \int \left(\vb b_a \cdot \vb b_b
            - \frac{ \left[\nabla \cdot \vb b_a\right]
                     \left[\nabla \cdot \vb b_b\right]
                   }{k^2}
      \right) G_0(k, \vb r) d^4 \vb r
\\
%====================================================================%
%--------------------------------------------------------------------%
\mb C_{ab}(k)
&=
 \frac{1}{ik}
 \int \left(\vb b_a \times \vb b_b \right) \cdot \nabla G_0(k,\vb r)
 \,d^4 \vb r
\end{align}
\label{GCab}
\end{subequations}
%====================================================================%
In these equations, I have 
%====================================================================%
\numeq{g0Def}
{ G_0(k,\vb r) = 
  \begin{dcases*} 
    \frac{e^{ikr}}{4\pi r}, \qquad &non-periodic
       \\[10pt]
    \sum_{\vb L} e^{i\vb k\subt{B}\cdot \vb L}
                   \frac{e^{ik|\vb r+\vb L|}}{4\pi |\vb r+\vb L|}
                 \qquad &Bloch-periodic with Bloch vector $\vb k\subt{B}$
  \end{dcases*} 
}
%====================================================================%

%%%%%%%%%%%%%%%%%%%%%%%%%%%%%%%%%%%%%%%%%%%%%%%%%%%%%%%%%%%%%%%%%%%%%%
%%%%%%%%%%%%%%%%%%%%%%%%%%%%%%%%%%%%%%%%%%%%%%%%%%%%%%%%%%%%%%%%%%%%%%
%%%%%%%%%%%%%%%%%%%%%%%%%%%%%%%%%%%%%%%%%%%%%%%%%%%%%%%%%%%%%%%%%%%%%%
\subsection*{Alternative notation}

Equations (\ref{MabrOmega}) and (\ref{GCab}) are the way I have
always defined things in {\sc scuff-em}; among their advantages
is the fact that $\mb C$ has the same units as $\mb G$ 
(namely, inverse length). However, for present purposes it is 
actually convenient to write (\ref{MabrOmega}) in the slightly different
form
%====================================================================%
\numeq{MabrOmega2}
{ M_{ab}^r(\omega)
 = \left(\begin{array}{cc}
       \frac{i\omega\mu_r }{c} \mb G_{ab}(k_r) & -\wh{\mb C}_{ab}(k_r) \\[5pt]
       -\wh{\mb C}_{ab}(k_r) & -\frac{i\omega \epsilon_r}{c} \mb G_{ab}(k_r)
   \end{array}\right)
}
%====================================================================%
where 
%====================================================================%
\numeq{whCab}
{ \wh{\mb C}(k)\equiv ik \mb C(k), 
  \qquad \wh{\mb C}_{ab}
   =
   \int \left(\vb b_a \times \vb b_b \right) \cdot \nabla G_0(k,\vb r)
   \,d^4 \vb r.
}
%====================================================================%

%%%%%%%%%%%%%%%%%%%%%%%%%%%%%%%%%%%%%%%%%%%%%%%%%%%%%%%%%%%%%%%%%%%%%%
%%%%%%%%%%%%%%%%%%%%%%%%%%%%%%%%%%%%%%%%%%%%%%%%%%%%%%%%%%%%%%%%%%%%%%
%%%%%%%%%%%%%%%%%%%%%%%%%%%%%%%%%%%%%%%%%%%%%%%%%%%%%%%%%%%%%%%%%%%%%%
\newpage
\section{Force and torque integrals}

\subsection{Force}
%%%%%%%%%%%%%%%%%%%%%%%%%%%%%%%%%%%%%%%%%%%%%%%%%%%%%%%%%%%%%%%%%%%%%%
\begin{align*}
F_i&= \frac{1}{2\omega} \text{Im }\int
 \left(\begin{array}{c} \vb K(\vb x) \\ \vb N(\vb x)\end{array}\right)^\dagger
 \left(\begin{array}{c} \partial_i \vb E(\vb x) \\ 
                        \partial_i \vb H(\vb x)\end{array}\right) \,d\vb x
\\
&= \frac{1}{2\omega} \text{Im }\int
 \left(\begin{array}{c} \vb K(\vb x) \\ 
                        \vb N(\vb x)
       \end{array}\right)^\dagger
%====================================================================%
 \left(\begin{array}{cc}  i\omega\mu_r  \partial_i \mb G(\vb x, \vb x^\prime)
                         &\partial_i \whmb C(\vb x, \vb x^\prime) \\
                         -\partial_i \whmb C(\vb x, \vb x^\prime)
                         &i\omega \epsilon_r \partial_i\mb G(\vb x, \vb x^\prime)
       \end{array}\right)
%====================================================================%
 \left( \begin{array}{c} \vb K(\vb x^\prime) \\
                         \vb N(\vb x^\prime)
        \end{array}\right)
\,d\vb x \,d\vb x^\prime
\\
%====================================================================%
&=\frac{1}{2\omega}\text{Im }\sum_{ab}
 \left(\begin{array}{c} k_a \\ n_a \end{array}\right)^\dagger
%====================================================================%
 \left(\begin{array}{cc}  i\omega \mu_r       \partial_i \mb G_{ab}
                         &\partial_i \whmb C_{ab} \\
                         -\partial_i \whmb C_{ab}
                         &i\omega\epsilon_r \partial_i \mb G_{ab}
       \end{array}\right)
%====================================================================%
 \left(\begin{array}{c} k_b \\ n_b \end{array}\right)
\\
&=-\sum_{b>a}
 \Bigg\{
      \frac{Z_0}{c}
      \ImBig{k_a^* k_b}
      \ImBig{\mu_r \mb \partial_i \mb G_{ab}}
  \,+\,
      \frac{1}{cZ_0}
      \ImBig{n_a^* n_b}
      \ImBig{\epsilon_r \mb \partial_i \mb G_{ab}}
\\
&\qquad\qquad
     +\frac{1}{\omega}
       \ReBig{k_a^* n_b - n_a^* k_b} 
       \ImBig{\partial_i \whmb C_{ab}}
 \Bigg\}
\end{align*}
%%%%%%%%%%%%%%%%%%%%%%%%%%%%%%%%%%%%%%%%%%%%%%%%%%%%%%%%%%%%%%%%%%%%%%

%=================================================
%=================================================
%=================================================
\subsection{Torque}
%%%%%%%%%%%%%%%%%%%%%%%%%%%%%%%%%%%%%%%%%%%%%%%%%%%%%%%%%%%%%%%%%%%%%%
\begin{align*}
\mc T_i
 &=
\mc T_i^{(1)} + \mc T_i^{(2)}
\\
\mc T_i^{(1)}
&=\frac{1}{2\omega} \text{Im }\int
 \left(\begin{array}{c} \vb K(\vb x) \\ \vb N(\vb x)\end{array}\right)^\dagger
 \left(\begin{array}{c} \partial_{\theta_i} \vb E(\vb x) \\ 
                        \partial_{\theta_i} \vb H(\vb x)\end{array}\right) \,d\vb x
\\
&=-\sum_{b>a}
 \Bigg\{
      \frac{Z_0}{c}
      \ImBig{k_a^* k_b}
      \ImBig{\mu_r \mb \partial_{\theta_i} \mb G_{ab}}
  \,+\,
      \frac{1}{cZ_0}
      \ImBig{n_a^* n_b}
      \ImBig{\epsilon_r \mb \partial_{\theta_i} \mb G_{ab}}
\\
&\qquad\qquad
    -\frac{1}{\omega}
     \ReBig{k_a^* n_b - n_a^* k_b} 
     \ImBig{\partial_{\theta_i} \whmb C_{ab}}
 \Bigg\}
\\[12pt]
%--------------------------------------------------------------------%
\mc T_i^{(2)}
&=\frac{1}{2\omega} \text{Im }\int
  \bigg\{ \vb K^* \times \vb E + \vb N^* \times \vb H \bigg\}_i
  \,d\vb x
\\
&=\frac{1}{2\omega} \varepsilon_{ijk}\text{Im }\iint
  \bigg\{  K^*_j \Big[ i\omega \mu_r \mb G_{k\ell}\Big] K_\ell
          +K^*_j \Big[ \wh{\mb C}_{k\ell}\Big] N_\ell
\\
&\qquad\qquad\qquad\qquad
          -N^*_j \Big[ \wh{\mb C}_{k\ell}\Big] K_\ell
          +N^*_j \Big[ i\omega \epsilon_r \mb G_{k\ell}\Big] 
           N_\ell\bigg\}
  \,d^4 \vb x
\\
&=-\sum_{b>a}
 \Bigg\{
      \frac{Z_0}{c}
      \ImBig{k_a^* k_b} \ImBig{\mu_r \GTiab}
 \,+\,
      \frac{1}{c Z_0}
      \ImBig{n_a^* n_b} \ImBig{\epsilon_r \GTiab}
\\
&\qquad\quad\qquad
     - \frac{1}{2\omega} \ReBig{k_a^* n_b - n_a^* k_b}
       \ImBig{\CTiab}
 \Bigg\}
\end{align*}
%====================================================================%

%====================================================================%
$$
 \GTiab
 \equiv \varepsilon_{ijk} \iint b_{aj} \mb G_{k\ell} b_{b\ell} \, d\vb r,
 \qquad 
 \CTiab
 \equiv \varepsilon_{ijk} \iint b_{aj} \whmb C_{k\ell} b_{b\ell} \, d\vb r
$$

%%--------------------------------------------------------------------%
%&=\varepsilon_{ijk} \iint b_{aj}
% \Big[\delta_{k\ell}G_0 + k^2 \partial_k \partial_\ell G_0\Big]
%ji%  b_{b\ell} \, d\vb r
%%%%%%%%%\\
%--------------------------------------------------------------------%
%--------------------------------------------------------------------%
%\\
%&=\underbrace{\varepsilon_{ijk} \varepsilon_{k\ell m}}
%            _{\delta_{i\ell} \delta_{jm} - \delta_{im}\delta_{j\ell}}
% \iint b_{aj} b_{b\ell} \partial_m G_0 \, d\vb r
%\\
%&= \iint 
%   \Big\{ \Big(\vb b_a \cdot \nabla G_0 \Big) b_{bi} - 
%          \Big(\vb b_a \cdot \vb b_b\Big) \partial_i G_0
%   \Big\}\,d\vb r
%\end{align*}
%====================================================================%

%--------------------------------------------------------------------%
%
%&=\frac{1}{2\omega} \varepsilon_{ijk}\text{Im }\int
% \left(\begin{array}{c} K_j(\vb x) \\ N_j(\vb x)\end{array}\right)^\dagger
% \left(\begin{array}{c} E_k(\vb x) \\ 
%                        H_k(\vb x)\end{array}\right) \,d\vb x
%\\
%%--------------------------------------------------------------------%
%&=\frac{1}{2} \varepsilon_{ijk}\text{Im }\int
% \left(\begin{array}{c} K_j(\vb x) \\ 
%                        N_j(\vb x)
%       \end{array}\right)^\dagger
%%====================================================================%
% \left(\begin{array}{cc}  i\mu_r       \mb G_{k\ell}(\vb x, \vb x^\prime)
%                         &i n_r        \mb C_{k\ell}(\vb x, \vb x^\prime) \\
%                         -i n_r        \mb C_{k\ell}(\vb x, \vb x^\prime)
%                         &i \epsilon_r \mb G_{k\ell}(\vb x, \vb x^\prime)
%       \end{array}\right)
%%====================================================================%
% \left( \begin{array}{c} K_\ell(\vb x^\prime) \\
%                         N_\ell(\vb x^\prime)
%        \end{array}\right)
%\,d\vb x \,d\vb x^\prime
%\\
%&=\frac{1}{2}\text{Im }\sum_{ab}
% \left(\begin{array}{c} k_a \\ n_a \end{array}\right)^\dagger
%%====================================================================%
% \left(\begin{array}{cc}  i\mu_r       \wt{\mb G}_{ab}
%                         &i n_r        \wt{\mb C}_{ab} \\
%                         -i n_r        \wt{\mb C}_{ab}
%                         &i \epsilon_r \wt{\mb G}_{ab}
%       \end{array}\right)
%%====================================================================%
% \left(\begin{array}{c} k_b \\ n_b \end{array}\right)
%\\
%&=-\sum_{b>a}
% \Bigg\{
%      Z_0
%      \ImBig{k_a^* k_b}
%      \ImBig{\mu_r \wt{\mb G}_{ab}}
% \\
%&\hphantom{ -\sum_{b>a} \Bigg\{ }
%     +\frac{1}{Z_0}
%      \ImBig{n_a^* n_b}
%      \ImBig{\epsilon_r \wt{\mb G}_{ab}}
%\\
%&\hphantom{ -\sum_{b>a} \Bigg\{ }
%     - \ReBig{k_a^* n_b - n_a^* k_b} 
%       \ReBig{n_r \wt{\mb C}_{ab}}
% \Bigg\}
%\end{align*}
%
%%%%%%%%%%%%%%%%%%%%%%%%%%%%%%%%%%%%%%%%%%%%%%%%%%%%%%%%%%%%%%%%%%%%%%
%%%%%%%%%%%%%%%%%%%%%%%%%%%%%%%%%%%%%%%%%%%%%%%%%%%%%%%%%%%%%%%%%%%%%%
%%%%%%%%%%%%%%%%%%%%%%%%%%%%%%%%%%%%%%%%%%%%%%%%%%%%%%%%%%%%%%%%%%%%%%
\newpage
\section{Frequency derivatives}

The $\omega$ derivative of (\ref{MabrOmega2}) reads
%====================================================================%
\begin{align*}
\frac{d}{d\omega}
 M_{ab}^r
 &= \frac{i}{c}\left(\begin{array}{cc}
    \big(\omega \mu_r\big)^\prime \mb G_{ab}(k_r) & 0
    \\[5pt]
    0 & \big(\omega \epsilon_r\big)^\prime \mb G_{ab}(k_r)
    \end{array}\right)
\\
&\qquad
  +\left(\begin{array}{cc}
       \frac{i\omega\mu_r }{c} \mb G^\prime_{ab}(k_r) 
      &-\wh{\mb C}^\prime_{ab}(k_r) 
      \\[5pt]
       -\wh{\mb C}^\prime_{ab}(k_r) 
      & -\frac{i\omega \epsilon_r}{c} \mb G_{ab}(k_r)
   \end{array}\right)
\end{align*}
where 
$$ \big(\omega \mu_r\big)^\prime 
    = \mu_r + \omega\frac{d\mu_r}{d\omega},   
   \qquad 
   \big(\omega \epsilon_r\big)^\prime 
   = \epsilon_r + \omega\frac{d\epsilon_r}{d\omega},
$$
%====================================================================%
and primes on $\mb G$ and $\wh{\mb C}$ denote differentiation with 
respect to $k$. 

The $k$ derivatives of the $\mb G, \wh{\mb C}$ matrix elements are
%====================================================================%
\begin{subequations}
\begin{align}
 \mb G^\prime_{ab}(k)
&= \frac{2}{k^3}
   \int \left[\nabla \cdot \vb b_a\right] 
        \left[\nabla \cdot \vb b_b\right] G_0(k,\vb r)
   \, d^4 \vb r
\\
%--------------------------------------------------------------------%
&\qquad
 +
 \int \left(\vb b_a \cdot \vb b_b 
            - \frac{ \left[\nabla \cdot \vb b_a\right]
                     \left[\nabla \cdot \vb b_b\right] 
                   }{k^2}
      \right) G_0^\prime(k,\vb r)\, d^4\vb r
\\
%--------------------------------------------------------------------%
\mb C^\prime_{ab}(k)
&= \int \left(\vb b_a \times \vb b_b \right) \cdot \nabla G_0^\prime(k,\vb r)
   \,d^4 \vb r
\end{align}
\label{dGCabdOmega}
\end{subequations}
%====================================================================%
In both the periodic and non-periodic cases, $k$ derivatives of $G_0$
may be related to spatial derivatives according to
%====================================================================%
\begin{align}
 \pard{}{k} G_0
 &= -i|\vb r|^2 \left( \frac{\vb r \cdot \nabla G_0}{|\vb r|} - ik G_0\right)
\\
 \pard{}{k} \nabla G_0 
&= -k \vb r G_0
\label{NonsingularKernels}
\end{align}
%====================================================================%
Importantly, the kernels defined by (\ref{NonsingularKernels}) are
both \textit{nonsingular} at $\vb r=0$, allowing the use of simple
numerical cubature to evaluate matrix elements.

%%%%%%%%%%%%%%%%%%%%%%%%%%%%%%%%%%%%%%%%%%%%%%%%%%%%%%%%%%%%%%%%%%%%%%
%%%%%%%%%%%%%%%%%%%%%%%%%%%%%%%%%%%%%%%%%%%%%%%%%%%%%%%%%%%%%%%%%%%%%%
%%%%%%%%%%%%%%%%%%%%%%%%%%%%%%%%%%%%%%%%%%%%%%%%%%%%%%%%%%%%%%%%%%%%%%
\newpage
\section{Computation of matrix elements}

\subsection{BEM matrix elements}

%====================================================================%
\begin{align}
 \mb G_{ab} 
&=
\iint \left( \vb b_a \cdot \vb b_b - \frac{4}{k^2}\right) G_0(r) \, d^4 \vb r
\\
&=
 \mb C_{ab} &=
 \frac{1}{ik}
 \varepsilon_{ijk}
\iint b_{ai} b_{bj} \partial_k G_0(r) \, d^4 \vb r
\end{align}
%====================================================================%

\subsection{Force integrals}

%====================================================================%
\begin{align}
 \partial_i \mb G_{ab} &=
\iint 
 \underbrace{\left( \vb b_a \cdot \vb b_b - \frac{4}{k^2}\right)}
           _{P\supt{EFIE}}
      \partial_i G_0(r) \, d^4 \vb r
\\[10pt]
 &=
 \sum_p \overline{\psi}_p
 \iint P\supt{EFIE} r_i r^p \, d\vb r
+
 \iint P\supt{EFIE} r_i \psi\supt{DS}(r) \, d^4 \vb r
\\[5pt]
 \text{Im } \partial_i \whmb C_{ab} 
&=
 \varepsilon_{jk\ell}
 \iint b_{aj} b_{bk} 
 \Big[
 \text{Im }\partial_i \partial_\ell G_0(r) 
 \Big] \, d \vb r
\end{align}
%====================================================================%

%=================================================
%=================================================
%=================================================
\subsection{Torque integrals}

%====================================================================%
\begin{align}
 \partial_{\theta_i} \mb G_{ab} 
&=
 \varepsilon_{ijk}
 \iint \left( \vb b_a \cdot \vb b_b - \frac{4}{k^2}\right)
       (\vb x_a-\vb x_0)_j \partial_k G_0(r) \, d^4 \vb r
\\
 \partial_{\theta_i} \mb C_{ab} &=
 \frac{1}{ik}
 \varepsilon_{ijk}
 \varepsilon_{jk\ell}
 \iint b_{aj} b_{bk} 
  (\vb x_a - \vb x_0)_j
  \partial_i \partial_\ell G_0(r) \, d^4\vb r
\end{align}
%====================================================================%

%====================================================================%
\begin{align*}
 \wt{\mb G}_{i;ab} 
&\equiv
   \varepsilon_{ijk} 
   \int b_{aj}(\vb x) 
        \underbrace{\mb G_{k\ell}(\vb x, \vb x^\prime)}
                  _{\left(   \delta_{k\ell} 
                            +\frac{1}{k^2}\partial_k \partial_\ell
                    \right)G_0}
        b_{b\ell}(\vb x^\prime) \, d\vb x \, d\vb x^\prime
\\
&=
   \int \left\{\left(\vb b_a \times \vb b_b\right)_i 
               \left(G_0 + \frac{1}{k^2}\psi\right)
               +\Big[\left(\vb b_a \times \vb r\right)_i
                     \left(\vb b_b \cdot \vb r\right)
                \Big]\zeta(r)
        \right\}\,d\vb x \, d\vb x^\prime
\\
 \wt{\mb C}_{i;ab} 
&\equiv
   \varepsilon_{ijk} 
   \int b_{aj}(\vb x) 
        \underbrace{\whmb C_{k\ell}(\vb x, \vb x^\prime)}
                  _{\epsilon_{k\ell m} r_m \psi(r)}
        b_{b\ell}(\vb x^\prime) \, d\vb x \, d\vb x^\prime
\\
&=
   \int\left\{ \Big( \vb b_a \cdot \vb r\Big) b_{bi}
              -\Big( \vb b_a \cdot \vb b_b\Big) r_i
       \right\} \psi(r) \, d\vb x \, d\vb x^\prime
\\
\end{align*}
%====================================================================%

$$\partial_i G_0(r) = r_i \psi(r) $$
$$\partial_i \partial_j G_0(r) = \delta_{ij}\psi(r) + r_i r_j \zeta(r) $$
$$\psi(r) = (ikr-1)\frac{e^{ikr}}{4\pi r^3}$$
$$\zeta(r) = \Big[ (ikr)^2 - 3ikr + 3\Big] \frac{e^{ikr}}{4\pi r^5}$$

%%%%%%%%%%%%%%%%%%%%%%%%%%%%%%%%%%%%%%%%%%%%%%%%%%%%%%%%%%%%%%%%%%%%%%
%%%%%%%%%%%%%%%%%%%%%%%%%%%%%%%%%%%%%%%%%%%%%%%%%%%%%%%%%%%%%%%%%%%%%%
%%%%%%%%%%%%%%%%%%%%%%%%%%%%%%%%%%%%%%%%%%%%%%%%%%%%%%%%%%%%%%%%%%%%%%
\appendix
\newpage
\section{Scalar and dyadic Green's functions and their most singular terms}

\subsection{Scalar GF}

%====================================================================%
\begin{align*}
  G_0(r) &= \frac{e^{ikr}}{4\pi r}
\\
 \partial_i G_0(r) &= 
  r_i \psi(r), \qquad \psi(r)\equiv (ikr-1)\frac{e^{ikr}}{4\pi r^3} 
\\
 \partial_i \partial_j G_0(r) &=
  \delta_{ij} \psi(r) + r_i r_j \zeta(r), \qquad
  \zeta(r)\equiv \frac{e^{ikr}}{4\pi r^5}\Big[3 - 3ikr + (ikr)^2\Big]
\end{align*}
%====================================================================%

\subsection{Desingularized scalar GF}
%====================================================================%
\begin{align*}
  G_0(r) &= \frac{1}{4\pi r} + \frac{\texttt{ExpRel}(ikr,1)}{4\pi r}
\\[14pt]
%--------------------------------------------------------------------%
 \partial_i G_0(r) &=
  r_i \psi\supt{S}(r)
  +
  r_i \psi\supt{DS}(r),
\\
 \psi\supt{DS}(r) &= \frac{\texttt{ExpRel}(ikr,3)}{4\pi r^3} 
\\
 \psi\supt{S}(r) &= \sum_{p} C_p r^p,
 \qquad
 C_{-3}=-\frac{1}{4\pi}, 
 \qquad 
 C_{-1} = -\frac{k^2}{8\pi},
 \qquad
 C_{0}=-\frac{ik^3}{8\pi}
\\[14pt]
%--------------------------------------------------------------------%
\text{Im } \partial_i \partial_j G_0(r)
&=\text{nonsingular}
\end{align*}
%====================================================================%

\subsection{Dyadic GFs}

%====================================================================%
\numeq{GCComponents}
{
 \mb G_{ij}(\vb r)
  = \frac{e^{ikr}}{4\pi (ik)^2 r^3}
    \left[ F_1(ikr) \delta_{ij} + F_2(ikr) \frac{r_i r_j}{r^2}\right],
\qquad
C_{ij}(k,\vb r) 
  = \frac{e^{ikr}}{4\pi (ik) r^3} \varepsilon_{ijk} r_k F_3(ikr),
}
$$ F_1(x) = 1-x+x^2, 
   \qquad 
   F_2(x) = -3 + 3x - x^2, 
   \qquad 
   F_3(x)=-1+x.
$$
%====================================================================%

\subsection*{Desingularized DGFs}

$$   \mb G_{ij}(\vb r)
   = \mb G_{ij}\supt{S}(\vb r)
    +\mb G_{ij}\supt{DS}(\vb r)
$$

$$ \mb G\supt{DS}_{ij}(\vb r)
   =
    \frac{\texttt{ExpRel}(ikr,3)}{4\pi (ik)^2 r^3}
    \left[ F_1(ikr) \delta_{ij} + F_2(ikr) \frac{r_i r_j}{r^2}\right],
  \qquad 
   \mb C\supt{DS}_{ij}(\vb r)
   =
    \frac{\texttt{ExpRel}(ikr,3)}{4\pi (ik) r^3} \varepsilon_{ijk}r_k
$$

$$ \mb G\supt{S}_{ij}(\vb r)
   = \Big[ \sum_p \Upsilon_p^1 r^p \Big] \delta_{ij} 
    +\Big[ \sum_p \Upsilon_p^2 r^p \Big] \frac{r_i r_j}{r^2}
$$

$$ \mb C\supt{S}_{ij}(\vb r)
   = \Big[ \sum_p \Upsilon_p^3 r^p \Big] \varepsilon_{ijk} r_k
$$

$$
 \Upsilon_{-3}^{1} = -\frac{1}{4\pi k^2},      \qquad 
 \Upsilon_{-1}^{1} = +\frac{1}{8\pi},          \qquad 
 \Upsilon_{ 0}^{1} = +\frac{ik}{8\pi},         \qquad 
 \Upsilon_{ 1}^{1} = -\frac{k^2}{8\pi},
$$

$$
 \Upsilon_{-3}^{2} = +\frac{3}{4 \pi k^2},      \qquad 
 \Upsilon_{-1}^{2} = +\frac{1}{8\pi},          \qquad 
 \Upsilon_{ 0}^{2} = +\frac{ik}{8\pi},         \qquad 
 \Upsilon_{ 1}^{2} = +\frac{k^2}{8\pi},
$$

$$
 \Upsilon_{-3}^{3} = +\frac{i}{4\pi k},  \qquad
 \Upsilon_{-1}^{3} = +\frac{ik}{8\pi},   \qquad
 \Upsilon_{ 0}^{3} = -\frac{k^2}{8\pi},  \qquad 
 \Upsilon_{ 1}^{3} = 0.
$$

\end{document}
